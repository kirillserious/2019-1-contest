\section{Принцип максимума Л.~С.~Понтрягина}

В этом разделе мы сформулируем принцип максимума и условия трансверсальности как необходимые условия существования оптимального управления для линейных систем. Вернёмся к заданной линейной задаче быстродействия \eqref{eq:main_system}:
$$
        \dot x(t) = A(t)x(t) + B(t)u(t),\quad t \in [t_0,\,T],
$$
$$
        x(t_0) \in \X_0,\; x(T) \in \X_1,
$$
$$
        u(t) \in \Omega,
$$
$$
        T - t_0 \rightarrow \inf.
$$

\begin{definition}
        Пара $(x^*,\,u^*)$ называется \textit{оптимальной} для задачи \eqref{eq:main_system}, если она удовлетворяет её условиям.
\end{definition}

\begin{theorem}[Л.~С.~Понтрягин]
        Пусть $(x^*,\,u^*)$~--- оптимальная пара на $[t_0,\,T]$, являющаяся решением задачи быстродействия. Тогда существует функция $\psi(t)$, определенная при $t \geqslant t_0$, являющаяся решением сопряженной системы
        $$
                \begin{cases}
                        \dot\psi(t) = -A^\T(t)\psi(t)\\
                        \psi(t_0) = \psi_0 \neq 0
                \end{cases}
        $$ 
        такая, что выполнены следующие условия:

        \begin{itemize}
                \item Принцип максимума
                $$
                        \langle Bu^*(t),\,\psi(t)\rangle = \rho(\psi(t)\,|\,B\Omega),
                $$
                для вех $t \geqslant t_0$, кроме точек разрыва функции $u^*(t)$.

                \item Условие трансверсальности на левом конце:
                $$
                        \langle \psi(t_0),\,x^*(t_0)\rangle = \rho(\psi(t_0)\,|\,\X_0),
                $$
                то есть $x^*(t_0)$ есть опорный вектор множества $\X_0$ в направлении $\psi(t_0)$.

                \item Условие трансверсальности на правом конце:
                $$
                        \langle -\psi(T),\,x^*(T)\rangle = \rho(-\psi(T)\,|\,\X_1),
                $$
                то есть $x^*(t_1)$ есть опорный вектор множества $\X_1$ в направлении $\psi(T)$.
        \end{itemize}
\end{theorem}