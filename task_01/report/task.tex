\section{Постановка задачи}

Задана линейная система обыкновенных дифференциальных уравнений:
\begin{equation}\label{eq:main_system}
        \dot x(t) = A(t)x + B(t)u(t), \quad t \in [t_0,\,+\infty).
\end{equation}
Здесь $x(t),\,u(t) \in \R^2$, $A(t),\,B(t) \in \R^{2 \times 2}$.
На значения управляющих параметров $u$ наложено ограничение: $u(t) \in \Omega$.
Пусть $\X_0$~--- начальное множество значений фазового вектора, $\X_1$~--- целевое множество значений фазового вектора.
Необходимо решить задачу быстродействия, то есть найти минимальное время $T > 0$, за которое траектория системы \eqref{eq:main_system}, выпущенная в момент времени $t_0$ из некоторой точки множества $\X_0$, может попасть в некоторую точку множества $\X_1$.
$$
        \Omega = \{\,x = (x_1,\,x_2)^\T \in \R^2 \;:\; ax_1^2 \leqslant x_2 \leqslant b - cx_1^2\,\},\quad a,\,b,c > 0;
$$ 
$$
        \mbox{$\X_0$~--- шар радиуса $r > 0$ с центром в точке $x_0$,}
$$
$$
        \X_1 = \{\,x \in \R^2 \;:\; Gx + g \leqslant 0 \,\},
        \quad G \in \R^{m \times 2},\,
        g \in \R^m,\,
        m \in \N.
$$
\begin{enumerate}
        \item Необходимо написать в среде MatLab программу с пользовательским интерфейсом, которая по заданным параметрам $A(\cdot)$, $B(\cdot)$, $t_0$, $a$, $b$, $c$, $x_0$, $x_1$, $r$, $m$, $G$, $g$ определяет, разрешима ли задача быстродействия.
        Если задача разрешима, то программа должна (приближённо) найти значение $T$, построить графики компонент оптимального управления, компонент оптимальной траектории, сопряжённых переменных.
        Программа должна рассчитывать погрешность выполнения условий трансверсальности для найденной ,,оптимальной‘‘ траектории.
        Программа должна давать пользователю возможность постепенно улучшать результаты расчетов за счёт изменения параметров численного метода и анализа получающихся приближённых результатов.

        \item В соответствующем заданию отчёте необходимо привести все теоретические выкладки, сделанные в ходе решения задачи оптимального управления, привести примеры построенных оптимальных управлений и траекторий (с иллюстрациями) для различных параметров системы (обязательно для различных собственных значений матрицы $A$).
        Необходимо также исследовать на непрерывность величину $T$ по начальному (целевому) множеству фазовых переменных.
\end{enumerate}