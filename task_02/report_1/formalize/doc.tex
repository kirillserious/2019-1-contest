\section{Формализация задачи}

\subsection{Принцип максимума как необходимое условие оптимальности}

В этом разделе мы сформулируем принцип максимума для общего случая нелинейных автономных управляемых систем в задаче с подвижными концами на конечном или бесконечном интервале времени. Принцип максимума, вместе с условиями трансверсальности, является необходимым условием, которому должно удовлетворять оптимальное управление. Доказательство этих результатов можно посмотреть в \cite{li}[Гл.~5].
        
Итак, рассмотрим автономный управляемый процесс
$$
        \dot x = f(x, u),
$$
с непрерывными $f(x, u)$ и $\frac{\partial f}{\partial x}(x, u)$ в пространстве $\R^{n+m}$. Пусть $X_0$ и $X_1$ $\subset \R^n$ есть заданные начальное и целевое множества и пусть $\Omega$ есть непустое ограничивающее множество в $\R^m$. Допустимое управление $u(t) \in \Omega$ на некотором конечном интервале времени $0 \leqslant t \leqslant t_1$ есть ограниченная измеримая функция, которой соответствует траектория $x(t, x^0)$, переводящая точку $x(0, x^0) = x^0 \in X_0$ в точку $x(t_1, x^0) = x^1 \in X_1$. Конечный момент времени $t_1$, начальная точка $x^0 \in X_0$ и конечная точка $x^1 \in X_1$ меняются вместе с управлением. Класс всех допустимых управлений обозначим через $\Delta$.
        
Каждому управлению $u(t)$ ($0 \leqslant t \leqslant t_1$) в $\Delta$ с траекторией $x(t)$ поставим в соответствие критерий качества
$$
        J(u) = \int\limits_0^{t_1} f_0(x(t), u(t)) dt,
$$
где $f_0(x, u)$ и $\frac{\partial f_0}{\partial x}(x, u)$ --- непрерывные в $\R^{n+m}$ функции. Допустимое управление $u^*(t)$ из $\Delta$ является (минимальным) оптимальным, если
$$
        J(u^*) \leqslant J(u) \mbox{ для всех } u \in \Delta.
$$
Согласно \cite{li}[Гл.~5] оптимальное управление $u^*(t)$ на интервале $0 \leqslant t \leqslant t_1$ удовлетворяет принципу максимума
$$
        \hat{H}(\hat{\psi}^*(t), \hat{x}^*(t), u^*(t)) = \hat{M}(\hat{\psi}^*(t), \hat{x}^*(t)) \mbox{ почти всюду}
$$
и
$$
        \hat{M}(\hat{\psi}^*(t), \hat{x}^*(t)) \equiv 0,\; \hat{\psi}^*_0(t) \leqslant 0 \mbox{ всюду.}
$$
Здесь расширенное состояние
$$
        \hat{x}^*(t) =
        \left[
        \begin{aligned}
                &x_0^* (t) \\
                &x^*   (t)
        \end{aligned}
        \right]
$$
есть решение расширенной системы уравнений
$$
        \begin{aligned}
                &\dot x_0 = f_0(x, u),&\\
                &\dot x_i = f_i(x, u),  &i = 1, \ldots, n,
        \end{aligned}
$$
а $\hat\psi^*(t)$ --- нетривиальное решение расширенной сопряженной системы уравнений
$$
        \begin{aligned}
                &\dot\psi_0 = 0, \\
                &\dot\psi_i = - \sum\limits_{j=0}^{n}\frac{\partial f_j}{\partial x_i}(x^*(t),\: u^*(t)), & i = 1,\: \ldots ,\: n,
        \end{aligned}
$$
где последние $n$ уравнений (с $f_0 \equiv 0$) образуют сопряженную систему. Расширенная функция Гамильтона--Понтрягина имеет вид
$$
        \hat H (\hat\psi,\: \hat x,\: u) = \psi_0 \cdot f_0(x,\: u) + \psi_1 \cdot f_1(x,\: u) + \ldots + \psi_n \cdot f_n(x,\: u)
$$
и
$$
        \hat M (\hat\psi,\: \hat x) = \max\limits_{u \in \Omega}\hat H (\hat\psi,\:\hat x,\: u) \mbox{ (если $\hat M$ существует).}
$$

%%%%%%%%%%%%%%%%%%%%%%%%%%%%%%%%%
%%%%%%%%%%%%%%%%%%%%%%%%%%%%%%%%%
%%%%%%%%%%%%%%%%%%%%%%%%%%%%%%%%%
%%%%%%%%%%%%%%%%%%%%%%%%%%%%%%%%%
%%%%%%%%%%%%%%%%%%%%%%%%%%%%%%%%%
\subsection{Формализация задачи}

Сначала для того, чтобы сделать все наши дальнейшие формулы и соотношения компактнее, сделаем замену следующих переменных:
$$
        v(t) = [v_1(t),\,v_2(t)]^\T = [u_1(t),\, u_2(t) + 1]^\T,
        \qquad
        m_1 = k_1 + 1,
        \qquad
        m_2 = k_2 + 1. 
$$

Теперь перепишем рассматриваемое дифференциальное уравнение (\ref{eq:main_equation}) в виде системы, построим для нее расширенную функцию Гамильтона--Понтрягина, а также сопряженную систему.

\begin{equation} \label{eq:main_system}
        \left\{
        \begin{aligned}
                \dot x_1 &= x_2 \\
                \dot x_2 &= v_1 - v_2x_2.
        \end{aligned}
        \right.
\end{equation}

\begin{equation} \label{eq:hamilton_function}
        \hat H (\hat\psi,\: \hat x,\: u) = \psi_0\cdot(v_1^2 + \alpha |v_1|) + \psi_1\cdot x_2 + \psi_2\cdot(v_1 - v_2x_2) \mbox{, где $\psi_0 \leqslant 0.$}
\end{equation} 
        
\begin{equation}\label{eq:auxiliary_system}
        \left\{
        \begin{aligned}
                &\dot\psi_1 = 0\\
                &\dot\psi_2 = -\psi_1 + v_2\psi_2.
        \end{aligned}
        \right.
\end{equation}

\begin{remark}
        Из принципа максимума следует, что для любой оптимальной траектории системы~(\ref{eq:main_system}) $\psi_1(t) \equiv \psi_1^0 \equiv \const$.
\end{remark}

\begin{remark}
        Покажем, что задача разрешима только при условии, что $\psi_0 < 0$. Пусть $\psi_0 = 0$, тогда задача максимизации функции Гамильтона--Понтрягина (\ref{eq:hamilton_function}) эквивалентна следующей:
        $$
                \psi_2(v_1 - v_2x_2) \to \max\limits_{v \in \Delta}.
        $$
        В случае заданных первых ограничений на управление (когда $v_1 \geqslant 0$) требуется положить $\psi_2 \leqslant 0$, так как в противном случае максимум не достигается. При таком допущении максимум будет достигнут на управлении $v_1 = 0$, при котором исходная система (\ref{eq:main_system}) имеет только тождественно нулевое решение, не удовлетворяющее целевому множеству ни при одном выборе параметров. Для случая со вторым типом ограничений ($v_1 \in \R$) можно рассуждать аналогично.
\end{remark}

Также видно, что вторые уравнения исходной и сопряженной систем (\ref{eq:main_system}, \ref{eq:auxiliary_system}) зависят только от одной переменной, поэтому мы можем найти их вид, применив формулу Коши. Можно проверить, что фундаментальная матрица имеет вид\footnote{в данном случае фундаментальная матрица имеет размер $1 \times 1$}
$
        X(t, \tau) = e^{-\int_\tau^t v_2(s) ds}.
$
Тогда, учитывая начальные условия, получаем искомые выражения:
\begin{equation} \label{eq:x_2}
        x_2(t) = \int\limits_0^t e^{-\int_\tau^t v_2(s)\,ds} v_1(\tau)\,d\tau,
\end{equation}
\begin{equation} \label{eq:psi_2}
        \psi_2(t) = e^{\,\int_0^t v_2(s)\,ds}\psi_2^0 - \int\limits_0^T e^{\int_\tau^t v_2(s)\,ds} \psi_1^0\,d\tau,\mbox{ где $\psi_2^0 = \psi_2(0)$.}
\end{equation}

