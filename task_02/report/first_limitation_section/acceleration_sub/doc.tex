\subsection{Режимы акселерации и отсутствия торможения}

Как уже было отмечено, в этом случае поведение системы описывается уравнениями (\ref{eq:firstlim_psi2}), (\ref{firstlim_x2}) и (\ref{firstlim_x1}).

Из (\ref{eq:firstlim_psi2}) и (\ref{firstlim_x2}) следует, что соотношения $x_1(T) = L$ и $x_2(T) = \varepsilon_0$ являются линейной системой по $A$ и $B$. После этого, если $A < 0$, то необходимо проверить условия:
$$
B > \frac{\alpha}{2} \qquad \mbox{и} \qquad t_1 > T \Leftrightarrow \frac{1}{k_1 + 1} \ln \frac{\frac{\alpha}{2} - B}{A} > T,
$$
чтобы проверить, действительно ли реализуется режим отсутствия торможения. Зная $A$ и $B$, мы получим $\psi_1^0$ и $\psi_2^0$, что позволяет полностью восстановить управление и траекторию.