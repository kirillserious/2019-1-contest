\subsection{Режим сильного торможения}

В этом режиме соотношения (\ref{eq:firslim_weak_x2}) и (\ref{eq:firslim_weak_x1}) определяют поведение системы на отрезке $t_1 \leqslant t \leqslant t_2$, а на отрезке $t_2 \leqslant t \leqslant T$ ее буду описывать следующие соотношения
\begin{equation}\label{eq:strong_x2}
        x_2(t) = x_2^2e^{(k_2+1)(t_2-t)},
\end{equation}
\begin{equation}\label{eq:strong_x1}
        x_1(t) =x_1^2 + \frac{x_2^2}{k_2+1}\left(1 - e^{(k_2+1)(t_2-t)}\right).
\end{equation}
Из соотношений $x_1(T) = L$, $x_2(T) = \varepsilon_0$ получаем, что
$$
        x_2^1 = \varepsilon_0e^{(k_1+1)(T - t_2)}.
$$

\begin{assertion}
        Пусть оптимальное управление $u^* = [u_1^*,\,u_2^*]^\T$ реализует режим сильного торможения. Тогда $x_2(T;\,u^*) = \varepsilon$.
\end{assertion}
\begin{proof}
        Предположим противное, что для некоторого оптимального управления $u^*$справедливо $x_2(T;\,u^*) = \varepsilon_0 < \varepsilon$. Пусть для этого конкретного управления моменты переключения наступили в $t_1$ и $\hat t_2$. Обозначим $x_i^1 = x_i(t_1;\,u^*)$, $i = 1,\,2$.

        Рассмотрим управление $\hat u(t) = Mu^*(t)$. Считаем, что $M$ и момент $t_2$ (то есть момент переключения) есть некоторые свободные параметры, и рассмотрим следующие функции от них:
        $$
                x_2^2(M,\,t_2) = Mx_2^1e^{(k_1+1)(t_1-t_2)},
        $$
        $$
                x_1^2(M,\,t_2) = M \left(x_1^1+\frac{x_2^1}{k_1+1}\left(1 - e^{(k_1 +1)(t_1 - t_2)}\right)\right),
        $$
        причем $x_i^2(1,\,\hat t_2) = x_i(\hat t_2;\,u^*),\;i=1,\,2$. Далее рассмотрим функцию
        $$
                F(M,\,t_2) = x_1^2(M,\,t_2) + \frac{x_2^2(M,\,t_2)}{1+k_2}\left(1 - e^{(1+k_1)(t_2 - T)}\right) - L.
        $$
        Отметим, что по условию $F(1,\,\hat t_2) = 0$. Кроме того
        $$
                \frac{\partial F(M,\,t_2)}{\partial M} = \frac{\partial x_1^2(M,\,t_2)}{\partial M} + \frac{\partial x_2^2(M,\,t_2)}{\partial M}\cdot\frac1{k_2+1}\left(1 - e^{(1+k_1)(t_2-T)}\right) =
        $$
        $$
                = x_1^1 + \frac{x_2^1}{k_1 + 1}\left(1 - e^{(k_1 + 1)(t_1-t_2)}\right) + x_2^1e^{(k_1+1)(t_1-t_2)} > 0, \mbox{ при $t_2 = \hat t_2$.}
        $$

        Таким образом, мы можем применить теорему о неявной функции (подробно в \cite{zorich}[Гл.~7]). Итак, существуют такие $\xi_1 > 0$, $\xi_2 > 0$ и такая функция $t_2 = f(M)$, опреледенная при $|t_2 - \hat t_2| < \xi_1$, $|M - 1| < \xi_2$, что $F(M,\,f(M)) = 0$. Поскольку
        $$
                x_2^2(1,\,\hat t_2)e^{(k_2+1)(\hat t_2 - T)} = \varepsilon_0 < \varepsilon,
        $$
        то найдутся такие $\tilde M < 1$ и $\tilde t_2 = f(\tilde M)$, что
        $$
                x_2^2(\tilde M,\,\tilde t_2)e^{(k_2+1)(\tilde t_2 - T)} \leqslant \varepsilon,
        $$
        что показывает допустимость управления $\hat u = [\tilde Mu_1^*(t),\,\hat u_2(t)]^\T$, где
        $$
                \hat u_2(t) =
                \begin{cases}
                k_1, &\mbox{при $t < \tilde t_2$}\\
                k_2, &\mbox{при $t> \tilde t_2$}.
                \end{cases}
        $$
        Но $J(\hat u) < J(u^*)$, что противоречит оптимальности $u^*$.
\end{proof}

Итак, при реализации случая сильного торможения необходимо, чтобы $x_2(T) = \varepsilon$. Отметим, что при фиксированном $t_2$ траектория от момента $0$ до момента $T-t_2$ есть оптимальная траектория, реализующая режим слабого торможения для набора параметров $L' = x_2^1$, $T' = t_2$, $\varepsilon_0 = x_2^2$, где
$$
        x_2^2 = \varepsilon e^{(k_2+1)(T-t_2)} \qquad \mbox{и} \qquad x_1^2=L - \frac{x_2^2}{k_2+1}\left(1-e^{(k_2+1)(T-t_2)}\right).
$$
Тогда из соотношения (\ref{eq:weak_find_eta}) при фиксированном $t_2$ восстанавливаются (в количестве максимум двух наборов) отвечающие ему $\psi_1^0$ и $\psi_2^0$.