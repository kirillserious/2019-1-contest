\section{Решение задачи при первом типе ограничений на управление}

В этом пункте рассмотрим ограничения на значения управляющих параметров вида
$$
        u(t) \in \Omega = \{[u_1,\,u_2]^\T\in\R^2\;:\; u_1 \geqslant 0,\, u_2\in[k_1,\,k_2]\}.
$$
В этом случае задача минимизации равносильна задаче
$$
        J(u) = \int\limits_0^T\left(u_1(t) + \frac{\alpha}{2}\right)^2\,dt \to \min\limits_u, \mbox{ где $u(t) \in \Omega$, $t \in [0,\,T]$},
$$
при этом задача максимизации функции Гамильтона--Понтрягина эквивалентна задаче максимизации по $u$
$$
        \psi_0 \left(u_1 + \frac{\alpha}{2}\right)^2 + \psi_2(u_1 - x_2(1 + u_2)) \to \max\limits_u, \mbox{ где $\psi_0 < 0.$}
$$
Посколько принцип максимума Л.~С.~Понтрягина определяет значение $\psi_0$ с точностью до положтельного множителя, положим $\psi_0 = -\frac{1}{2}$. Тогда задача максимизации функции Гамильтона--Понтрягина, очевидно, распадается на максимизации по отдельным компонентам $u_1$ и $u_2$:
$$
        \begin{aligned}
        -\frac{1}{2}\left( u_1 + \frac{\alpha}{2}\right)^2 + \psi_2u_1 & \to \max\limits_{u_1\in[0,\,+\infty)}, \\
        -\psi_2x_2u_2 &\to \max\limits_{u_2 \in [k_1,\,k_2]}.
        \end{aligned}
$$
Эти две задачи имеют следующие решени:
\begin{equation}\label{eq:firstlim_u_1}
        u_1 = 
        \begin{cases}
                \psi_2 - \frac{\alpha}{2}, &\mbox{при $\psi_2 \geqslant \frac{\alpha}{2}$} \\
                0, & \mbox{при $\psi_2 < \frac{\alpha}{2}$},
        \end{cases}
\end{equation}
\begin{equation}\label{eq:firstlim_u_2}
        u_2 = 
        \begin{cases}
                k_2, &\mbox{при $\psi_2x_2<0$} \\
                [k_1,\,k_2], & \mbox{при $\psi_2x_2 = 0$} \\
                k_1, & \mbox{при $\psi_2x_2 > 0$}.
        \end{cases}
\end{equation}

В силу утверждения \ref{th:first} $u_1(t) > 0$ по меньшей мере в некоторой окрестности нуля, что в сочетании с формулами (\ref{eq:psi_2}) и (\ref{eq:firstlim_u_1}), накладывает ограничение $\psi_2^0 > \frac{\alpha}{2}$. Аналогично, это разрешает неопределенность формулы (\ref{eq:firstlim_u_2}): в некоторой окрестности нуля $u_2(t) = k_1$, и по непрерывности можно положить\footnote{Хотя конкретное значение $u_2$ в одной точке никакой роли не играет: везде в дальнейшем его вклад в решение будет производиться из-под интеграла.} $u_2(0) = k_1$. В этой же окрестности поведение системы однозначно определяется фомулой (\ref{eq:psi_2}), которую, взимая интегралы можно переписать в следующем виде:
\begin{equation} \label{eq:firstlim_psi2}
\psi_2(t) = \underbrace{\psi_2^0 - \frac{\psi_1^0}{k_1+1}}_Ae^{(k_1 + 1)t} + \underbrace{\frac{\psi_1^0}{k_1+1}}_B.
\end{equation}
При этом видно, что задание $A$ и $B$ однозначно определяет вид возможной траектории, \textit{подозреваемой} на оптимальность.

Выпишем также выражения для $x_2(t)$ и $x_1(t)$:
\begin{equation}\label{firstlim_x2}
        x_2(t) = \frac{1}{k_1 + 1} \left( A \sh((k_1 + 1) t) + \left( B - \frac{1}{2}\right)\left( 1 - e^{-(k_1 + 1)t}\right)\right),
\end{equation}
\begin{equation}\label{firstlim_x1}
        x_1(t) = \frac{1}{k_1 + 1} \left( A \ch((k_1 + 1) t) + \left( B - \frac{1}{2}\right)\left(t - \frac{1}{k_1 + 1}\left( 1 - e^{-(k_1 + 1)t}\right)\right)\right).
\end{equation}
Формулы (\ref{firstlim_x2}) и (\ref{firstlim_x1}) верны, пока $u_1(t) \neq 0$ (то есть пока не наступило \textit{выключение} $u_1$ в момент $\psi_2(t) = \frac{\alpha}{2}$); формула же (\ref{eq:firstlim_psi2}) верна, пока не наступила неопределенность по $u_2$, то есть пока $\psi_2(t) > 0$. 

При этом, если $A > 0$, то $\psi_2(t)$ строго возрастает, и формулы верны при любом $t$. Если же $A < 0$, то формулы (\ref{eq:firstlim_u_1}) и (\ref{eq:firstlim_u_2}) показывают, что нас будут интересовать моменты времени $t_1$ и $t_2$, в которые $\psi_2(t_1) = \frac{\alpha}{2}$ и $\psi_2(t_2) = 0$. В силу строгой монотонности $\psi_2(t)$ эти моменты единственны и $t_1 \leqslant t_2$ ($t_1$ и $t_2$ равны только в ситуации, когда параметр $\alpha$ равен нулю). В дальнейшем будем обозначать $x_j^i = x_j(t_i),\; j=1,\,2,\;i=1,\,2$.

Всвязи с написанным введем следующую классификацию рассматриваемых управлений (и отвечающих им траекторий):
\begin{enumerate}
        \item Пусть $A \geqslant 0$. Тогда будем говорить, что управление реализует \textit{режим акселерации}.
        \item Пусть $A < 0$.
        \begin{enumerate}
                \item Если $t_1 > T$, то будем говорить, что управление реализует \textit{режим отсутствия торможения}.
                \item Если $t_1 < T, \; t_2 > T$, то будем говорить, что управление реализует \textit{режим слабого торможения}.
                \item Если $t_2 < T$, то будем говорить, что управление реализует \textit{режим сильного торможения}.
        \end{enumerate}
\end{enumerate}

Разберем по порядку каждый из этих режимов. Будем полагать, что оптимальное управление лежит в соответствующем режиме и что для оптимальной траектории $x_2(T) = \varepsilon_0 \leqslant \varepsilon$.

\documentclass[a4paper, 11pt]{article}


\usepackage{amsmath}
\usepackage{amssymb}
\usepackage{hyperref}
\usepackage{url}
\usepackage{a4wide}
\usepackage[utf8]{inputenc}
\usepackage[main = russian, english]{babel}
\usepackage[pdftex]{graphicx}
\usepackage{float}
\usepackage{subcaption}
\usepackage{indentfirst}

% Красивый внешний вид теорем, определений и доказательств
\usepackage{amsthm}


\newenvironment{compactlist}{
        \begin{list}{{$\bullet$}}{
                        \setlength\partopsep{0pt}
                        \setlength\parskip{0pt}
                        \setlength\parsep{0pt}
                        \setlength\topsep{0pt}
                        \setlength\itemsep{0pt}
                }
        }{
        \end{list}
}
\theoremstyle{definition}
\newtheorem{definition}{Определение}

\theoremstyle{plane}
\newtheorem{theorem}{Теорема}
\newtheorem{assertion}{Утверждение}

\theoremstyle{remark}
\newtheorem{remark}{Замечание}

\renewcommand*{\proofname}{Доказательство}
\renewcommand\qedsymbol{$\blacksquare$}

\newcommand{\R}{\mathbb{R}}
\newcommand{\N}{\mathbb{N}}
\DeclareMathOperator{\sgn}{sgn}

\begin{document}
        \documentclass[a4paper, 11pt]{article}


\usepackage{amsmath}
\usepackage{amssymb}
\usepackage{hyperref}
\usepackage{url}
\usepackage{a4wide}
\usepackage[utf8]{inputenc}
\usepackage[main = russian, english]{babel}
\usepackage[pdftex]{graphicx}
\usepackage{float}
\usepackage{subcaption}
\usepackage{indentfirst}

% Красивый внешний вид теорем, определений и доказательств
\usepackage{amsthm}


\newenvironment{compactlist}{
        \begin{list}{{$\bullet$}}{
                        \setlength\partopsep{0pt}
                        \setlength\parskip{0pt}
                        \setlength\parsep{0pt}
                        \setlength\topsep{0pt}
                        \setlength\itemsep{0pt}
                }
        }{
        \end{list}
}
\theoremstyle{definition}
\newtheorem{definition}{Определение}

\theoremstyle{plane}
\newtheorem{theorem}{Теорема}
\newtheorem{assertion}{Утверждение}

\theoremstyle{remark}
\newtheorem{remark}{Замечание}

\renewcommand*{\proofname}{Доказательство}
\renewcommand\qedsymbol{$\blacksquare$}

\newcommand{\R}{\mathbb{R}}
\newcommand{\N}{\mathbb{N}}
\DeclareMathOperator{\sgn}{sgn}

\begin{document}
        \documentclass[a4paper, 11pt]{article}


\usepackage{amsmath}
\usepackage{amssymb}
\usepackage{hyperref}
\usepackage{url}
\usepackage{a4wide}
\usepackage[utf8]{inputenc}
\usepackage[main = russian, english]{babel}
\usepackage[pdftex]{graphicx}
\usepackage{float}
\usepackage{subcaption}
\usepackage{indentfirst}

% Красивый внешний вид теорем, определений и доказательств
\usepackage{amsthm}


\newenvironment{compactlist}{
        \begin{list}{{$\bullet$}}{
                        \setlength\partopsep{0pt}
                        \setlength\parskip{0pt}
                        \setlength\parsep{0pt}
                        \setlength\topsep{0pt}
                        \setlength\itemsep{0pt}
                }
        }{
        \end{list}
}
\theoremstyle{definition}
\newtheorem{definition}{Определение}

\theoremstyle{plane}
\newtheorem{theorem}{Теорема}
\newtheorem{assertion}{Утверждение}

\theoremstyle{remark}
\newtheorem{remark}{Замечание}

\renewcommand*{\proofname}{Доказательство}
\renewcommand\qedsymbol{$\blacksquare$}

\newcommand{\R}{\mathbb{R}}
\newcommand{\N}{\mathbb{N}}
\DeclareMathOperator{\sgn}{sgn}

\begin{document}
        \include{title_page/doc}

        \tableofcontents
        \clearpage
        
        \include{formulation_of_the_problem/doc}
        \include{research_of_the_system/doc}
        \include{algorithm/doc}
        \include{examples/doc}

        \begin{thebibliography}{9}
                \bibitem{pontryagin83} Л.~С.~Понтрягин, В.~Г.~Болтянский, Р.~В.~Гамрелидзе, Е.~Ф.~Мищенко. Математическая теория оптимальных процеccов. М.: Наука, 1983.
                \bibitem{li72} Э.~Б.~Ли, Л.~Маркус. Основы теории оптимального управления. М: Наука, 1972.
        \end{thebibliography} 
\end{document}

        \tableofcontents
        \clearpage
        
        \documentclass[a4paper, 11pt]{article}


\usepackage{amsmath}
\usepackage{amssymb}
\usepackage{hyperref}
\usepackage{url}
\usepackage{a4wide}
\usepackage[utf8]{inputenc}
\usepackage[main = russian, english]{babel}
\usepackage[pdftex]{graphicx}
\usepackage{float}
\usepackage{subcaption}
\usepackage{indentfirst}

% Красивый внешний вид теорем, определений и доказательств
\usepackage{amsthm}


\newenvironment{compactlist}{
        \begin{list}{{$\bullet$}}{
                        \setlength\partopsep{0pt}
                        \setlength\parskip{0pt}
                        \setlength\parsep{0pt}
                        \setlength\topsep{0pt}
                        \setlength\itemsep{0pt}
                }
        }{
        \end{list}
}
\theoremstyle{definition}
\newtheorem{definition}{Определение}

\theoremstyle{plane}
\newtheorem{theorem}{Теорема}
\newtheorem{assertion}{Утверждение}

\theoremstyle{remark}
\newtheorem{remark}{Замечание}

\renewcommand*{\proofname}{Доказательство}
\renewcommand\qedsymbol{$\blacksquare$}

\newcommand{\R}{\mathbb{R}}
\newcommand{\N}{\mathbb{N}}
\DeclareMathOperator{\sgn}{sgn}

\begin{document}
        \include{title_page/doc}

        \tableofcontents
        \clearpage
        
        \include{formulation_of_the_problem/doc}
        \include{research_of_the_system/doc}
        \include{algorithm/doc}
        \include{examples/doc}

        \begin{thebibliography}{9}
                \bibitem{pontryagin83} Л.~С.~Понтрягин, В.~Г.~Болтянский, Р.~В.~Гамрелидзе, Е.~Ф.~Мищенко. Математическая теория оптимальных процеccов. М.: Наука, 1983.
                \bibitem{li72} Э.~Б.~Ли, Л.~Маркус. Основы теории оптимального управления. М: Наука, 1972.
        \end{thebibliography} 
\end{document}
        \documentclass[a4paper, 11pt]{article}


\usepackage{amsmath}
\usepackage{amssymb}
\usepackage{hyperref}
\usepackage{url}
\usepackage{a4wide}
\usepackage[utf8]{inputenc}
\usepackage[main = russian, english]{babel}
\usepackage[pdftex]{graphicx}
\usepackage{float}
\usepackage{subcaption}
\usepackage{indentfirst}

% Красивый внешний вид теорем, определений и доказательств
\usepackage{amsthm}


\newenvironment{compactlist}{
        \begin{list}{{$\bullet$}}{
                        \setlength\partopsep{0pt}
                        \setlength\parskip{0pt}
                        \setlength\parsep{0pt}
                        \setlength\topsep{0pt}
                        \setlength\itemsep{0pt}
                }
        }{
        \end{list}
}
\theoremstyle{definition}
\newtheorem{definition}{Определение}

\theoremstyle{plane}
\newtheorem{theorem}{Теорема}
\newtheorem{assertion}{Утверждение}

\theoremstyle{remark}
\newtheorem{remark}{Замечание}

\renewcommand*{\proofname}{Доказательство}
\renewcommand\qedsymbol{$\blacksquare$}

\newcommand{\R}{\mathbb{R}}
\newcommand{\N}{\mathbb{N}}
\DeclareMathOperator{\sgn}{sgn}

\begin{document}
        \include{title_page/doc}

        \tableofcontents
        \clearpage
        
        \include{formulation_of_the_problem/doc}
        \include{research_of_the_system/doc}
        \include{algorithm/doc}
        \include{examples/doc}

        \begin{thebibliography}{9}
                \bibitem{pontryagin83} Л.~С.~Понтрягин, В.~Г.~Болтянский, Р.~В.~Гамрелидзе, Е.~Ф.~Мищенко. Математическая теория оптимальных процеccов. М.: Наука, 1983.
                \bibitem{li72} Э.~Б.~Ли, Л.~Маркус. Основы теории оптимального управления. М: Наука, 1972.
        \end{thebibliography} 
\end{document}
        \documentclass[a4paper, 11pt]{article}


\usepackage{amsmath}
\usepackage{amssymb}
\usepackage{hyperref}
\usepackage{url}
\usepackage{a4wide}
\usepackage[utf8]{inputenc}
\usepackage[main = russian, english]{babel}
\usepackage[pdftex]{graphicx}
\usepackage{float}
\usepackage{subcaption}
\usepackage{indentfirst}

% Красивый внешний вид теорем, определений и доказательств
\usepackage{amsthm}


\newenvironment{compactlist}{
        \begin{list}{{$\bullet$}}{
                        \setlength\partopsep{0pt}
                        \setlength\parskip{0pt}
                        \setlength\parsep{0pt}
                        \setlength\topsep{0pt}
                        \setlength\itemsep{0pt}
                }
        }{
        \end{list}
}
\theoremstyle{definition}
\newtheorem{definition}{Определение}

\theoremstyle{plane}
\newtheorem{theorem}{Теорема}
\newtheorem{assertion}{Утверждение}

\theoremstyle{remark}
\newtheorem{remark}{Замечание}

\renewcommand*{\proofname}{Доказательство}
\renewcommand\qedsymbol{$\blacksquare$}

\newcommand{\R}{\mathbb{R}}
\newcommand{\N}{\mathbb{N}}
\DeclareMathOperator{\sgn}{sgn}

\begin{document}
        \include{title_page/doc}

        \tableofcontents
        \clearpage
        
        \include{formulation_of_the_problem/doc}
        \include{research_of_the_system/doc}
        \include{algorithm/doc}
        \include{examples/doc}

        \begin{thebibliography}{9}
                \bibitem{pontryagin83} Л.~С.~Понтрягин, В.~Г.~Болтянский, Р.~В.~Гамрелидзе, Е.~Ф.~Мищенко. Математическая теория оптимальных процеccов. М.: Наука, 1983.
                \bibitem{li72} Э.~Б.~Ли, Л.~Маркус. Основы теории оптимального управления. М: Наука, 1972.
        \end{thebibliography} 
\end{document}
        \documentclass[a4paper, 11pt]{article}


\usepackage{amsmath}
\usepackage{amssymb}
\usepackage{hyperref}
\usepackage{url}
\usepackage{a4wide}
\usepackage[utf8]{inputenc}
\usepackage[main = russian, english]{babel}
\usepackage[pdftex]{graphicx}
\usepackage{float}
\usepackage{subcaption}
\usepackage{indentfirst}

% Красивый внешний вид теорем, определений и доказательств
\usepackage{amsthm}


\newenvironment{compactlist}{
        \begin{list}{{$\bullet$}}{
                        \setlength\partopsep{0pt}
                        \setlength\parskip{0pt}
                        \setlength\parsep{0pt}
                        \setlength\topsep{0pt}
                        \setlength\itemsep{0pt}
                }
        }{
        \end{list}
}
\theoremstyle{definition}
\newtheorem{definition}{Определение}

\theoremstyle{plane}
\newtheorem{theorem}{Теорема}
\newtheorem{assertion}{Утверждение}

\theoremstyle{remark}
\newtheorem{remark}{Замечание}

\renewcommand*{\proofname}{Доказательство}
\renewcommand\qedsymbol{$\blacksquare$}

\newcommand{\R}{\mathbb{R}}
\newcommand{\N}{\mathbb{N}}
\DeclareMathOperator{\sgn}{sgn}

\begin{document}
        \include{title_page/doc}

        \tableofcontents
        \clearpage
        
        \include{formulation_of_the_problem/doc}
        \include{research_of_the_system/doc}
        \include{algorithm/doc}
        \include{examples/doc}

        \begin{thebibliography}{9}
                \bibitem{pontryagin83} Л.~С.~Понтрягин, В.~Г.~Болтянский, Р.~В.~Гамрелидзе, Е.~Ф.~Мищенко. Математическая теория оптимальных процеccов. М.: Наука, 1983.
                \bibitem{li72} Э.~Б.~Ли, Л.~Маркус. Основы теории оптимального управления. М: Наука, 1972.
        \end{thebibliography} 
\end{document}

        \begin{thebibliography}{9}
                \bibitem{pontryagin83} Л.~С.~Понтрягин, В.~Г.~Болтянский, Р.~В.~Гамрелидзе, Е.~Ф.~Мищенко. Математическая теория оптимальных процеccов. М.: Наука, 1983.
                \bibitem{li72} Э.~Б.~Ли, Л.~Маркус. Основы теории оптимального управления. М: Наука, 1972.
        \end{thebibliography} 
\end{document}

        \tableofcontents
        \clearpage
        
        \documentclass[a4paper, 11pt]{article}


\usepackage{amsmath}
\usepackage{amssymb}
\usepackage{hyperref}
\usepackage{url}
\usepackage{a4wide}
\usepackage[utf8]{inputenc}
\usepackage[main = russian, english]{babel}
\usepackage[pdftex]{graphicx}
\usepackage{float}
\usepackage{subcaption}
\usepackage{indentfirst}

% Красивый внешний вид теорем, определений и доказательств
\usepackage{amsthm}


\newenvironment{compactlist}{
        \begin{list}{{$\bullet$}}{
                        \setlength\partopsep{0pt}
                        \setlength\parskip{0pt}
                        \setlength\parsep{0pt}
                        \setlength\topsep{0pt}
                        \setlength\itemsep{0pt}
                }
        }{
        \end{list}
}
\theoremstyle{definition}
\newtheorem{definition}{Определение}

\theoremstyle{plane}
\newtheorem{theorem}{Теорема}
\newtheorem{assertion}{Утверждение}

\theoremstyle{remark}
\newtheorem{remark}{Замечание}

\renewcommand*{\proofname}{Доказательство}
\renewcommand\qedsymbol{$\blacksquare$}

\newcommand{\R}{\mathbb{R}}
\newcommand{\N}{\mathbb{N}}
\DeclareMathOperator{\sgn}{sgn}

\begin{document}
        \documentclass[a4paper, 11pt]{article}


\usepackage{amsmath}
\usepackage{amssymb}
\usepackage{hyperref}
\usepackage{url}
\usepackage{a4wide}
\usepackage[utf8]{inputenc}
\usepackage[main = russian, english]{babel}
\usepackage[pdftex]{graphicx}
\usepackage{float}
\usepackage{subcaption}
\usepackage{indentfirst}

% Красивый внешний вид теорем, определений и доказательств
\usepackage{amsthm}


\newenvironment{compactlist}{
        \begin{list}{{$\bullet$}}{
                        \setlength\partopsep{0pt}
                        \setlength\parskip{0pt}
                        \setlength\parsep{0pt}
                        \setlength\topsep{0pt}
                        \setlength\itemsep{0pt}
                }
        }{
        \end{list}
}
\theoremstyle{definition}
\newtheorem{definition}{Определение}

\theoremstyle{plane}
\newtheorem{theorem}{Теорема}
\newtheorem{assertion}{Утверждение}

\theoremstyle{remark}
\newtheorem{remark}{Замечание}

\renewcommand*{\proofname}{Доказательство}
\renewcommand\qedsymbol{$\blacksquare$}

\newcommand{\R}{\mathbb{R}}
\newcommand{\N}{\mathbb{N}}
\DeclareMathOperator{\sgn}{sgn}

\begin{document}
        \include{title_page/doc}

        \tableofcontents
        \clearpage
        
        \include{formulation_of_the_problem/doc}
        \include{research_of_the_system/doc}
        \include{algorithm/doc}
        \include{examples/doc}

        \begin{thebibliography}{9}
                \bibitem{pontryagin83} Л.~С.~Понтрягин, В.~Г.~Болтянский, Р.~В.~Гамрелидзе, Е.~Ф.~Мищенко. Математическая теория оптимальных процеccов. М.: Наука, 1983.
                \bibitem{li72} Э.~Б.~Ли, Л.~Маркус. Основы теории оптимального управления. М: Наука, 1972.
        \end{thebibliography} 
\end{document}

        \tableofcontents
        \clearpage
        
        \documentclass[a4paper, 11pt]{article}


\usepackage{amsmath}
\usepackage{amssymb}
\usepackage{hyperref}
\usepackage{url}
\usepackage{a4wide}
\usepackage[utf8]{inputenc}
\usepackage[main = russian, english]{babel}
\usepackage[pdftex]{graphicx}
\usepackage{float}
\usepackage{subcaption}
\usepackage{indentfirst}

% Красивый внешний вид теорем, определений и доказательств
\usepackage{amsthm}


\newenvironment{compactlist}{
        \begin{list}{{$\bullet$}}{
                        \setlength\partopsep{0pt}
                        \setlength\parskip{0pt}
                        \setlength\parsep{0pt}
                        \setlength\topsep{0pt}
                        \setlength\itemsep{0pt}
                }
        }{
        \end{list}
}
\theoremstyle{definition}
\newtheorem{definition}{Определение}

\theoremstyle{plane}
\newtheorem{theorem}{Теорема}
\newtheorem{assertion}{Утверждение}

\theoremstyle{remark}
\newtheorem{remark}{Замечание}

\renewcommand*{\proofname}{Доказательство}
\renewcommand\qedsymbol{$\blacksquare$}

\newcommand{\R}{\mathbb{R}}
\newcommand{\N}{\mathbb{N}}
\DeclareMathOperator{\sgn}{sgn}

\begin{document}
        \include{title_page/doc}

        \tableofcontents
        \clearpage
        
        \include{formulation_of_the_problem/doc}
        \include{research_of_the_system/doc}
        \include{algorithm/doc}
        \include{examples/doc}

        \begin{thebibliography}{9}
                \bibitem{pontryagin83} Л.~С.~Понтрягин, В.~Г.~Болтянский, Р.~В.~Гамрелидзе, Е.~Ф.~Мищенко. Математическая теория оптимальных процеccов. М.: Наука, 1983.
                \bibitem{li72} Э.~Б.~Ли, Л.~Маркус. Основы теории оптимального управления. М: Наука, 1972.
        \end{thebibliography} 
\end{document}
        \documentclass[a4paper, 11pt]{article}


\usepackage{amsmath}
\usepackage{amssymb}
\usepackage{hyperref}
\usepackage{url}
\usepackage{a4wide}
\usepackage[utf8]{inputenc}
\usepackage[main = russian, english]{babel}
\usepackage[pdftex]{graphicx}
\usepackage{float}
\usepackage{subcaption}
\usepackage{indentfirst}

% Красивый внешний вид теорем, определений и доказательств
\usepackage{amsthm}


\newenvironment{compactlist}{
        \begin{list}{{$\bullet$}}{
                        \setlength\partopsep{0pt}
                        \setlength\parskip{0pt}
                        \setlength\parsep{0pt}
                        \setlength\topsep{0pt}
                        \setlength\itemsep{0pt}
                }
        }{
        \end{list}
}
\theoremstyle{definition}
\newtheorem{definition}{Определение}

\theoremstyle{plane}
\newtheorem{theorem}{Теорема}
\newtheorem{assertion}{Утверждение}

\theoremstyle{remark}
\newtheorem{remark}{Замечание}

\renewcommand*{\proofname}{Доказательство}
\renewcommand\qedsymbol{$\blacksquare$}

\newcommand{\R}{\mathbb{R}}
\newcommand{\N}{\mathbb{N}}
\DeclareMathOperator{\sgn}{sgn}

\begin{document}
        \include{title_page/doc}

        \tableofcontents
        \clearpage
        
        \include{formulation_of_the_problem/doc}
        \include{research_of_the_system/doc}
        \include{algorithm/doc}
        \include{examples/doc}

        \begin{thebibliography}{9}
                \bibitem{pontryagin83} Л.~С.~Понтрягин, В.~Г.~Болтянский, Р.~В.~Гамрелидзе, Е.~Ф.~Мищенко. Математическая теория оптимальных процеccов. М.: Наука, 1983.
                \bibitem{li72} Э.~Б.~Ли, Л.~Маркус. Основы теории оптимального управления. М: Наука, 1972.
        \end{thebibliography} 
\end{document}
        \documentclass[a4paper, 11pt]{article}


\usepackage{amsmath}
\usepackage{amssymb}
\usepackage{hyperref}
\usepackage{url}
\usepackage{a4wide}
\usepackage[utf8]{inputenc}
\usepackage[main = russian, english]{babel}
\usepackage[pdftex]{graphicx}
\usepackage{float}
\usepackage{subcaption}
\usepackage{indentfirst}

% Красивый внешний вид теорем, определений и доказательств
\usepackage{amsthm}


\newenvironment{compactlist}{
        \begin{list}{{$\bullet$}}{
                        \setlength\partopsep{0pt}
                        \setlength\parskip{0pt}
                        \setlength\parsep{0pt}
                        \setlength\topsep{0pt}
                        \setlength\itemsep{0pt}
                }
        }{
        \end{list}
}
\theoremstyle{definition}
\newtheorem{definition}{Определение}

\theoremstyle{plane}
\newtheorem{theorem}{Теорема}
\newtheorem{assertion}{Утверждение}

\theoremstyle{remark}
\newtheorem{remark}{Замечание}

\renewcommand*{\proofname}{Доказательство}
\renewcommand\qedsymbol{$\blacksquare$}

\newcommand{\R}{\mathbb{R}}
\newcommand{\N}{\mathbb{N}}
\DeclareMathOperator{\sgn}{sgn}

\begin{document}
        \include{title_page/doc}

        \tableofcontents
        \clearpage
        
        \include{formulation_of_the_problem/doc}
        \include{research_of_the_system/doc}
        \include{algorithm/doc}
        \include{examples/doc}

        \begin{thebibliography}{9}
                \bibitem{pontryagin83} Л.~С.~Понтрягин, В.~Г.~Болтянский, Р.~В.~Гамрелидзе, Е.~Ф.~Мищенко. Математическая теория оптимальных процеccов. М.: Наука, 1983.
                \bibitem{li72} Э.~Б.~Ли, Л.~Маркус. Основы теории оптимального управления. М: Наука, 1972.
        \end{thebibliography} 
\end{document}
        \documentclass[a4paper, 11pt]{article}


\usepackage{amsmath}
\usepackage{amssymb}
\usepackage{hyperref}
\usepackage{url}
\usepackage{a4wide}
\usepackage[utf8]{inputenc}
\usepackage[main = russian, english]{babel}
\usepackage[pdftex]{graphicx}
\usepackage{float}
\usepackage{subcaption}
\usepackage{indentfirst}

% Красивый внешний вид теорем, определений и доказательств
\usepackage{amsthm}


\newenvironment{compactlist}{
        \begin{list}{{$\bullet$}}{
                        \setlength\partopsep{0pt}
                        \setlength\parskip{0pt}
                        \setlength\parsep{0pt}
                        \setlength\topsep{0pt}
                        \setlength\itemsep{0pt}
                }
        }{
        \end{list}
}
\theoremstyle{definition}
\newtheorem{definition}{Определение}

\theoremstyle{plane}
\newtheorem{theorem}{Теорема}
\newtheorem{assertion}{Утверждение}

\theoremstyle{remark}
\newtheorem{remark}{Замечание}

\renewcommand*{\proofname}{Доказательство}
\renewcommand\qedsymbol{$\blacksquare$}

\newcommand{\R}{\mathbb{R}}
\newcommand{\N}{\mathbb{N}}
\DeclareMathOperator{\sgn}{sgn}

\begin{document}
        \include{title_page/doc}

        \tableofcontents
        \clearpage
        
        \include{formulation_of_the_problem/doc}
        \include{research_of_the_system/doc}
        \include{algorithm/doc}
        \include{examples/doc}

        \begin{thebibliography}{9}
                \bibitem{pontryagin83} Л.~С.~Понтрягин, В.~Г.~Болтянский, Р.~В.~Гамрелидзе, Е.~Ф.~Мищенко. Математическая теория оптимальных процеccов. М.: Наука, 1983.
                \bibitem{li72} Э.~Б.~Ли, Л.~Маркус. Основы теории оптимального управления. М: Наука, 1972.
        \end{thebibliography} 
\end{document}

        \begin{thebibliography}{9}
                \bibitem{pontryagin83} Л.~С.~Понтрягин, В.~Г.~Болтянский, Р.~В.~Гамрелидзе, Е.~Ф.~Мищенко. Математическая теория оптимальных процеccов. М.: Наука, 1983.
                \bibitem{li72} Э.~Б.~Ли, Л.~Маркус. Основы теории оптимального управления. М: Наука, 1972.
        \end{thebibliography} 
\end{document}
        \documentclass[a4paper, 11pt]{article}


\usepackage{amsmath}
\usepackage{amssymb}
\usepackage{hyperref}
\usepackage{url}
\usepackage{a4wide}
\usepackage[utf8]{inputenc}
\usepackage[main = russian, english]{babel}
\usepackage[pdftex]{graphicx}
\usepackage{float}
\usepackage{subcaption}
\usepackage{indentfirst}

% Красивый внешний вид теорем, определений и доказательств
\usepackage{amsthm}


\newenvironment{compactlist}{
        \begin{list}{{$\bullet$}}{
                        \setlength\partopsep{0pt}
                        \setlength\parskip{0pt}
                        \setlength\parsep{0pt}
                        \setlength\topsep{0pt}
                        \setlength\itemsep{0pt}
                }
        }{
        \end{list}
}
\theoremstyle{definition}
\newtheorem{definition}{Определение}

\theoremstyle{plane}
\newtheorem{theorem}{Теорема}
\newtheorem{assertion}{Утверждение}

\theoremstyle{remark}
\newtheorem{remark}{Замечание}

\renewcommand*{\proofname}{Доказательство}
\renewcommand\qedsymbol{$\blacksquare$}

\newcommand{\R}{\mathbb{R}}
\newcommand{\N}{\mathbb{N}}
\DeclareMathOperator{\sgn}{sgn}

\begin{document}
        \documentclass[a4paper, 11pt]{article}


\usepackage{amsmath}
\usepackage{amssymb}
\usepackage{hyperref}
\usepackage{url}
\usepackage{a4wide}
\usepackage[utf8]{inputenc}
\usepackage[main = russian, english]{babel}
\usepackage[pdftex]{graphicx}
\usepackage{float}
\usepackage{subcaption}
\usepackage{indentfirst}

% Красивый внешний вид теорем, определений и доказательств
\usepackage{amsthm}


\newenvironment{compactlist}{
        \begin{list}{{$\bullet$}}{
                        \setlength\partopsep{0pt}
                        \setlength\parskip{0pt}
                        \setlength\parsep{0pt}
                        \setlength\topsep{0pt}
                        \setlength\itemsep{0pt}
                }
        }{
        \end{list}
}
\theoremstyle{definition}
\newtheorem{definition}{Определение}

\theoremstyle{plane}
\newtheorem{theorem}{Теорема}
\newtheorem{assertion}{Утверждение}

\theoremstyle{remark}
\newtheorem{remark}{Замечание}

\renewcommand*{\proofname}{Доказательство}
\renewcommand\qedsymbol{$\blacksquare$}

\newcommand{\R}{\mathbb{R}}
\newcommand{\N}{\mathbb{N}}
\DeclareMathOperator{\sgn}{sgn}

\begin{document}
        \include{title_page/doc}

        \tableofcontents
        \clearpage
        
        \include{formulation_of_the_problem/doc}
        \include{research_of_the_system/doc}
        \include{algorithm/doc}
        \include{examples/doc}

        \begin{thebibliography}{9}
                \bibitem{pontryagin83} Л.~С.~Понтрягин, В.~Г.~Болтянский, Р.~В.~Гамрелидзе, Е.~Ф.~Мищенко. Математическая теория оптимальных процеccов. М.: Наука, 1983.
                \bibitem{li72} Э.~Б.~Ли, Л.~Маркус. Основы теории оптимального управления. М: Наука, 1972.
        \end{thebibliography} 
\end{document}

        \tableofcontents
        \clearpage
        
        \documentclass[a4paper, 11pt]{article}


\usepackage{amsmath}
\usepackage{amssymb}
\usepackage{hyperref}
\usepackage{url}
\usepackage{a4wide}
\usepackage[utf8]{inputenc}
\usepackage[main = russian, english]{babel}
\usepackage[pdftex]{graphicx}
\usepackage{float}
\usepackage{subcaption}
\usepackage{indentfirst}

% Красивый внешний вид теорем, определений и доказательств
\usepackage{amsthm}


\newenvironment{compactlist}{
        \begin{list}{{$\bullet$}}{
                        \setlength\partopsep{0pt}
                        \setlength\parskip{0pt}
                        \setlength\parsep{0pt}
                        \setlength\topsep{0pt}
                        \setlength\itemsep{0pt}
                }
        }{
        \end{list}
}
\theoremstyle{definition}
\newtheorem{definition}{Определение}

\theoremstyle{plane}
\newtheorem{theorem}{Теорема}
\newtheorem{assertion}{Утверждение}

\theoremstyle{remark}
\newtheorem{remark}{Замечание}

\renewcommand*{\proofname}{Доказательство}
\renewcommand\qedsymbol{$\blacksquare$}

\newcommand{\R}{\mathbb{R}}
\newcommand{\N}{\mathbb{N}}
\DeclareMathOperator{\sgn}{sgn}

\begin{document}
        \include{title_page/doc}

        \tableofcontents
        \clearpage
        
        \include{formulation_of_the_problem/doc}
        \include{research_of_the_system/doc}
        \include{algorithm/doc}
        \include{examples/doc}

        \begin{thebibliography}{9}
                \bibitem{pontryagin83} Л.~С.~Понтрягин, В.~Г.~Болтянский, Р.~В.~Гамрелидзе, Е.~Ф.~Мищенко. Математическая теория оптимальных процеccов. М.: Наука, 1983.
                \bibitem{li72} Э.~Б.~Ли, Л.~Маркус. Основы теории оптимального управления. М: Наука, 1972.
        \end{thebibliography} 
\end{document}
        \documentclass[a4paper, 11pt]{article}


\usepackage{amsmath}
\usepackage{amssymb}
\usepackage{hyperref}
\usepackage{url}
\usepackage{a4wide}
\usepackage[utf8]{inputenc}
\usepackage[main = russian, english]{babel}
\usepackage[pdftex]{graphicx}
\usepackage{float}
\usepackage{subcaption}
\usepackage{indentfirst}

% Красивый внешний вид теорем, определений и доказательств
\usepackage{amsthm}


\newenvironment{compactlist}{
        \begin{list}{{$\bullet$}}{
                        \setlength\partopsep{0pt}
                        \setlength\parskip{0pt}
                        \setlength\parsep{0pt}
                        \setlength\topsep{0pt}
                        \setlength\itemsep{0pt}
                }
        }{
        \end{list}
}
\theoremstyle{definition}
\newtheorem{definition}{Определение}

\theoremstyle{plane}
\newtheorem{theorem}{Теорема}
\newtheorem{assertion}{Утверждение}

\theoremstyle{remark}
\newtheorem{remark}{Замечание}

\renewcommand*{\proofname}{Доказательство}
\renewcommand\qedsymbol{$\blacksquare$}

\newcommand{\R}{\mathbb{R}}
\newcommand{\N}{\mathbb{N}}
\DeclareMathOperator{\sgn}{sgn}

\begin{document}
        \include{title_page/doc}

        \tableofcontents
        \clearpage
        
        \include{formulation_of_the_problem/doc}
        \include{research_of_the_system/doc}
        \include{algorithm/doc}
        \include{examples/doc}

        \begin{thebibliography}{9}
                \bibitem{pontryagin83} Л.~С.~Понтрягин, В.~Г.~Болтянский, Р.~В.~Гамрелидзе, Е.~Ф.~Мищенко. Математическая теория оптимальных процеccов. М.: Наука, 1983.
                \bibitem{li72} Э.~Б.~Ли, Л.~Маркус. Основы теории оптимального управления. М: Наука, 1972.
        \end{thebibliography} 
\end{document}
        \documentclass[a4paper, 11pt]{article}


\usepackage{amsmath}
\usepackage{amssymb}
\usepackage{hyperref}
\usepackage{url}
\usepackage{a4wide}
\usepackage[utf8]{inputenc}
\usepackage[main = russian, english]{babel}
\usepackage[pdftex]{graphicx}
\usepackage{float}
\usepackage{subcaption}
\usepackage{indentfirst}

% Красивый внешний вид теорем, определений и доказательств
\usepackage{amsthm}


\newenvironment{compactlist}{
        \begin{list}{{$\bullet$}}{
                        \setlength\partopsep{0pt}
                        \setlength\parskip{0pt}
                        \setlength\parsep{0pt}
                        \setlength\topsep{0pt}
                        \setlength\itemsep{0pt}
                }
        }{
        \end{list}
}
\theoremstyle{definition}
\newtheorem{definition}{Определение}

\theoremstyle{plane}
\newtheorem{theorem}{Теорема}
\newtheorem{assertion}{Утверждение}

\theoremstyle{remark}
\newtheorem{remark}{Замечание}

\renewcommand*{\proofname}{Доказательство}
\renewcommand\qedsymbol{$\blacksquare$}

\newcommand{\R}{\mathbb{R}}
\newcommand{\N}{\mathbb{N}}
\DeclareMathOperator{\sgn}{sgn}

\begin{document}
        \include{title_page/doc}

        \tableofcontents
        \clearpage
        
        \include{formulation_of_the_problem/doc}
        \include{research_of_the_system/doc}
        \include{algorithm/doc}
        \include{examples/doc}

        \begin{thebibliography}{9}
                \bibitem{pontryagin83} Л.~С.~Понтрягин, В.~Г.~Болтянский, Р.~В.~Гамрелидзе, Е.~Ф.~Мищенко. Математическая теория оптимальных процеccов. М.: Наука, 1983.
                \bibitem{li72} Э.~Б.~Ли, Л.~Маркус. Основы теории оптимального управления. М: Наука, 1972.
        \end{thebibliography} 
\end{document}
        \documentclass[a4paper, 11pt]{article}


\usepackage{amsmath}
\usepackage{amssymb}
\usepackage{hyperref}
\usepackage{url}
\usepackage{a4wide}
\usepackage[utf8]{inputenc}
\usepackage[main = russian, english]{babel}
\usepackage[pdftex]{graphicx}
\usepackage{float}
\usepackage{subcaption}
\usepackage{indentfirst}

% Красивый внешний вид теорем, определений и доказательств
\usepackage{amsthm}


\newenvironment{compactlist}{
        \begin{list}{{$\bullet$}}{
                        \setlength\partopsep{0pt}
                        \setlength\parskip{0pt}
                        \setlength\parsep{0pt}
                        \setlength\topsep{0pt}
                        \setlength\itemsep{0pt}
                }
        }{
        \end{list}
}
\theoremstyle{definition}
\newtheorem{definition}{Определение}

\theoremstyle{plane}
\newtheorem{theorem}{Теорема}
\newtheorem{assertion}{Утверждение}

\theoremstyle{remark}
\newtheorem{remark}{Замечание}

\renewcommand*{\proofname}{Доказательство}
\renewcommand\qedsymbol{$\blacksquare$}

\newcommand{\R}{\mathbb{R}}
\newcommand{\N}{\mathbb{N}}
\DeclareMathOperator{\sgn}{sgn}

\begin{document}
        \include{title_page/doc}

        \tableofcontents
        \clearpage
        
        \include{formulation_of_the_problem/doc}
        \include{research_of_the_system/doc}
        \include{algorithm/doc}
        \include{examples/doc}

        \begin{thebibliography}{9}
                \bibitem{pontryagin83} Л.~С.~Понтрягин, В.~Г.~Болтянский, Р.~В.~Гамрелидзе, Е.~Ф.~Мищенко. Математическая теория оптимальных процеccов. М.: Наука, 1983.
                \bibitem{li72} Э.~Б.~Ли, Л.~Маркус. Основы теории оптимального управления. М: Наука, 1972.
        \end{thebibliography} 
\end{document}

        \begin{thebibliography}{9}
                \bibitem{pontryagin83} Л.~С.~Понтрягин, В.~Г.~Болтянский, Р.~В.~Гамрелидзе, Е.~Ф.~Мищенко. Математическая теория оптимальных процеccов. М.: Наука, 1983.
                \bibitem{li72} Э.~Б.~Ли, Л.~Маркус. Основы теории оптимального управления. М: Наука, 1972.
        \end{thebibliography} 
\end{document}
        \documentclass[a4paper, 11pt]{article}


\usepackage{amsmath}
\usepackage{amssymb}
\usepackage{hyperref}
\usepackage{url}
\usepackage{a4wide}
\usepackage[utf8]{inputenc}
\usepackage[main = russian, english]{babel}
\usepackage[pdftex]{graphicx}
\usepackage{float}
\usepackage{subcaption}
\usepackage{indentfirst}

% Красивый внешний вид теорем, определений и доказательств
\usepackage{amsthm}


\newenvironment{compactlist}{
        \begin{list}{{$\bullet$}}{
                        \setlength\partopsep{0pt}
                        \setlength\parskip{0pt}
                        \setlength\parsep{0pt}
                        \setlength\topsep{0pt}
                        \setlength\itemsep{0pt}
                }
        }{
        \end{list}
}
\theoremstyle{definition}
\newtheorem{definition}{Определение}

\theoremstyle{plane}
\newtheorem{theorem}{Теорема}
\newtheorem{assertion}{Утверждение}

\theoremstyle{remark}
\newtheorem{remark}{Замечание}

\renewcommand*{\proofname}{Доказательство}
\renewcommand\qedsymbol{$\blacksquare$}

\newcommand{\R}{\mathbb{R}}
\newcommand{\N}{\mathbb{N}}
\DeclareMathOperator{\sgn}{sgn}

\begin{document}
        \documentclass[a4paper, 11pt]{article}


\usepackage{amsmath}
\usepackage{amssymb}
\usepackage{hyperref}
\usepackage{url}
\usepackage{a4wide}
\usepackage[utf8]{inputenc}
\usepackage[main = russian, english]{babel}
\usepackage[pdftex]{graphicx}
\usepackage{float}
\usepackage{subcaption}
\usepackage{indentfirst}

% Красивый внешний вид теорем, определений и доказательств
\usepackage{amsthm}


\newenvironment{compactlist}{
        \begin{list}{{$\bullet$}}{
                        \setlength\partopsep{0pt}
                        \setlength\parskip{0pt}
                        \setlength\parsep{0pt}
                        \setlength\topsep{0pt}
                        \setlength\itemsep{0pt}
                }
        }{
        \end{list}
}
\theoremstyle{definition}
\newtheorem{definition}{Определение}

\theoremstyle{plane}
\newtheorem{theorem}{Теорема}
\newtheorem{assertion}{Утверждение}

\theoremstyle{remark}
\newtheorem{remark}{Замечание}

\renewcommand*{\proofname}{Доказательство}
\renewcommand\qedsymbol{$\blacksquare$}

\newcommand{\R}{\mathbb{R}}
\newcommand{\N}{\mathbb{N}}
\DeclareMathOperator{\sgn}{sgn}

\begin{document}
        \include{title_page/doc}

        \tableofcontents
        \clearpage
        
        \include{formulation_of_the_problem/doc}
        \include{research_of_the_system/doc}
        \include{algorithm/doc}
        \include{examples/doc}

        \begin{thebibliography}{9}
                \bibitem{pontryagin83} Л.~С.~Понтрягин, В.~Г.~Болтянский, Р.~В.~Гамрелидзе, Е.~Ф.~Мищенко. Математическая теория оптимальных процеccов. М.: Наука, 1983.
                \bibitem{li72} Э.~Б.~Ли, Л.~Маркус. Основы теории оптимального управления. М: Наука, 1972.
        \end{thebibliography} 
\end{document}

        \tableofcontents
        \clearpage
        
        \documentclass[a4paper, 11pt]{article}


\usepackage{amsmath}
\usepackage{amssymb}
\usepackage{hyperref}
\usepackage{url}
\usepackage{a4wide}
\usepackage[utf8]{inputenc}
\usepackage[main = russian, english]{babel}
\usepackage[pdftex]{graphicx}
\usepackage{float}
\usepackage{subcaption}
\usepackage{indentfirst}

% Красивый внешний вид теорем, определений и доказательств
\usepackage{amsthm}


\newenvironment{compactlist}{
        \begin{list}{{$\bullet$}}{
                        \setlength\partopsep{0pt}
                        \setlength\parskip{0pt}
                        \setlength\parsep{0pt}
                        \setlength\topsep{0pt}
                        \setlength\itemsep{0pt}
                }
        }{
        \end{list}
}
\theoremstyle{definition}
\newtheorem{definition}{Определение}

\theoremstyle{plane}
\newtheorem{theorem}{Теорема}
\newtheorem{assertion}{Утверждение}

\theoremstyle{remark}
\newtheorem{remark}{Замечание}

\renewcommand*{\proofname}{Доказательство}
\renewcommand\qedsymbol{$\blacksquare$}

\newcommand{\R}{\mathbb{R}}
\newcommand{\N}{\mathbb{N}}
\DeclareMathOperator{\sgn}{sgn}

\begin{document}
        \include{title_page/doc}

        \tableofcontents
        \clearpage
        
        \include{formulation_of_the_problem/doc}
        \include{research_of_the_system/doc}
        \include{algorithm/doc}
        \include{examples/doc}

        \begin{thebibliography}{9}
                \bibitem{pontryagin83} Л.~С.~Понтрягин, В.~Г.~Болтянский, Р.~В.~Гамрелидзе, Е.~Ф.~Мищенко. Математическая теория оптимальных процеccов. М.: Наука, 1983.
                \bibitem{li72} Э.~Б.~Ли, Л.~Маркус. Основы теории оптимального управления. М: Наука, 1972.
        \end{thebibliography} 
\end{document}
        \documentclass[a4paper, 11pt]{article}


\usepackage{amsmath}
\usepackage{amssymb}
\usepackage{hyperref}
\usepackage{url}
\usepackage{a4wide}
\usepackage[utf8]{inputenc}
\usepackage[main = russian, english]{babel}
\usepackage[pdftex]{graphicx}
\usepackage{float}
\usepackage{subcaption}
\usepackage{indentfirst}

% Красивый внешний вид теорем, определений и доказательств
\usepackage{amsthm}


\newenvironment{compactlist}{
        \begin{list}{{$\bullet$}}{
                        \setlength\partopsep{0pt}
                        \setlength\parskip{0pt}
                        \setlength\parsep{0pt}
                        \setlength\topsep{0pt}
                        \setlength\itemsep{0pt}
                }
        }{
        \end{list}
}
\theoremstyle{definition}
\newtheorem{definition}{Определение}

\theoremstyle{plane}
\newtheorem{theorem}{Теорема}
\newtheorem{assertion}{Утверждение}

\theoremstyle{remark}
\newtheorem{remark}{Замечание}

\renewcommand*{\proofname}{Доказательство}
\renewcommand\qedsymbol{$\blacksquare$}

\newcommand{\R}{\mathbb{R}}
\newcommand{\N}{\mathbb{N}}
\DeclareMathOperator{\sgn}{sgn}

\begin{document}
        \include{title_page/doc}

        \tableofcontents
        \clearpage
        
        \include{formulation_of_the_problem/doc}
        \include{research_of_the_system/doc}
        \include{algorithm/doc}
        \include{examples/doc}

        \begin{thebibliography}{9}
                \bibitem{pontryagin83} Л.~С.~Понтрягин, В.~Г.~Болтянский, Р.~В.~Гамрелидзе, Е.~Ф.~Мищенко. Математическая теория оптимальных процеccов. М.: Наука, 1983.
                \bibitem{li72} Э.~Б.~Ли, Л.~Маркус. Основы теории оптимального управления. М: Наука, 1972.
        \end{thebibliography} 
\end{document}
        \documentclass[a4paper, 11pt]{article}


\usepackage{amsmath}
\usepackage{amssymb}
\usepackage{hyperref}
\usepackage{url}
\usepackage{a4wide}
\usepackage[utf8]{inputenc}
\usepackage[main = russian, english]{babel}
\usepackage[pdftex]{graphicx}
\usepackage{float}
\usepackage{subcaption}
\usepackage{indentfirst}

% Красивый внешний вид теорем, определений и доказательств
\usepackage{amsthm}


\newenvironment{compactlist}{
        \begin{list}{{$\bullet$}}{
                        \setlength\partopsep{0pt}
                        \setlength\parskip{0pt}
                        \setlength\parsep{0pt}
                        \setlength\topsep{0pt}
                        \setlength\itemsep{0pt}
                }
        }{
        \end{list}
}
\theoremstyle{definition}
\newtheorem{definition}{Определение}

\theoremstyle{plane}
\newtheorem{theorem}{Теорема}
\newtheorem{assertion}{Утверждение}

\theoremstyle{remark}
\newtheorem{remark}{Замечание}

\renewcommand*{\proofname}{Доказательство}
\renewcommand\qedsymbol{$\blacksquare$}

\newcommand{\R}{\mathbb{R}}
\newcommand{\N}{\mathbb{N}}
\DeclareMathOperator{\sgn}{sgn}

\begin{document}
        \include{title_page/doc}

        \tableofcontents
        \clearpage
        
        \include{formulation_of_the_problem/doc}
        \include{research_of_the_system/doc}
        \include{algorithm/doc}
        \include{examples/doc}

        \begin{thebibliography}{9}
                \bibitem{pontryagin83} Л.~С.~Понтрягин, В.~Г.~Болтянский, Р.~В.~Гамрелидзе, Е.~Ф.~Мищенко. Математическая теория оптимальных процеccов. М.: Наука, 1983.
                \bibitem{li72} Э.~Б.~Ли, Л.~Маркус. Основы теории оптимального управления. М: Наука, 1972.
        \end{thebibliography} 
\end{document}
        \documentclass[a4paper, 11pt]{article}


\usepackage{amsmath}
\usepackage{amssymb}
\usepackage{hyperref}
\usepackage{url}
\usepackage{a4wide}
\usepackage[utf8]{inputenc}
\usepackage[main = russian, english]{babel}
\usepackage[pdftex]{graphicx}
\usepackage{float}
\usepackage{subcaption}
\usepackage{indentfirst}

% Красивый внешний вид теорем, определений и доказательств
\usepackage{amsthm}


\newenvironment{compactlist}{
        \begin{list}{{$\bullet$}}{
                        \setlength\partopsep{0pt}
                        \setlength\parskip{0pt}
                        \setlength\parsep{0pt}
                        \setlength\topsep{0pt}
                        \setlength\itemsep{0pt}
                }
        }{
        \end{list}
}
\theoremstyle{definition}
\newtheorem{definition}{Определение}

\theoremstyle{plane}
\newtheorem{theorem}{Теорема}
\newtheorem{assertion}{Утверждение}

\theoremstyle{remark}
\newtheorem{remark}{Замечание}

\renewcommand*{\proofname}{Доказательство}
\renewcommand\qedsymbol{$\blacksquare$}

\newcommand{\R}{\mathbb{R}}
\newcommand{\N}{\mathbb{N}}
\DeclareMathOperator{\sgn}{sgn}

\begin{document}
        \include{title_page/doc}

        \tableofcontents
        \clearpage
        
        \include{formulation_of_the_problem/doc}
        \include{research_of_the_system/doc}
        \include{algorithm/doc}
        \include{examples/doc}

        \begin{thebibliography}{9}
                \bibitem{pontryagin83} Л.~С.~Понтрягин, В.~Г.~Болтянский, Р.~В.~Гамрелидзе, Е.~Ф.~Мищенко. Математическая теория оптимальных процеccов. М.: Наука, 1983.
                \bibitem{li72} Э.~Б.~Ли, Л.~Маркус. Основы теории оптимального управления. М: Наука, 1972.
        \end{thebibliography} 
\end{document}

        \begin{thebibliography}{9}
                \bibitem{pontryagin83} Л.~С.~Понтрягин, В.~Г.~Болтянский, Р.~В.~Гамрелидзе, Е.~Ф.~Мищенко. Математическая теория оптимальных процеccов. М.: Наука, 1983.
                \bibitem{li72} Э.~Б.~Ли, Л.~Маркус. Основы теории оптимального управления. М: Наука, 1972.
        \end{thebibliography} 
\end{document}
        \documentclass[a4paper, 11pt]{article}


\usepackage{amsmath}
\usepackage{amssymb}
\usepackage{hyperref}
\usepackage{url}
\usepackage{a4wide}
\usepackage[utf8]{inputenc}
\usepackage[main = russian, english]{babel}
\usepackage[pdftex]{graphicx}
\usepackage{float}
\usepackage{subcaption}
\usepackage{indentfirst}

% Красивый внешний вид теорем, определений и доказательств
\usepackage{amsthm}


\newenvironment{compactlist}{
        \begin{list}{{$\bullet$}}{
                        \setlength\partopsep{0pt}
                        \setlength\parskip{0pt}
                        \setlength\parsep{0pt}
                        \setlength\topsep{0pt}
                        \setlength\itemsep{0pt}
                }
        }{
        \end{list}
}
\theoremstyle{definition}
\newtheorem{definition}{Определение}

\theoremstyle{plane}
\newtheorem{theorem}{Теорема}
\newtheorem{assertion}{Утверждение}

\theoremstyle{remark}
\newtheorem{remark}{Замечание}

\renewcommand*{\proofname}{Доказательство}
\renewcommand\qedsymbol{$\blacksquare$}

\newcommand{\R}{\mathbb{R}}
\newcommand{\N}{\mathbb{N}}
\DeclareMathOperator{\sgn}{sgn}

\begin{document}
        \documentclass[a4paper, 11pt]{article}


\usepackage{amsmath}
\usepackage{amssymb}
\usepackage{hyperref}
\usepackage{url}
\usepackage{a4wide}
\usepackage[utf8]{inputenc}
\usepackage[main = russian, english]{babel}
\usepackage[pdftex]{graphicx}
\usepackage{float}
\usepackage{subcaption}
\usepackage{indentfirst}

% Красивый внешний вид теорем, определений и доказательств
\usepackage{amsthm}


\newenvironment{compactlist}{
        \begin{list}{{$\bullet$}}{
                        \setlength\partopsep{0pt}
                        \setlength\parskip{0pt}
                        \setlength\parsep{0pt}
                        \setlength\topsep{0pt}
                        \setlength\itemsep{0pt}
                }
        }{
        \end{list}
}
\theoremstyle{definition}
\newtheorem{definition}{Определение}

\theoremstyle{plane}
\newtheorem{theorem}{Теорема}
\newtheorem{assertion}{Утверждение}

\theoremstyle{remark}
\newtheorem{remark}{Замечание}

\renewcommand*{\proofname}{Доказательство}
\renewcommand\qedsymbol{$\blacksquare$}

\newcommand{\R}{\mathbb{R}}
\newcommand{\N}{\mathbb{N}}
\DeclareMathOperator{\sgn}{sgn}

\begin{document}
        \include{title_page/doc}

        \tableofcontents
        \clearpage
        
        \include{formulation_of_the_problem/doc}
        \include{research_of_the_system/doc}
        \include{algorithm/doc}
        \include{examples/doc}

        \begin{thebibliography}{9}
                \bibitem{pontryagin83} Л.~С.~Понтрягин, В.~Г.~Болтянский, Р.~В.~Гамрелидзе, Е.~Ф.~Мищенко. Математическая теория оптимальных процеccов. М.: Наука, 1983.
                \bibitem{li72} Э.~Б.~Ли, Л.~Маркус. Основы теории оптимального управления. М: Наука, 1972.
        \end{thebibliography} 
\end{document}

        \tableofcontents
        \clearpage
        
        \documentclass[a4paper, 11pt]{article}


\usepackage{amsmath}
\usepackage{amssymb}
\usepackage{hyperref}
\usepackage{url}
\usepackage{a4wide}
\usepackage[utf8]{inputenc}
\usepackage[main = russian, english]{babel}
\usepackage[pdftex]{graphicx}
\usepackage{float}
\usepackage{subcaption}
\usepackage{indentfirst}

% Красивый внешний вид теорем, определений и доказательств
\usepackage{amsthm}


\newenvironment{compactlist}{
        \begin{list}{{$\bullet$}}{
                        \setlength\partopsep{0pt}
                        \setlength\parskip{0pt}
                        \setlength\parsep{0pt}
                        \setlength\topsep{0pt}
                        \setlength\itemsep{0pt}
                }
        }{
        \end{list}
}
\theoremstyle{definition}
\newtheorem{definition}{Определение}

\theoremstyle{plane}
\newtheorem{theorem}{Теорема}
\newtheorem{assertion}{Утверждение}

\theoremstyle{remark}
\newtheorem{remark}{Замечание}

\renewcommand*{\proofname}{Доказательство}
\renewcommand\qedsymbol{$\blacksquare$}

\newcommand{\R}{\mathbb{R}}
\newcommand{\N}{\mathbb{N}}
\DeclareMathOperator{\sgn}{sgn}

\begin{document}
        \include{title_page/doc}

        \tableofcontents
        \clearpage
        
        \include{formulation_of_the_problem/doc}
        \include{research_of_the_system/doc}
        \include{algorithm/doc}
        \include{examples/doc}

        \begin{thebibliography}{9}
                \bibitem{pontryagin83} Л.~С.~Понтрягин, В.~Г.~Болтянский, Р.~В.~Гамрелидзе, Е.~Ф.~Мищенко. Математическая теория оптимальных процеccов. М.: Наука, 1983.
                \bibitem{li72} Э.~Б.~Ли, Л.~Маркус. Основы теории оптимального управления. М: Наука, 1972.
        \end{thebibliography} 
\end{document}
        \documentclass[a4paper, 11pt]{article}


\usepackage{amsmath}
\usepackage{amssymb}
\usepackage{hyperref}
\usepackage{url}
\usepackage{a4wide}
\usepackage[utf8]{inputenc}
\usepackage[main = russian, english]{babel}
\usepackage[pdftex]{graphicx}
\usepackage{float}
\usepackage{subcaption}
\usepackage{indentfirst}

% Красивый внешний вид теорем, определений и доказательств
\usepackage{amsthm}


\newenvironment{compactlist}{
        \begin{list}{{$\bullet$}}{
                        \setlength\partopsep{0pt}
                        \setlength\parskip{0pt}
                        \setlength\parsep{0pt}
                        \setlength\topsep{0pt}
                        \setlength\itemsep{0pt}
                }
        }{
        \end{list}
}
\theoremstyle{definition}
\newtheorem{definition}{Определение}

\theoremstyle{plane}
\newtheorem{theorem}{Теорема}
\newtheorem{assertion}{Утверждение}

\theoremstyle{remark}
\newtheorem{remark}{Замечание}

\renewcommand*{\proofname}{Доказательство}
\renewcommand\qedsymbol{$\blacksquare$}

\newcommand{\R}{\mathbb{R}}
\newcommand{\N}{\mathbb{N}}
\DeclareMathOperator{\sgn}{sgn}

\begin{document}
        \include{title_page/doc}

        \tableofcontents
        \clearpage
        
        \include{formulation_of_the_problem/doc}
        \include{research_of_the_system/doc}
        \include{algorithm/doc}
        \include{examples/doc}

        \begin{thebibliography}{9}
                \bibitem{pontryagin83} Л.~С.~Понтрягин, В.~Г.~Болтянский, Р.~В.~Гамрелидзе, Е.~Ф.~Мищенко. Математическая теория оптимальных процеccов. М.: Наука, 1983.
                \bibitem{li72} Э.~Б.~Ли, Л.~Маркус. Основы теории оптимального управления. М: Наука, 1972.
        \end{thebibliography} 
\end{document}
        \documentclass[a4paper, 11pt]{article}


\usepackage{amsmath}
\usepackage{amssymb}
\usepackage{hyperref}
\usepackage{url}
\usepackage{a4wide}
\usepackage[utf8]{inputenc}
\usepackage[main = russian, english]{babel}
\usepackage[pdftex]{graphicx}
\usepackage{float}
\usepackage{subcaption}
\usepackage{indentfirst}

% Красивый внешний вид теорем, определений и доказательств
\usepackage{amsthm}


\newenvironment{compactlist}{
        \begin{list}{{$\bullet$}}{
                        \setlength\partopsep{0pt}
                        \setlength\parskip{0pt}
                        \setlength\parsep{0pt}
                        \setlength\topsep{0pt}
                        \setlength\itemsep{0pt}
                }
        }{
        \end{list}
}
\theoremstyle{definition}
\newtheorem{definition}{Определение}

\theoremstyle{plane}
\newtheorem{theorem}{Теорема}
\newtheorem{assertion}{Утверждение}

\theoremstyle{remark}
\newtheorem{remark}{Замечание}

\renewcommand*{\proofname}{Доказательство}
\renewcommand\qedsymbol{$\blacksquare$}

\newcommand{\R}{\mathbb{R}}
\newcommand{\N}{\mathbb{N}}
\DeclareMathOperator{\sgn}{sgn}

\begin{document}
        \include{title_page/doc}

        \tableofcontents
        \clearpage
        
        \include{formulation_of_the_problem/doc}
        \include{research_of_the_system/doc}
        \include{algorithm/doc}
        \include{examples/doc}

        \begin{thebibliography}{9}
                \bibitem{pontryagin83} Л.~С.~Понтрягин, В.~Г.~Болтянский, Р.~В.~Гамрелидзе, Е.~Ф.~Мищенко. Математическая теория оптимальных процеccов. М.: Наука, 1983.
                \bibitem{li72} Э.~Б.~Ли, Л.~Маркус. Основы теории оптимального управления. М: Наука, 1972.
        \end{thebibliography} 
\end{document}
        \documentclass[a4paper, 11pt]{article}


\usepackage{amsmath}
\usepackage{amssymb}
\usepackage{hyperref}
\usepackage{url}
\usepackage{a4wide}
\usepackage[utf8]{inputenc}
\usepackage[main = russian, english]{babel}
\usepackage[pdftex]{graphicx}
\usepackage{float}
\usepackage{subcaption}
\usepackage{indentfirst}

% Красивый внешний вид теорем, определений и доказательств
\usepackage{amsthm}


\newenvironment{compactlist}{
        \begin{list}{{$\bullet$}}{
                        \setlength\partopsep{0pt}
                        \setlength\parskip{0pt}
                        \setlength\parsep{0pt}
                        \setlength\topsep{0pt}
                        \setlength\itemsep{0pt}
                }
        }{
        \end{list}
}
\theoremstyle{definition}
\newtheorem{definition}{Определение}

\theoremstyle{plane}
\newtheorem{theorem}{Теорема}
\newtheorem{assertion}{Утверждение}

\theoremstyle{remark}
\newtheorem{remark}{Замечание}

\renewcommand*{\proofname}{Доказательство}
\renewcommand\qedsymbol{$\blacksquare$}

\newcommand{\R}{\mathbb{R}}
\newcommand{\N}{\mathbb{N}}
\DeclareMathOperator{\sgn}{sgn}

\begin{document}
        \include{title_page/doc}

        \tableofcontents
        \clearpage
        
        \include{formulation_of_the_problem/doc}
        \include{research_of_the_system/doc}
        \include{algorithm/doc}
        \include{examples/doc}

        \begin{thebibliography}{9}
                \bibitem{pontryagin83} Л.~С.~Понтрягин, В.~Г.~Болтянский, Р.~В.~Гамрелидзе, Е.~Ф.~Мищенко. Математическая теория оптимальных процеccов. М.: Наука, 1983.
                \bibitem{li72} Э.~Б.~Ли, Л.~Маркус. Основы теории оптимального управления. М: Наука, 1972.
        \end{thebibliography} 
\end{document}

        \begin{thebibliography}{9}
                \bibitem{pontryagin83} Л.~С.~Понтрягин, В.~Г.~Болтянский, Р.~В.~Гамрелидзе, Е.~Ф.~Мищенко. Математическая теория оптимальных процеccов. М.: Наука, 1983.
                \bibitem{li72} Э.~Б.~Ли, Л.~Маркус. Основы теории оптимального управления. М: Наука, 1972.
        \end{thebibliography} 
\end{document}

        \begin{thebibliography}{9}
                \bibitem{pontryagin83} Л.~С.~Понтрягин, В.~Г.~Болтянский, Р.~В.~Гамрелидзе, Е.~Ф.~Мищенко. Математическая теория оптимальных процеccов. М.: Наука, 1983.
                \bibitem{li72} Э.~Б.~Ли, Л.~Маркус. Основы теории оптимального управления. М: Наука, 1972.
        \end{thebibliography} 
\end{document}
\documentclass[a4paper, 11pt]{article}


\usepackage{amsmath}
\usepackage{amssymb}
\usepackage{hyperref}
\usepackage{url}
\usepackage{a4wide}
\usepackage[utf8]{inputenc}
\usepackage[main = russian, english]{babel}
\usepackage[pdftex]{graphicx}
\usepackage{float}
\usepackage{subcaption}
\usepackage{indentfirst}

% Красивый внешний вид теорем, определений и доказательств
\usepackage{amsthm}


\newenvironment{compactlist}{
        \begin{list}{{$\bullet$}}{
                        \setlength\partopsep{0pt}
                        \setlength\parskip{0pt}
                        \setlength\parsep{0pt}
                        \setlength\topsep{0pt}
                        \setlength\itemsep{0pt}
                }
        }{
        \end{list}
}
\theoremstyle{definition}
\newtheorem{definition}{Определение}

\theoremstyle{plane}
\newtheorem{theorem}{Теорема}
\newtheorem{assertion}{Утверждение}

\theoremstyle{remark}
\newtheorem{remark}{Замечание}

\renewcommand*{\proofname}{Доказательство}
\renewcommand\qedsymbol{$\blacksquare$}

\newcommand{\R}{\mathbb{R}}
\newcommand{\N}{\mathbb{N}}
\DeclareMathOperator{\sgn}{sgn}

\begin{document}
        \documentclass[a4paper, 11pt]{article}


\usepackage{amsmath}
\usepackage{amssymb}
\usepackage{hyperref}
\usepackage{url}
\usepackage{a4wide}
\usepackage[utf8]{inputenc}
\usepackage[main = russian, english]{babel}
\usepackage[pdftex]{graphicx}
\usepackage{float}
\usepackage{subcaption}
\usepackage{indentfirst}

% Красивый внешний вид теорем, определений и доказательств
\usepackage{amsthm}


\newenvironment{compactlist}{
        \begin{list}{{$\bullet$}}{
                        \setlength\partopsep{0pt}
                        \setlength\parskip{0pt}
                        \setlength\parsep{0pt}
                        \setlength\topsep{0pt}
                        \setlength\itemsep{0pt}
                }
        }{
        \end{list}
}
\theoremstyle{definition}
\newtheorem{definition}{Определение}

\theoremstyle{plane}
\newtheorem{theorem}{Теорема}
\newtheorem{assertion}{Утверждение}

\theoremstyle{remark}
\newtheorem{remark}{Замечание}

\renewcommand*{\proofname}{Доказательство}
\renewcommand\qedsymbol{$\blacksquare$}

\newcommand{\R}{\mathbb{R}}
\newcommand{\N}{\mathbb{N}}
\DeclareMathOperator{\sgn}{sgn}

\begin{document}
        \documentclass[a4paper, 11pt]{article}


\usepackage{amsmath}
\usepackage{amssymb}
\usepackage{hyperref}
\usepackage{url}
\usepackage{a4wide}
\usepackage[utf8]{inputenc}
\usepackage[main = russian, english]{babel}
\usepackage[pdftex]{graphicx}
\usepackage{float}
\usepackage{subcaption}
\usepackage{indentfirst}

% Красивый внешний вид теорем, определений и доказательств
\usepackage{amsthm}


\newenvironment{compactlist}{
        \begin{list}{{$\bullet$}}{
                        \setlength\partopsep{0pt}
                        \setlength\parskip{0pt}
                        \setlength\parsep{0pt}
                        \setlength\topsep{0pt}
                        \setlength\itemsep{0pt}
                }
        }{
        \end{list}
}
\theoremstyle{definition}
\newtheorem{definition}{Определение}

\theoremstyle{plane}
\newtheorem{theorem}{Теорема}
\newtheorem{assertion}{Утверждение}

\theoremstyle{remark}
\newtheorem{remark}{Замечание}

\renewcommand*{\proofname}{Доказательство}
\renewcommand\qedsymbol{$\blacksquare$}

\newcommand{\R}{\mathbb{R}}
\newcommand{\N}{\mathbb{N}}
\DeclareMathOperator{\sgn}{sgn}

\begin{document}
        \include{title_page/doc}

        \tableofcontents
        \clearpage
        
        \include{formulation_of_the_problem/doc}
        \include{research_of_the_system/doc}
        \include{algorithm/doc}
        \include{examples/doc}

        \begin{thebibliography}{9}
                \bibitem{pontryagin83} Л.~С.~Понтрягин, В.~Г.~Болтянский, Р.~В.~Гамрелидзе, Е.~Ф.~Мищенко. Математическая теория оптимальных процеccов. М.: Наука, 1983.
                \bibitem{li72} Э.~Б.~Ли, Л.~Маркус. Основы теории оптимального управления. М: Наука, 1972.
        \end{thebibliography} 
\end{document}

        \tableofcontents
        \clearpage
        
        \documentclass[a4paper, 11pt]{article}


\usepackage{amsmath}
\usepackage{amssymb}
\usepackage{hyperref}
\usepackage{url}
\usepackage{a4wide}
\usepackage[utf8]{inputenc}
\usepackage[main = russian, english]{babel}
\usepackage[pdftex]{graphicx}
\usepackage{float}
\usepackage{subcaption}
\usepackage{indentfirst}

% Красивый внешний вид теорем, определений и доказательств
\usepackage{amsthm}


\newenvironment{compactlist}{
        \begin{list}{{$\bullet$}}{
                        \setlength\partopsep{0pt}
                        \setlength\parskip{0pt}
                        \setlength\parsep{0pt}
                        \setlength\topsep{0pt}
                        \setlength\itemsep{0pt}
                }
        }{
        \end{list}
}
\theoremstyle{definition}
\newtheorem{definition}{Определение}

\theoremstyle{plane}
\newtheorem{theorem}{Теорема}
\newtheorem{assertion}{Утверждение}

\theoremstyle{remark}
\newtheorem{remark}{Замечание}

\renewcommand*{\proofname}{Доказательство}
\renewcommand\qedsymbol{$\blacksquare$}

\newcommand{\R}{\mathbb{R}}
\newcommand{\N}{\mathbb{N}}
\DeclareMathOperator{\sgn}{sgn}

\begin{document}
        \include{title_page/doc}

        \tableofcontents
        \clearpage
        
        \include{formulation_of_the_problem/doc}
        \include{research_of_the_system/doc}
        \include{algorithm/doc}
        \include{examples/doc}

        \begin{thebibliography}{9}
                \bibitem{pontryagin83} Л.~С.~Понтрягин, В.~Г.~Болтянский, Р.~В.~Гамрелидзе, Е.~Ф.~Мищенко. Математическая теория оптимальных процеccов. М.: Наука, 1983.
                \bibitem{li72} Э.~Б.~Ли, Л.~Маркус. Основы теории оптимального управления. М: Наука, 1972.
        \end{thebibliography} 
\end{document}
        \documentclass[a4paper, 11pt]{article}


\usepackage{amsmath}
\usepackage{amssymb}
\usepackage{hyperref}
\usepackage{url}
\usepackage{a4wide}
\usepackage[utf8]{inputenc}
\usepackage[main = russian, english]{babel}
\usepackage[pdftex]{graphicx}
\usepackage{float}
\usepackage{subcaption}
\usepackage{indentfirst}

% Красивый внешний вид теорем, определений и доказательств
\usepackage{amsthm}


\newenvironment{compactlist}{
        \begin{list}{{$\bullet$}}{
                        \setlength\partopsep{0pt}
                        \setlength\parskip{0pt}
                        \setlength\parsep{0pt}
                        \setlength\topsep{0pt}
                        \setlength\itemsep{0pt}
                }
        }{
        \end{list}
}
\theoremstyle{definition}
\newtheorem{definition}{Определение}

\theoremstyle{plane}
\newtheorem{theorem}{Теорема}
\newtheorem{assertion}{Утверждение}

\theoremstyle{remark}
\newtheorem{remark}{Замечание}

\renewcommand*{\proofname}{Доказательство}
\renewcommand\qedsymbol{$\blacksquare$}

\newcommand{\R}{\mathbb{R}}
\newcommand{\N}{\mathbb{N}}
\DeclareMathOperator{\sgn}{sgn}

\begin{document}
        \include{title_page/doc}

        \tableofcontents
        \clearpage
        
        \include{formulation_of_the_problem/doc}
        \include{research_of_the_system/doc}
        \include{algorithm/doc}
        \include{examples/doc}

        \begin{thebibliography}{9}
                \bibitem{pontryagin83} Л.~С.~Понтрягин, В.~Г.~Болтянский, Р.~В.~Гамрелидзе, Е.~Ф.~Мищенко. Математическая теория оптимальных процеccов. М.: Наука, 1983.
                \bibitem{li72} Э.~Б.~Ли, Л.~Маркус. Основы теории оптимального управления. М: Наука, 1972.
        \end{thebibliography} 
\end{document}
        \documentclass[a4paper, 11pt]{article}


\usepackage{amsmath}
\usepackage{amssymb}
\usepackage{hyperref}
\usepackage{url}
\usepackage{a4wide}
\usepackage[utf8]{inputenc}
\usepackage[main = russian, english]{babel}
\usepackage[pdftex]{graphicx}
\usepackage{float}
\usepackage{subcaption}
\usepackage{indentfirst}

% Красивый внешний вид теорем, определений и доказательств
\usepackage{amsthm}


\newenvironment{compactlist}{
        \begin{list}{{$\bullet$}}{
                        \setlength\partopsep{0pt}
                        \setlength\parskip{0pt}
                        \setlength\parsep{0pt}
                        \setlength\topsep{0pt}
                        \setlength\itemsep{0pt}
                }
        }{
        \end{list}
}
\theoremstyle{definition}
\newtheorem{definition}{Определение}

\theoremstyle{plane}
\newtheorem{theorem}{Теорема}
\newtheorem{assertion}{Утверждение}

\theoremstyle{remark}
\newtheorem{remark}{Замечание}

\renewcommand*{\proofname}{Доказательство}
\renewcommand\qedsymbol{$\blacksquare$}

\newcommand{\R}{\mathbb{R}}
\newcommand{\N}{\mathbb{N}}
\DeclareMathOperator{\sgn}{sgn}

\begin{document}
        \include{title_page/doc}

        \tableofcontents
        \clearpage
        
        \include{formulation_of_the_problem/doc}
        \include{research_of_the_system/doc}
        \include{algorithm/doc}
        \include{examples/doc}

        \begin{thebibliography}{9}
                \bibitem{pontryagin83} Л.~С.~Понтрягин, В.~Г.~Болтянский, Р.~В.~Гамрелидзе, Е.~Ф.~Мищенко. Математическая теория оптимальных процеccов. М.: Наука, 1983.
                \bibitem{li72} Э.~Б.~Ли, Л.~Маркус. Основы теории оптимального управления. М: Наука, 1972.
        \end{thebibliography} 
\end{document}
        \documentclass[a4paper, 11pt]{article}


\usepackage{amsmath}
\usepackage{amssymb}
\usepackage{hyperref}
\usepackage{url}
\usepackage{a4wide}
\usepackage[utf8]{inputenc}
\usepackage[main = russian, english]{babel}
\usepackage[pdftex]{graphicx}
\usepackage{float}
\usepackage{subcaption}
\usepackage{indentfirst}

% Красивый внешний вид теорем, определений и доказательств
\usepackage{amsthm}


\newenvironment{compactlist}{
        \begin{list}{{$\bullet$}}{
                        \setlength\partopsep{0pt}
                        \setlength\parskip{0pt}
                        \setlength\parsep{0pt}
                        \setlength\topsep{0pt}
                        \setlength\itemsep{0pt}
                }
        }{
        \end{list}
}
\theoremstyle{definition}
\newtheorem{definition}{Определение}

\theoremstyle{plane}
\newtheorem{theorem}{Теорема}
\newtheorem{assertion}{Утверждение}

\theoremstyle{remark}
\newtheorem{remark}{Замечание}

\renewcommand*{\proofname}{Доказательство}
\renewcommand\qedsymbol{$\blacksquare$}

\newcommand{\R}{\mathbb{R}}
\newcommand{\N}{\mathbb{N}}
\DeclareMathOperator{\sgn}{sgn}

\begin{document}
        \include{title_page/doc}

        \tableofcontents
        \clearpage
        
        \include{formulation_of_the_problem/doc}
        \include{research_of_the_system/doc}
        \include{algorithm/doc}
        \include{examples/doc}

        \begin{thebibliography}{9}
                \bibitem{pontryagin83} Л.~С.~Понтрягин, В.~Г.~Болтянский, Р.~В.~Гамрелидзе, Е.~Ф.~Мищенко. Математическая теория оптимальных процеccов. М.: Наука, 1983.
                \bibitem{li72} Э.~Б.~Ли, Л.~Маркус. Основы теории оптимального управления. М: Наука, 1972.
        \end{thebibliography} 
\end{document}

        \begin{thebibliography}{9}
                \bibitem{pontryagin83} Л.~С.~Понтрягин, В.~Г.~Болтянский, Р.~В.~Гамрелидзе, Е.~Ф.~Мищенко. Математическая теория оптимальных процеccов. М.: Наука, 1983.
                \bibitem{li72} Э.~Б.~Ли, Л.~Маркус. Основы теории оптимального управления. М: Наука, 1972.
        \end{thebibliography} 
\end{document}

        \tableofcontents
        \clearpage
        
        \documentclass[a4paper, 11pt]{article}


\usepackage{amsmath}
\usepackage{amssymb}
\usepackage{hyperref}
\usepackage{url}
\usepackage{a4wide}
\usepackage[utf8]{inputenc}
\usepackage[main = russian, english]{babel}
\usepackage[pdftex]{graphicx}
\usepackage{float}
\usepackage{subcaption}
\usepackage{indentfirst}

% Красивый внешний вид теорем, определений и доказательств
\usepackage{amsthm}


\newenvironment{compactlist}{
        \begin{list}{{$\bullet$}}{
                        \setlength\partopsep{0pt}
                        \setlength\parskip{0pt}
                        \setlength\parsep{0pt}
                        \setlength\topsep{0pt}
                        \setlength\itemsep{0pt}
                }
        }{
        \end{list}
}
\theoremstyle{definition}
\newtheorem{definition}{Определение}

\theoremstyle{plane}
\newtheorem{theorem}{Теорема}
\newtheorem{assertion}{Утверждение}

\theoremstyle{remark}
\newtheorem{remark}{Замечание}

\renewcommand*{\proofname}{Доказательство}
\renewcommand\qedsymbol{$\blacksquare$}

\newcommand{\R}{\mathbb{R}}
\newcommand{\N}{\mathbb{N}}
\DeclareMathOperator{\sgn}{sgn}

\begin{document}
        \documentclass[a4paper, 11pt]{article}


\usepackage{amsmath}
\usepackage{amssymb}
\usepackage{hyperref}
\usepackage{url}
\usepackage{a4wide}
\usepackage[utf8]{inputenc}
\usepackage[main = russian, english]{babel}
\usepackage[pdftex]{graphicx}
\usepackage{float}
\usepackage{subcaption}
\usepackage{indentfirst}

% Красивый внешний вид теорем, определений и доказательств
\usepackage{amsthm}


\newenvironment{compactlist}{
        \begin{list}{{$\bullet$}}{
                        \setlength\partopsep{0pt}
                        \setlength\parskip{0pt}
                        \setlength\parsep{0pt}
                        \setlength\topsep{0pt}
                        \setlength\itemsep{0pt}
                }
        }{
        \end{list}
}
\theoremstyle{definition}
\newtheorem{definition}{Определение}

\theoremstyle{plane}
\newtheorem{theorem}{Теорема}
\newtheorem{assertion}{Утверждение}

\theoremstyle{remark}
\newtheorem{remark}{Замечание}

\renewcommand*{\proofname}{Доказательство}
\renewcommand\qedsymbol{$\blacksquare$}

\newcommand{\R}{\mathbb{R}}
\newcommand{\N}{\mathbb{N}}
\DeclareMathOperator{\sgn}{sgn}

\begin{document}
        \include{title_page/doc}

        \tableofcontents
        \clearpage
        
        \include{formulation_of_the_problem/doc}
        \include{research_of_the_system/doc}
        \include{algorithm/doc}
        \include{examples/doc}

        \begin{thebibliography}{9}
                \bibitem{pontryagin83} Л.~С.~Понтрягин, В.~Г.~Болтянский, Р.~В.~Гамрелидзе, Е.~Ф.~Мищенко. Математическая теория оптимальных процеccов. М.: Наука, 1983.
                \bibitem{li72} Э.~Б.~Ли, Л.~Маркус. Основы теории оптимального управления. М: Наука, 1972.
        \end{thebibliography} 
\end{document}

        \tableofcontents
        \clearpage
        
        \documentclass[a4paper, 11pt]{article}


\usepackage{amsmath}
\usepackage{amssymb}
\usepackage{hyperref}
\usepackage{url}
\usepackage{a4wide}
\usepackage[utf8]{inputenc}
\usepackage[main = russian, english]{babel}
\usepackage[pdftex]{graphicx}
\usepackage{float}
\usepackage{subcaption}
\usepackage{indentfirst}

% Красивый внешний вид теорем, определений и доказательств
\usepackage{amsthm}


\newenvironment{compactlist}{
        \begin{list}{{$\bullet$}}{
                        \setlength\partopsep{0pt}
                        \setlength\parskip{0pt}
                        \setlength\parsep{0pt}
                        \setlength\topsep{0pt}
                        \setlength\itemsep{0pt}
                }
        }{
        \end{list}
}
\theoremstyle{definition}
\newtheorem{definition}{Определение}

\theoremstyle{plane}
\newtheorem{theorem}{Теорема}
\newtheorem{assertion}{Утверждение}

\theoremstyle{remark}
\newtheorem{remark}{Замечание}

\renewcommand*{\proofname}{Доказательство}
\renewcommand\qedsymbol{$\blacksquare$}

\newcommand{\R}{\mathbb{R}}
\newcommand{\N}{\mathbb{N}}
\DeclareMathOperator{\sgn}{sgn}

\begin{document}
        \include{title_page/doc}

        \tableofcontents
        \clearpage
        
        \include{formulation_of_the_problem/doc}
        \include{research_of_the_system/doc}
        \include{algorithm/doc}
        \include{examples/doc}

        \begin{thebibliography}{9}
                \bibitem{pontryagin83} Л.~С.~Понтрягин, В.~Г.~Болтянский, Р.~В.~Гамрелидзе, Е.~Ф.~Мищенко. Математическая теория оптимальных процеccов. М.: Наука, 1983.
                \bibitem{li72} Э.~Б.~Ли, Л.~Маркус. Основы теории оптимального управления. М: Наука, 1972.
        \end{thebibliography} 
\end{document}
        \documentclass[a4paper, 11pt]{article}


\usepackage{amsmath}
\usepackage{amssymb}
\usepackage{hyperref}
\usepackage{url}
\usepackage{a4wide}
\usepackage[utf8]{inputenc}
\usepackage[main = russian, english]{babel}
\usepackage[pdftex]{graphicx}
\usepackage{float}
\usepackage{subcaption}
\usepackage{indentfirst}

% Красивый внешний вид теорем, определений и доказательств
\usepackage{amsthm}


\newenvironment{compactlist}{
        \begin{list}{{$\bullet$}}{
                        \setlength\partopsep{0pt}
                        \setlength\parskip{0pt}
                        \setlength\parsep{0pt}
                        \setlength\topsep{0pt}
                        \setlength\itemsep{0pt}
                }
        }{
        \end{list}
}
\theoremstyle{definition}
\newtheorem{definition}{Определение}

\theoremstyle{plane}
\newtheorem{theorem}{Теорема}
\newtheorem{assertion}{Утверждение}

\theoremstyle{remark}
\newtheorem{remark}{Замечание}

\renewcommand*{\proofname}{Доказательство}
\renewcommand\qedsymbol{$\blacksquare$}

\newcommand{\R}{\mathbb{R}}
\newcommand{\N}{\mathbb{N}}
\DeclareMathOperator{\sgn}{sgn}

\begin{document}
        \include{title_page/doc}

        \tableofcontents
        \clearpage
        
        \include{formulation_of_the_problem/doc}
        \include{research_of_the_system/doc}
        \include{algorithm/doc}
        \include{examples/doc}

        \begin{thebibliography}{9}
                \bibitem{pontryagin83} Л.~С.~Понтрягин, В.~Г.~Болтянский, Р.~В.~Гамрелидзе, Е.~Ф.~Мищенко. Математическая теория оптимальных процеccов. М.: Наука, 1983.
                \bibitem{li72} Э.~Б.~Ли, Л.~Маркус. Основы теории оптимального управления. М: Наука, 1972.
        \end{thebibliography} 
\end{document}
        \documentclass[a4paper, 11pt]{article}


\usepackage{amsmath}
\usepackage{amssymb}
\usepackage{hyperref}
\usepackage{url}
\usepackage{a4wide}
\usepackage[utf8]{inputenc}
\usepackage[main = russian, english]{babel}
\usepackage[pdftex]{graphicx}
\usepackage{float}
\usepackage{subcaption}
\usepackage{indentfirst}

% Красивый внешний вид теорем, определений и доказательств
\usepackage{amsthm}


\newenvironment{compactlist}{
        \begin{list}{{$\bullet$}}{
                        \setlength\partopsep{0pt}
                        \setlength\parskip{0pt}
                        \setlength\parsep{0pt}
                        \setlength\topsep{0pt}
                        \setlength\itemsep{0pt}
                }
        }{
        \end{list}
}
\theoremstyle{definition}
\newtheorem{definition}{Определение}

\theoremstyle{plane}
\newtheorem{theorem}{Теорема}
\newtheorem{assertion}{Утверждение}

\theoremstyle{remark}
\newtheorem{remark}{Замечание}

\renewcommand*{\proofname}{Доказательство}
\renewcommand\qedsymbol{$\blacksquare$}

\newcommand{\R}{\mathbb{R}}
\newcommand{\N}{\mathbb{N}}
\DeclareMathOperator{\sgn}{sgn}

\begin{document}
        \include{title_page/doc}

        \tableofcontents
        \clearpage
        
        \include{formulation_of_the_problem/doc}
        \include{research_of_the_system/doc}
        \include{algorithm/doc}
        \include{examples/doc}

        \begin{thebibliography}{9}
                \bibitem{pontryagin83} Л.~С.~Понтрягин, В.~Г.~Болтянский, Р.~В.~Гамрелидзе, Е.~Ф.~Мищенко. Математическая теория оптимальных процеccов. М.: Наука, 1983.
                \bibitem{li72} Э.~Б.~Ли, Л.~Маркус. Основы теории оптимального управления. М: Наука, 1972.
        \end{thebibliography} 
\end{document}
        \documentclass[a4paper, 11pt]{article}


\usepackage{amsmath}
\usepackage{amssymb}
\usepackage{hyperref}
\usepackage{url}
\usepackage{a4wide}
\usepackage[utf8]{inputenc}
\usepackage[main = russian, english]{babel}
\usepackage[pdftex]{graphicx}
\usepackage{float}
\usepackage{subcaption}
\usepackage{indentfirst}

% Красивый внешний вид теорем, определений и доказательств
\usepackage{amsthm}


\newenvironment{compactlist}{
        \begin{list}{{$\bullet$}}{
                        \setlength\partopsep{0pt}
                        \setlength\parskip{0pt}
                        \setlength\parsep{0pt}
                        \setlength\topsep{0pt}
                        \setlength\itemsep{0pt}
                }
        }{
        \end{list}
}
\theoremstyle{definition}
\newtheorem{definition}{Определение}

\theoremstyle{plane}
\newtheorem{theorem}{Теорема}
\newtheorem{assertion}{Утверждение}

\theoremstyle{remark}
\newtheorem{remark}{Замечание}

\renewcommand*{\proofname}{Доказательство}
\renewcommand\qedsymbol{$\blacksquare$}

\newcommand{\R}{\mathbb{R}}
\newcommand{\N}{\mathbb{N}}
\DeclareMathOperator{\sgn}{sgn}

\begin{document}
        \include{title_page/doc}

        \tableofcontents
        \clearpage
        
        \include{formulation_of_the_problem/doc}
        \include{research_of_the_system/doc}
        \include{algorithm/doc}
        \include{examples/doc}

        \begin{thebibliography}{9}
                \bibitem{pontryagin83} Л.~С.~Понтрягин, В.~Г.~Болтянский, Р.~В.~Гамрелидзе, Е.~Ф.~Мищенко. Математическая теория оптимальных процеccов. М.: Наука, 1983.
                \bibitem{li72} Э.~Б.~Ли, Л.~Маркус. Основы теории оптимального управления. М: Наука, 1972.
        \end{thebibliography} 
\end{document}

        \begin{thebibliography}{9}
                \bibitem{pontryagin83} Л.~С.~Понтрягин, В.~Г.~Болтянский, Р.~В.~Гамрелидзе, Е.~Ф.~Мищенко. Математическая теория оптимальных процеccов. М.: Наука, 1983.
                \bibitem{li72} Э.~Б.~Ли, Л.~Маркус. Основы теории оптимального управления. М: Наука, 1972.
        \end{thebibliography} 
\end{document}
        \documentclass[a4paper, 11pt]{article}


\usepackage{amsmath}
\usepackage{amssymb}
\usepackage{hyperref}
\usepackage{url}
\usepackage{a4wide}
\usepackage[utf8]{inputenc}
\usepackage[main = russian, english]{babel}
\usepackage[pdftex]{graphicx}
\usepackage{float}
\usepackage{subcaption}
\usepackage{indentfirst}

% Красивый внешний вид теорем, определений и доказательств
\usepackage{amsthm}


\newenvironment{compactlist}{
        \begin{list}{{$\bullet$}}{
                        \setlength\partopsep{0pt}
                        \setlength\parskip{0pt}
                        \setlength\parsep{0pt}
                        \setlength\topsep{0pt}
                        \setlength\itemsep{0pt}
                }
        }{
        \end{list}
}
\theoremstyle{definition}
\newtheorem{definition}{Определение}

\theoremstyle{plane}
\newtheorem{theorem}{Теорема}
\newtheorem{assertion}{Утверждение}

\theoremstyle{remark}
\newtheorem{remark}{Замечание}

\renewcommand*{\proofname}{Доказательство}
\renewcommand\qedsymbol{$\blacksquare$}

\newcommand{\R}{\mathbb{R}}
\newcommand{\N}{\mathbb{N}}
\DeclareMathOperator{\sgn}{sgn}

\begin{document}
        \documentclass[a4paper, 11pt]{article}


\usepackage{amsmath}
\usepackage{amssymb}
\usepackage{hyperref}
\usepackage{url}
\usepackage{a4wide}
\usepackage[utf8]{inputenc}
\usepackage[main = russian, english]{babel}
\usepackage[pdftex]{graphicx}
\usepackage{float}
\usepackage{subcaption}
\usepackage{indentfirst}

% Красивый внешний вид теорем, определений и доказательств
\usepackage{amsthm}


\newenvironment{compactlist}{
        \begin{list}{{$\bullet$}}{
                        \setlength\partopsep{0pt}
                        \setlength\parskip{0pt}
                        \setlength\parsep{0pt}
                        \setlength\topsep{0pt}
                        \setlength\itemsep{0pt}
                }
        }{
        \end{list}
}
\theoremstyle{definition}
\newtheorem{definition}{Определение}

\theoremstyle{plane}
\newtheorem{theorem}{Теорема}
\newtheorem{assertion}{Утверждение}

\theoremstyle{remark}
\newtheorem{remark}{Замечание}

\renewcommand*{\proofname}{Доказательство}
\renewcommand\qedsymbol{$\blacksquare$}

\newcommand{\R}{\mathbb{R}}
\newcommand{\N}{\mathbb{N}}
\DeclareMathOperator{\sgn}{sgn}

\begin{document}
        \include{title_page/doc}

        \tableofcontents
        \clearpage
        
        \include{formulation_of_the_problem/doc}
        \include{research_of_the_system/doc}
        \include{algorithm/doc}
        \include{examples/doc}

        \begin{thebibliography}{9}
                \bibitem{pontryagin83} Л.~С.~Понтрягин, В.~Г.~Болтянский, Р.~В.~Гамрелидзе, Е.~Ф.~Мищенко. Математическая теория оптимальных процеccов. М.: Наука, 1983.
                \bibitem{li72} Э.~Б.~Ли, Л.~Маркус. Основы теории оптимального управления. М: Наука, 1972.
        \end{thebibliography} 
\end{document}

        \tableofcontents
        \clearpage
        
        \documentclass[a4paper, 11pt]{article}


\usepackage{amsmath}
\usepackage{amssymb}
\usepackage{hyperref}
\usepackage{url}
\usepackage{a4wide}
\usepackage[utf8]{inputenc}
\usepackage[main = russian, english]{babel}
\usepackage[pdftex]{graphicx}
\usepackage{float}
\usepackage{subcaption}
\usepackage{indentfirst}

% Красивый внешний вид теорем, определений и доказательств
\usepackage{amsthm}


\newenvironment{compactlist}{
        \begin{list}{{$\bullet$}}{
                        \setlength\partopsep{0pt}
                        \setlength\parskip{0pt}
                        \setlength\parsep{0pt}
                        \setlength\topsep{0pt}
                        \setlength\itemsep{0pt}
                }
        }{
        \end{list}
}
\theoremstyle{definition}
\newtheorem{definition}{Определение}

\theoremstyle{plane}
\newtheorem{theorem}{Теорема}
\newtheorem{assertion}{Утверждение}

\theoremstyle{remark}
\newtheorem{remark}{Замечание}

\renewcommand*{\proofname}{Доказательство}
\renewcommand\qedsymbol{$\blacksquare$}

\newcommand{\R}{\mathbb{R}}
\newcommand{\N}{\mathbb{N}}
\DeclareMathOperator{\sgn}{sgn}

\begin{document}
        \include{title_page/doc}

        \tableofcontents
        \clearpage
        
        \include{formulation_of_the_problem/doc}
        \include{research_of_the_system/doc}
        \include{algorithm/doc}
        \include{examples/doc}

        \begin{thebibliography}{9}
                \bibitem{pontryagin83} Л.~С.~Понтрягин, В.~Г.~Болтянский, Р.~В.~Гамрелидзе, Е.~Ф.~Мищенко. Математическая теория оптимальных процеccов. М.: Наука, 1983.
                \bibitem{li72} Э.~Б.~Ли, Л.~Маркус. Основы теории оптимального управления. М: Наука, 1972.
        \end{thebibliography} 
\end{document}
        \documentclass[a4paper, 11pt]{article}


\usepackage{amsmath}
\usepackage{amssymb}
\usepackage{hyperref}
\usepackage{url}
\usepackage{a4wide}
\usepackage[utf8]{inputenc}
\usepackage[main = russian, english]{babel}
\usepackage[pdftex]{graphicx}
\usepackage{float}
\usepackage{subcaption}
\usepackage{indentfirst}

% Красивый внешний вид теорем, определений и доказательств
\usepackage{amsthm}


\newenvironment{compactlist}{
        \begin{list}{{$\bullet$}}{
                        \setlength\partopsep{0pt}
                        \setlength\parskip{0pt}
                        \setlength\parsep{0pt}
                        \setlength\topsep{0pt}
                        \setlength\itemsep{0pt}
                }
        }{
        \end{list}
}
\theoremstyle{definition}
\newtheorem{definition}{Определение}

\theoremstyle{plane}
\newtheorem{theorem}{Теорема}
\newtheorem{assertion}{Утверждение}

\theoremstyle{remark}
\newtheorem{remark}{Замечание}

\renewcommand*{\proofname}{Доказательство}
\renewcommand\qedsymbol{$\blacksquare$}

\newcommand{\R}{\mathbb{R}}
\newcommand{\N}{\mathbb{N}}
\DeclareMathOperator{\sgn}{sgn}

\begin{document}
        \include{title_page/doc}

        \tableofcontents
        \clearpage
        
        \include{formulation_of_the_problem/doc}
        \include{research_of_the_system/doc}
        \include{algorithm/doc}
        \include{examples/doc}

        \begin{thebibliography}{9}
                \bibitem{pontryagin83} Л.~С.~Понтрягин, В.~Г.~Болтянский, Р.~В.~Гамрелидзе, Е.~Ф.~Мищенко. Математическая теория оптимальных процеccов. М.: Наука, 1983.
                \bibitem{li72} Э.~Б.~Ли, Л.~Маркус. Основы теории оптимального управления. М: Наука, 1972.
        \end{thebibliography} 
\end{document}
        \documentclass[a4paper, 11pt]{article}


\usepackage{amsmath}
\usepackage{amssymb}
\usepackage{hyperref}
\usepackage{url}
\usepackage{a4wide}
\usepackage[utf8]{inputenc}
\usepackage[main = russian, english]{babel}
\usepackage[pdftex]{graphicx}
\usepackage{float}
\usepackage{subcaption}
\usepackage{indentfirst}

% Красивый внешний вид теорем, определений и доказательств
\usepackage{amsthm}


\newenvironment{compactlist}{
        \begin{list}{{$\bullet$}}{
                        \setlength\partopsep{0pt}
                        \setlength\parskip{0pt}
                        \setlength\parsep{0pt}
                        \setlength\topsep{0pt}
                        \setlength\itemsep{0pt}
                }
        }{
        \end{list}
}
\theoremstyle{definition}
\newtheorem{definition}{Определение}

\theoremstyle{plane}
\newtheorem{theorem}{Теорема}
\newtheorem{assertion}{Утверждение}

\theoremstyle{remark}
\newtheorem{remark}{Замечание}

\renewcommand*{\proofname}{Доказательство}
\renewcommand\qedsymbol{$\blacksquare$}

\newcommand{\R}{\mathbb{R}}
\newcommand{\N}{\mathbb{N}}
\DeclareMathOperator{\sgn}{sgn}

\begin{document}
        \include{title_page/doc}

        \tableofcontents
        \clearpage
        
        \include{formulation_of_the_problem/doc}
        \include{research_of_the_system/doc}
        \include{algorithm/doc}
        \include{examples/doc}

        \begin{thebibliography}{9}
                \bibitem{pontryagin83} Л.~С.~Понтрягин, В.~Г.~Болтянский, Р.~В.~Гамрелидзе, Е.~Ф.~Мищенко. Математическая теория оптимальных процеccов. М.: Наука, 1983.
                \bibitem{li72} Э.~Б.~Ли, Л.~Маркус. Основы теории оптимального управления. М: Наука, 1972.
        \end{thebibliography} 
\end{document}
        \documentclass[a4paper, 11pt]{article}


\usepackage{amsmath}
\usepackage{amssymb}
\usepackage{hyperref}
\usepackage{url}
\usepackage{a4wide}
\usepackage[utf8]{inputenc}
\usepackage[main = russian, english]{babel}
\usepackage[pdftex]{graphicx}
\usepackage{float}
\usepackage{subcaption}
\usepackage{indentfirst}

% Красивый внешний вид теорем, определений и доказательств
\usepackage{amsthm}


\newenvironment{compactlist}{
        \begin{list}{{$\bullet$}}{
                        \setlength\partopsep{0pt}
                        \setlength\parskip{0pt}
                        \setlength\parsep{0pt}
                        \setlength\topsep{0pt}
                        \setlength\itemsep{0pt}
                }
        }{
        \end{list}
}
\theoremstyle{definition}
\newtheorem{definition}{Определение}

\theoremstyle{plane}
\newtheorem{theorem}{Теорема}
\newtheorem{assertion}{Утверждение}

\theoremstyle{remark}
\newtheorem{remark}{Замечание}

\renewcommand*{\proofname}{Доказательство}
\renewcommand\qedsymbol{$\blacksquare$}

\newcommand{\R}{\mathbb{R}}
\newcommand{\N}{\mathbb{N}}
\DeclareMathOperator{\sgn}{sgn}

\begin{document}
        \include{title_page/doc}

        \tableofcontents
        \clearpage
        
        \include{formulation_of_the_problem/doc}
        \include{research_of_the_system/doc}
        \include{algorithm/doc}
        \include{examples/doc}

        \begin{thebibliography}{9}
                \bibitem{pontryagin83} Л.~С.~Понтрягин, В.~Г.~Болтянский, Р.~В.~Гамрелидзе, Е.~Ф.~Мищенко. Математическая теория оптимальных процеccов. М.: Наука, 1983.
                \bibitem{li72} Э.~Б.~Ли, Л.~Маркус. Основы теории оптимального управления. М: Наука, 1972.
        \end{thebibliography} 
\end{document}

        \begin{thebibliography}{9}
                \bibitem{pontryagin83} Л.~С.~Понтрягин, В.~Г.~Болтянский, Р.~В.~Гамрелидзе, Е.~Ф.~Мищенко. Математическая теория оптимальных процеccов. М.: Наука, 1983.
                \bibitem{li72} Э.~Б.~Ли, Л.~Маркус. Основы теории оптимального управления. М: Наука, 1972.
        \end{thebibliography} 
\end{document}
        \documentclass[a4paper, 11pt]{article}


\usepackage{amsmath}
\usepackage{amssymb}
\usepackage{hyperref}
\usepackage{url}
\usepackage{a4wide}
\usepackage[utf8]{inputenc}
\usepackage[main = russian, english]{babel}
\usepackage[pdftex]{graphicx}
\usepackage{float}
\usepackage{subcaption}
\usepackage{indentfirst}

% Красивый внешний вид теорем, определений и доказательств
\usepackage{amsthm}


\newenvironment{compactlist}{
        \begin{list}{{$\bullet$}}{
                        \setlength\partopsep{0pt}
                        \setlength\parskip{0pt}
                        \setlength\parsep{0pt}
                        \setlength\topsep{0pt}
                        \setlength\itemsep{0pt}
                }
        }{
        \end{list}
}
\theoremstyle{definition}
\newtheorem{definition}{Определение}

\theoremstyle{plane}
\newtheorem{theorem}{Теорема}
\newtheorem{assertion}{Утверждение}

\theoremstyle{remark}
\newtheorem{remark}{Замечание}

\renewcommand*{\proofname}{Доказательство}
\renewcommand\qedsymbol{$\blacksquare$}

\newcommand{\R}{\mathbb{R}}
\newcommand{\N}{\mathbb{N}}
\DeclareMathOperator{\sgn}{sgn}

\begin{document}
        \documentclass[a4paper, 11pt]{article}


\usepackage{amsmath}
\usepackage{amssymb}
\usepackage{hyperref}
\usepackage{url}
\usepackage{a4wide}
\usepackage[utf8]{inputenc}
\usepackage[main = russian, english]{babel}
\usepackage[pdftex]{graphicx}
\usepackage{float}
\usepackage{subcaption}
\usepackage{indentfirst}

% Красивый внешний вид теорем, определений и доказательств
\usepackage{amsthm}


\newenvironment{compactlist}{
        \begin{list}{{$\bullet$}}{
                        \setlength\partopsep{0pt}
                        \setlength\parskip{0pt}
                        \setlength\parsep{0pt}
                        \setlength\topsep{0pt}
                        \setlength\itemsep{0pt}
                }
        }{
        \end{list}
}
\theoremstyle{definition}
\newtheorem{definition}{Определение}

\theoremstyle{plane}
\newtheorem{theorem}{Теорема}
\newtheorem{assertion}{Утверждение}

\theoremstyle{remark}
\newtheorem{remark}{Замечание}

\renewcommand*{\proofname}{Доказательство}
\renewcommand\qedsymbol{$\blacksquare$}

\newcommand{\R}{\mathbb{R}}
\newcommand{\N}{\mathbb{N}}
\DeclareMathOperator{\sgn}{sgn}

\begin{document}
        \include{title_page/doc}

        \tableofcontents
        \clearpage
        
        \include{formulation_of_the_problem/doc}
        \include{research_of_the_system/doc}
        \include{algorithm/doc}
        \include{examples/doc}

        \begin{thebibliography}{9}
                \bibitem{pontryagin83} Л.~С.~Понтрягин, В.~Г.~Болтянский, Р.~В.~Гамрелидзе, Е.~Ф.~Мищенко. Математическая теория оптимальных процеccов. М.: Наука, 1983.
                \bibitem{li72} Э.~Б.~Ли, Л.~Маркус. Основы теории оптимального управления. М: Наука, 1972.
        \end{thebibliography} 
\end{document}

        \tableofcontents
        \clearpage
        
        \documentclass[a4paper, 11pt]{article}


\usepackage{amsmath}
\usepackage{amssymb}
\usepackage{hyperref}
\usepackage{url}
\usepackage{a4wide}
\usepackage[utf8]{inputenc}
\usepackage[main = russian, english]{babel}
\usepackage[pdftex]{graphicx}
\usepackage{float}
\usepackage{subcaption}
\usepackage{indentfirst}

% Красивый внешний вид теорем, определений и доказательств
\usepackage{amsthm}


\newenvironment{compactlist}{
        \begin{list}{{$\bullet$}}{
                        \setlength\partopsep{0pt}
                        \setlength\parskip{0pt}
                        \setlength\parsep{0pt}
                        \setlength\topsep{0pt}
                        \setlength\itemsep{0pt}
                }
        }{
        \end{list}
}
\theoremstyle{definition}
\newtheorem{definition}{Определение}

\theoremstyle{plane}
\newtheorem{theorem}{Теорема}
\newtheorem{assertion}{Утверждение}

\theoremstyle{remark}
\newtheorem{remark}{Замечание}

\renewcommand*{\proofname}{Доказательство}
\renewcommand\qedsymbol{$\blacksquare$}

\newcommand{\R}{\mathbb{R}}
\newcommand{\N}{\mathbb{N}}
\DeclareMathOperator{\sgn}{sgn}

\begin{document}
        \include{title_page/doc}

        \tableofcontents
        \clearpage
        
        \include{formulation_of_the_problem/doc}
        \include{research_of_the_system/doc}
        \include{algorithm/doc}
        \include{examples/doc}

        \begin{thebibliography}{9}
                \bibitem{pontryagin83} Л.~С.~Понтрягин, В.~Г.~Болтянский, Р.~В.~Гамрелидзе, Е.~Ф.~Мищенко. Математическая теория оптимальных процеccов. М.: Наука, 1983.
                \bibitem{li72} Э.~Б.~Ли, Л.~Маркус. Основы теории оптимального управления. М: Наука, 1972.
        \end{thebibliography} 
\end{document}
        \documentclass[a4paper, 11pt]{article}


\usepackage{amsmath}
\usepackage{amssymb}
\usepackage{hyperref}
\usepackage{url}
\usepackage{a4wide}
\usepackage[utf8]{inputenc}
\usepackage[main = russian, english]{babel}
\usepackage[pdftex]{graphicx}
\usepackage{float}
\usepackage{subcaption}
\usepackage{indentfirst}

% Красивый внешний вид теорем, определений и доказательств
\usepackage{amsthm}


\newenvironment{compactlist}{
        \begin{list}{{$\bullet$}}{
                        \setlength\partopsep{0pt}
                        \setlength\parskip{0pt}
                        \setlength\parsep{0pt}
                        \setlength\topsep{0pt}
                        \setlength\itemsep{0pt}
                }
        }{
        \end{list}
}
\theoremstyle{definition}
\newtheorem{definition}{Определение}

\theoremstyle{plane}
\newtheorem{theorem}{Теорема}
\newtheorem{assertion}{Утверждение}

\theoremstyle{remark}
\newtheorem{remark}{Замечание}

\renewcommand*{\proofname}{Доказательство}
\renewcommand\qedsymbol{$\blacksquare$}

\newcommand{\R}{\mathbb{R}}
\newcommand{\N}{\mathbb{N}}
\DeclareMathOperator{\sgn}{sgn}

\begin{document}
        \include{title_page/doc}

        \tableofcontents
        \clearpage
        
        \include{formulation_of_the_problem/doc}
        \include{research_of_the_system/doc}
        \include{algorithm/doc}
        \include{examples/doc}

        \begin{thebibliography}{9}
                \bibitem{pontryagin83} Л.~С.~Понтрягин, В.~Г.~Болтянский, Р.~В.~Гамрелидзе, Е.~Ф.~Мищенко. Математическая теория оптимальных процеccов. М.: Наука, 1983.
                \bibitem{li72} Э.~Б.~Ли, Л.~Маркус. Основы теории оптимального управления. М: Наука, 1972.
        \end{thebibliography} 
\end{document}
        \documentclass[a4paper, 11pt]{article}


\usepackage{amsmath}
\usepackage{amssymb}
\usepackage{hyperref}
\usepackage{url}
\usepackage{a4wide}
\usepackage[utf8]{inputenc}
\usepackage[main = russian, english]{babel}
\usepackage[pdftex]{graphicx}
\usepackage{float}
\usepackage{subcaption}
\usepackage{indentfirst}

% Красивый внешний вид теорем, определений и доказательств
\usepackage{amsthm}


\newenvironment{compactlist}{
        \begin{list}{{$\bullet$}}{
                        \setlength\partopsep{0pt}
                        \setlength\parskip{0pt}
                        \setlength\parsep{0pt}
                        \setlength\topsep{0pt}
                        \setlength\itemsep{0pt}
                }
        }{
        \end{list}
}
\theoremstyle{definition}
\newtheorem{definition}{Определение}

\theoremstyle{plane}
\newtheorem{theorem}{Теорема}
\newtheorem{assertion}{Утверждение}

\theoremstyle{remark}
\newtheorem{remark}{Замечание}

\renewcommand*{\proofname}{Доказательство}
\renewcommand\qedsymbol{$\blacksquare$}

\newcommand{\R}{\mathbb{R}}
\newcommand{\N}{\mathbb{N}}
\DeclareMathOperator{\sgn}{sgn}

\begin{document}
        \include{title_page/doc}

        \tableofcontents
        \clearpage
        
        \include{formulation_of_the_problem/doc}
        \include{research_of_the_system/doc}
        \include{algorithm/doc}
        \include{examples/doc}

        \begin{thebibliography}{9}
                \bibitem{pontryagin83} Л.~С.~Понтрягин, В.~Г.~Болтянский, Р.~В.~Гамрелидзе, Е.~Ф.~Мищенко. Математическая теория оптимальных процеccов. М.: Наука, 1983.
                \bibitem{li72} Э.~Б.~Ли, Л.~Маркус. Основы теории оптимального управления. М: Наука, 1972.
        \end{thebibliography} 
\end{document}
        \documentclass[a4paper, 11pt]{article}


\usepackage{amsmath}
\usepackage{amssymb}
\usepackage{hyperref}
\usepackage{url}
\usepackage{a4wide}
\usepackage[utf8]{inputenc}
\usepackage[main = russian, english]{babel}
\usepackage[pdftex]{graphicx}
\usepackage{float}
\usepackage{subcaption}
\usepackage{indentfirst}

% Красивый внешний вид теорем, определений и доказательств
\usepackage{amsthm}


\newenvironment{compactlist}{
        \begin{list}{{$\bullet$}}{
                        \setlength\partopsep{0pt}
                        \setlength\parskip{0pt}
                        \setlength\parsep{0pt}
                        \setlength\topsep{0pt}
                        \setlength\itemsep{0pt}
                }
        }{
        \end{list}
}
\theoremstyle{definition}
\newtheorem{definition}{Определение}

\theoremstyle{plane}
\newtheorem{theorem}{Теорема}
\newtheorem{assertion}{Утверждение}

\theoremstyle{remark}
\newtheorem{remark}{Замечание}

\renewcommand*{\proofname}{Доказательство}
\renewcommand\qedsymbol{$\blacksquare$}

\newcommand{\R}{\mathbb{R}}
\newcommand{\N}{\mathbb{N}}
\DeclareMathOperator{\sgn}{sgn}

\begin{document}
        \include{title_page/doc}

        \tableofcontents
        \clearpage
        
        \include{formulation_of_the_problem/doc}
        \include{research_of_the_system/doc}
        \include{algorithm/doc}
        \include{examples/doc}

        \begin{thebibliography}{9}
                \bibitem{pontryagin83} Л.~С.~Понтрягин, В.~Г.~Болтянский, Р.~В.~Гамрелидзе, Е.~Ф.~Мищенко. Математическая теория оптимальных процеccов. М.: Наука, 1983.
                \bibitem{li72} Э.~Б.~Ли, Л.~Маркус. Основы теории оптимального управления. М: Наука, 1972.
        \end{thebibliography} 
\end{document}

        \begin{thebibliography}{9}
                \bibitem{pontryagin83} Л.~С.~Понтрягин, В.~Г.~Болтянский, Р.~В.~Гамрелидзе, Е.~Ф.~Мищенко. Математическая теория оптимальных процеccов. М.: Наука, 1983.
                \bibitem{li72} Э.~Б.~Ли, Л.~Маркус. Основы теории оптимального управления. М: Наука, 1972.
        \end{thebibliography} 
\end{document}
        \documentclass[a4paper, 11pt]{article}


\usepackage{amsmath}
\usepackage{amssymb}
\usepackage{hyperref}
\usepackage{url}
\usepackage{a4wide}
\usepackage[utf8]{inputenc}
\usepackage[main = russian, english]{babel}
\usepackage[pdftex]{graphicx}
\usepackage{float}
\usepackage{subcaption}
\usepackage{indentfirst}

% Красивый внешний вид теорем, определений и доказательств
\usepackage{amsthm}


\newenvironment{compactlist}{
        \begin{list}{{$\bullet$}}{
                        \setlength\partopsep{0pt}
                        \setlength\parskip{0pt}
                        \setlength\parsep{0pt}
                        \setlength\topsep{0pt}
                        \setlength\itemsep{0pt}
                }
        }{
        \end{list}
}
\theoremstyle{definition}
\newtheorem{definition}{Определение}

\theoremstyle{plane}
\newtheorem{theorem}{Теорема}
\newtheorem{assertion}{Утверждение}

\theoremstyle{remark}
\newtheorem{remark}{Замечание}

\renewcommand*{\proofname}{Доказательство}
\renewcommand\qedsymbol{$\blacksquare$}

\newcommand{\R}{\mathbb{R}}
\newcommand{\N}{\mathbb{N}}
\DeclareMathOperator{\sgn}{sgn}

\begin{document}
        \documentclass[a4paper, 11pt]{article}


\usepackage{amsmath}
\usepackage{amssymb}
\usepackage{hyperref}
\usepackage{url}
\usepackage{a4wide}
\usepackage[utf8]{inputenc}
\usepackage[main = russian, english]{babel}
\usepackage[pdftex]{graphicx}
\usepackage{float}
\usepackage{subcaption}
\usepackage{indentfirst}

% Красивый внешний вид теорем, определений и доказательств
\usepackage{amsthm}


\newenvironment{compactlist}{
        \begin{list}{{$\bullet$}}{
                        \setlength\partopsep{0pt}
                        \setlength\parskip{0pt}
                        \setlength\parsep{0pt}
                        \setlength\topsep{0pt}
                        \setlength\itemsep{0pt}
                }
        }{
        \end{list}
}
\theoremstyle{definition}
\newtheorem{definition}{Определение}

\theoremstyle{plane}
\newtheorem{theorem}{Теорема}
\newtheorem{assertion}{Утверждение}

\theoremstyle{remark}
\newtheorem{remark}{Замечание}

\renewcommand*{\proofname}{Доказательство}
\renewcommand\qedsymbol{$\blacksquare$}

\newcommand{\R}{\mathbb{R}}
\newcommand{\N}{\mathbb{N}}
\DeclareMathOperator{\sgn}{sgn}

\begin{document}
        \include{title_page/doc}

        \tableofcontents
        \clearpage
        
        \include{formulation_of_the_problem/doc}
        \include{research_of_the_system/doc}
        \include{algorithm/doc}
        \include{examples/doc}

        \begin{thebibliography}{9}
                \bibitem{pontryagin83} Л.~С.~Понтрягин, В.~Г.~Болтянский, Р.~В.~Гамрелидзе, Е.~Ф.~Мищенко. Математическая теория оптимальных процеccов. М.: Наука, 1983.
                \bibitem{li72} Э.~Б.~Ли, Л.~Маркус. Основы теории оптимального управления. М: Наука, 1972.
        \end{thebibliography} 
\end{document}

        \tableofcontents
        \clearpage
        
        \documentclass[a4paper, 11pt]{article}


\usepackage{amsmath}
\usepackage{amssymb}
\usepackage{hyperref}
\usepackage{url}
\usepackage{a4wide}
\usepackage[utf8]{inputenc}
\usepackage[main = russian, english]{babel}
\usepackage[pdftex]{graphicx}
\usepackage{float}
\usepackage{subcaption}
\usepackage{indentfirst}

% Красивый внешний вид теорем, определений и доказательств
\usepackage{amsthm}


\newenvironment{compactlist}{
        \begin{list}{{$\bullet$}}{
                        \setlength\partopsep{0pt}
                        \setlength\parskip{0pt}
                        \setlength\parsep{0pt}
                        \setlength\topsep{0pt}
                        \setlength\itemsep{0pt}
                }
        }{
        \end{list}
}
\theoremstyle{definition}
\newtheorem{definition}{Определение}

\theoremstyle{plane}
\newtheorem{theorem}{Теорема}
\newtheorem{assertion}{Утверждение}

\theoremstyle{remark}
\newtheorem{remark}{Замечание}

\renewcommand*{\proofname}{Доказательство}
\renewcommand\qedsymbol{$\blacksquare$}

\newcommand{\R}{\mathbb{R}}
\newcommand{\N}{\mathbb{N}}
\DeclareMathOperator{\sgn}{sgn}

\begin{document}
        \include{title_page/doc}

        \tableofcontents
        \clearpage
        
        \include{formulation_of_the_problem/doc}
        \include{research_of_the_system/doc}
        \include{algorithm/doc}
        \include{examples/doc}

        \begin{thebibliography}{9}
                \bibitem{pontryagin83} Л.~С.~Понтрягин, В.~Г.~Болтянский, Р.~В.~Гамрелидзе, Е.~Ф.~Мищенко. Математическая теория оптимальных процеccов. М.: Наука, 1983.
                \bibitem{li72} Э.~Б.~Ли, Л.~Маркус. Основы теории оптимального управления. М: Наука, 1972.
        \end{thebibliography} 
\end{document}
        \documentclass[a4paper, 11pt]{article}


\usepackage{amsmath}
\usepackage{amssymb}
\usepackage{hyperref}
\usepackage{url}
\usepackage{a4wide}
\usepackage[utf8]{inputenc}
\usepackage[main = russian, english]{babel}
\usepackage[pdftex]{graphicx}
\usepackage{float}
\usepackage{subcaption}
\usepackage{indentfirst}

% Красивый внешний вид теорем, определений и доказательств
\usepackage{amsthm}


\newenvironment{compactlist}{
        \begin{list}{{$\bullet$}}{
                        \setlength\partopsep{0pt}
                        \setlength\parskip{0pt}
                        \setlength\parsep{0pt}
                        \setlength\topsep{0pt}
                        \setlength\itemsep{0pt}
                }
        }{
        \end{list}
}
\theoremstyle{definition}
\newtheorem{definition}{Определение}

\theoremstyle{plane}
\newtheorem{theorem}{Теорема}
\newtheorem{assertion}{Утверждение}

\theoremstyle{remark}
\newtheorem{remark}{Замечание}

\renewcommand*{\proofname}{Доказательство}
\renewcommand\qedsymbol{$\blacksquare$}

\newcommand{\R}{\mathbb{R}}
\newcommand{\N}{\mathbb{N}}
\DeclareMathOperator{\sgn}{sgn}

\begin{document}
        \include{title_page/doc}

        \tableofcontents
        \clearpage
        
        \include{formulation_of_the_problem/doc}
        \include{research_of_the_system/doc}
        \include{algorithm/doc}
        \include{examples/doc}

        \begin{thebibliography}{9}
                \bibitem{pontryagin83} Л.~С.~Понтрягин, В.~Г.~Болтянский, Р.~В.~Гамрелидзе, Е.~Ф.~Мищенко. Математическая теория оптимальных процеccов. М.: Наука, 1983.
                \bibitem{li72} Э.~Б.~Ли, Л.~Маркус. Основы теории оптимального управления. М: Наука, 1972.
        \end{thebibliography} 
\end{document}
        \documentclass[a4paper, 11pt]{article}


\usepackage{amsmath}
\usepackage{amssymb}
\usepackage{hyperref}
\usepackage{url}
\usepackage{a4wide}
\usepackage[utf8]{inputenc}
\usepackage[main = russian, english]{babel}
\usepackage[pdftex]{graphicx}
\usepackage{float}
\usepackage{subcaption}
\usepackage{indentfirst}

% Красивый внешний вид теорем, определений и доказательств
\usepackage{amsthm}


\newenvironment{compactlist}{
        \begin{list}{{$\bullet$}}{
                        \setlength\partopsep{0pt}
                        \setlength\parskip{0pt}
                        \setlength\parsep{0pt}
                        \setlength\topsep{0pt}
                        \setlength\itemsep{0pt}
                }
        }{
        \end{list}
}
\theoremstyle{definition}
\newtheorem{definition}{Определение}

\theoremstyle{plane}
\newtheorem{theorem}{Теорема}
\newtheorem{assertion}{Утверждение}

\theoremstyle{remark}
\newtheorem{remark}{Замечание}

\renewcommand*{\proofname}{Доказательство}
\renewcommand\qedsymbol{$\blacksquare$}

\newcommand{\R}{\mathbb{R}}
\newcommand{\N}{\mathbb{N}}
\DeclareMathOperator{\sgn}{sgn}

\begin{document}
        \include{title_page/doc}

        \tableofcontents
        \clearpage
        
        \include{formulation_of_the_problem/doc}
        \include{research_of_the_system/doc}
        \include{algorithm/doc}
        \include{examples/doc}

        \begin{thebibliography}{9}
                \bibitem{pontryagin83} Л.~С.~Понтрягин, В.~Г.~Болтянский, Р.~В.~Гамрелидзе, Е.~Ф.~Мищенко. Математическая теория оптимальных процеccов. М.: Наука, 1983.
                \bibitem{li72} Э.~Б.~Ли, Л.~Маркус. Основы теории оптимального управления. М: Наука, 1972.
        \end{thebibliography} 
\end{document}
        \documentclass[a4paper, 11pt]{article}


\usepackage{amsmath}
\usepackage{amssymb}
\usepackage{hyperref}
\usepackage{url}
\usepackage{a4wide}
\usepackage[utf8]{inputenc}
\usepackage[main = russian, english]{babel}
\usepackage[pdftex]{graphicx}
\usepackage{float}
\usepackage{subcaption}
\usepackage{indentfirst}

% Красивый внешний вид теорем, определений и доказательств
\usepackage{amsthm}


\newenvironment{compactlist}{
        \begin{list}{{$\bullet$}}{
                        \setlength\partopsep{0pt}
                        \setlength\parskip{0pt}
                        \setlength\parsep{0pt}
                        \setlength\topsep{0pt}
                        \setlength\itemsep{0pt}
                }
        }{
        \end{list}
}
\theoremstyle{definition}
\newtheorem{definition}{Определение}

\theoremstyle{plane}
\newtheorem{theorem}{Теорема}
\newtheorem{assertion}{Утверждение}

\theoremstyle{remark}
\newtheorem{remark}{Замечание}

\renewcommand*{\proofname}{Доказательство}
\renewcommand\qedsymbol{$\blacksquare$}

\newcommand{\R}{\mathbb{R}}
\newcommand{\N}{\mathbb{N}}
\DeclareMathOperator{\sgn}{sgn}

\begin{document}
        \include{title_page/doc}

        \tableofcontents
        \clearpage
        
        \include{formulation_of_the_problem/doc}
        \include{research_of_the_system/doc}
        \include{algorithm/doc}
        \include{examples/doc}

        \begin{thebibliography}{9}
                \bibitem{pontryagin83} Л.~С.~Понтрягин, В.~Г.~Болтянский, Р.~В.~Гамрелидзе, Е.~Ф.~Мищенко. Математическая теория оптимальных процеccов. М.: Наука, 1983.
                \bibitem{li72} Э.~Б.~Ли, Л.~Маркус. Основы теории оптимального управления. М: Наука, 1972.
        \end{thebibliography} 
\end{document}

        \begin{thebibliography}{9}
                \bibitem{pontryagin83} Л.~С.~Понтрягин, В.~Г.~Болтянский, Р.~В.~Гамрелидзе, Е.~Ф.~Мищенко. Математическая теория оптимальных процеccов. М.: Наука, 1983.
                \bibitem{li72} Э.~Б.~Ли, Л.~Маркус. Основы теории оптимального управления. М: Наука, 1972.
        \end{thebibliography} 
\end{document}

        \begin{thebibliography}{9}
                \bibitem{pontryagin83} Л.~С.~Понтрягин, В.~Г.~Болтянский, Р.~В.~Гамрелидзе, Е.~Ф.~Мищенко. Математическая теория оптимальных процеccов. М.: Наука, 1983.
                \bibitem{li72} Э.~Б.~Ли, Л.~Маркус. Основы теории оптимального управления. М: Наука, 1972.
        \end{thebibliography} 
\end{document}
\documentclass[a4paper, 11pt]{article}


\usepackage{amsmath}
\usepackage{amssymb}
\usepackage{hyperref}
\usepackage{url}
\usepackage{a4wide}
\usepackage[utf8]{inputenc}
\usepackage[main = russian, english]{babel}
\usepackage[pdftex]{graphicx}
\usepackage{float}
\usepackage{subcaption}
\usepackage{indentfirst}

% Красивый внешний вид теорем, определений и доказательств
\usepackage{amsthm}


\newenvironment{compactlist}{
        \begin{list}{{$\bullet$}}{
                        \setlength\partopsep{0pt}
                        \setlength\parskip{0pt}
                        \setlength\parsep{0pt}
                        \setlength\topsep{0pt}
                        \setlength\itemsep{0pt}
                }
        }{
        \end{list}
}
\theoremstyle{definition}
\newtheorem{definition}{Определение}

\theoremstyle{plane}
\newtheorem{theorem}{Теорема}
\newtheorem{assertion}{Утверждение}

\theoremstyle{remark}
\newtheorem{remark}{Замечание}

\renewcommand*{\proofname}{Доказательство}
\renewcommand\qedsymbol{$\blacksquare$}

\newcommand{\R}{\mathbb{R}}
\newcommand{\N}{\mathbb{N}}
\DeclareMathOperator{\sgn}{sgn}

\begin{document}
        \documentclass[a4paper, 11pt]{article}


\usepackage{amsmath}
\usepackage{amssymb}
\usepackage{hyperref}
\usepackage{url}
\usepackage{a4wide}
\usepackage[utf8]{inputenc}
\usepackage[main = russian, english]{babel}
\usepackage[pdftex]{graphicx}
\usepackage{float}
\usepackage{subcaption}
\usepackage{indentfirst}

% Красивый внешний вид теорем, определений и доказательств
\usepackage{amsthm}


\newenvironment{compactlist}{
        \begin{list}{{$\bullet$}}{
                        \setlength\partopsep{0pt}
                        \setlength\parskip{0pt}
                        \setlength\parsep{0pt}
                        \setlength\topsep{0pt}
                        \setlength\itemsep{0pt}
                }
        }{
        \end{list}
}
\theoremstyle{definition}
\newtheorem{definition}{Определение}

\theoremstyle{plane}
\newtheorem{theorem}{Теорема}
\newtheorem{assertion}{Утверждение}

\theoremstyle{remark}
\newtheorem{remark}{Замечание}

\renewcommand*{\proofname}{Доказательство}
\renewcommand\qedsymbol{$\blacksquare$}

\newcommand{\R}{\mathbb{R}}
\newcommand{\N}{\mathbb{N}}
\DeclareMathOperator{\sgn}{sgn}

\begin{document}
        \documentclass[a4paper, 11pt]{article}


\usepackage{amsmath}
\usepackage{amssymb}
\usepackage{hyperref}
\usepackage{url}
\usepackage{a4wide}
\usepackage[utf8]{inputenc}
\usepackage[main = russian, english]{babel}
\usepackage[pdftex]{graphicx}
\usepackage{float}
\usepackage{subcaption}
\usepackage{indentfirst}

% Красивый внешний вид теорем, определений и доказательств
\usepackage{amsthm}


\newenvironment{compactlist}{
        \begin{list}{{$\bullet$}}{
                        \setlength\partopsep{0pt}
                        \setlength\parskip{0pt}
                        \setlength\parsep{0pt}
                        \setlength\topsep{0pt}
                        \setlength\itemsep{0pt}
                }
        }{
        \end{list}
}
\theoremstyle{definition}
\newtheorem{definition}{Определение}

\theoremstyle{plane}
\newtheorem{theorem}{Теорема}
\newtheorem{assertion}{Утверждение}

\theoremstyle{remark}
\newtheorem{remark}{Замечание}

\renewcommand*{\proofname}{Доказательство}
\renewcommand\qedsymbol{$\blacksquare$}

\newcommand{\R}{\mathbb{R}}
\newcommand{\N}{\mathbb{N}}
\DeclareMathOperator{\sgn}{sgn}

\begin{document}
        \include{title_page/doc}

        \tableofcontents
        \clearpage
        
        \include{formulation_of_the_problem/doc}
        \include{research_of_the_system/doc}
        \include{algorithm/doc}
        \include{examples/doc}

        \begin{thebibliography}{9}
                \bibitem{pontryagin83} Л.~С.~Понтрягин, В.~Г.~Болтянский, Р.~В.~Гамрелидзе, Е.~Ф.~Мищенко. Математическая теория оптимальных процеccов. М.: Наука, 1983.
                \bibitem{li72} Э.~Б.~Ли, Л.~Маркус. Основы теории оптимального управления. М: Наука, 1972.
        \end{thebibliography} 
\end{document}

        \tableofcontents
        \clearpage
        
        \documentclass[a4paper, 11pt]{article}


\usepackage{amsmath}
\usepackage{amssymb}
\usepackage{hyperref}
\usepackage{url}
\usepackage{a4wide}
\usepackage[utf8]{inputenc}
\usepackage[main = russian, english]{babel}
\usepackage[pdftex]{graphicx}
\usepackage{float}
\usepackage{subcaption}
\usepackage{indentfirst}

% Красивый внешний вид теорем, определений и доказательств
\usepackage{amsthm}


\newenvironment{compactlist}{
        \begin{list}{{$\bullet$}}{
                        \setlength\partopsep{0pt}
                        \setlength\parskip{0pt}
                        \setlength\parsep{0pt}
                        \setlength\topsep{0pt}
                        \setlength\itemsep{0pt}
                }
        }{
        \end{list}
}
\theoremstyle{definition}
\newtheorem{definition}{Определение}

\theoremstyle{plane}
\newtheorem{theorem}{Теорема}
\newtheorem{assertion}{Утверждение}

\theoremstyle{remark}
\newtheorem{remark}{Замечание}

\renewcommand*{\proofname}{Доказательство}
\renewcommand\qedsymbol{$\blacksquare$}

\newcommand{\R}{\mathbb{R}}
\newcommand{\N}{\mathbb{N}}
\DeclareMathOperator{\sgn}{sgn}

\begin{document}
        \include{title_page/doc}

        \tableofcontents
        \clearpage
        
        \include{formulation_of_the_problem/doc}
        \include{research_of_the_system/doc}
        \include{algorithm/doc}
        \include{examples/doc}

        \begin{thebibliography}{9}
                \bibitem{pontryagin83} Л.~С.~Понтрягин, В.~Г.~Болтянский, Р.~В.~Гамрелидзе, Е.~Ф.~Мищенко. Математическая теория оптимальных процеccов. М.: Наука, 1983.
                \bibitem{li72} Э.~Б.~Ли, Л.~Маркус. Основы теории оптимального управления. М: Наука, 1972.
        \end{thebibliography} 
\end{document}
        \documentclass[a4paper, 11pt]{article}


\usepackage{amsmath}
\usepackage{amssymb}
\usepackage{hyperref}
\usepackage{url}
\usepackage{a4wide}
\usepackage[utf8]{inputenc}
\usepackage[main = russian, english]{babel}
\usepackage[pdftex]{graphicx}
\usepackage{float}
\usepackage{subcaption}
\usepackage{indentfirst}

% Красивый внешний вид теорем, определений и доказательств
\usepackage{amsthm}


\newenvironment{compactlist}{
        \begin{list}{{$\bullet$}}{
                        \setlength\partopsep{0pt}
                        \setlength\parskip{0pt}
                        \setlength\parsep{0pt}
                        \setlength\topsep{0pt}
                        \setlength\itemsep{0pt}
                }
        }{
        \end{list}
}
\theoremstyle{definition}
\newtheorem{definition}{Определение}

\theoremstyle{plane}
\newtheorem{theorem}{Теорема}
\newtheorem{assertion}{Утверждение}

\theoremstyle{remark}
\newtheorem{remark}{Замечание}

\renewcommand*{\proofname}{Доказательство}
\renewcommand\qedsymbol{$\blacksquare$}

\newcommand{\R}{\mathbb{R}}
\newcommand{\N}{\mathbb{N}}
\DeclareMathOperator{\sgn}{sgn}

\begin{document}
        \include{title_page/doc}

        \tableofcontents
        \clearpage
        
        \include{formulation_of_the_problem/doc}
        \include{research_of_the_system/doc}
        \include{algorithm/doc}
        \include{examples/doc}

        \begin{thebibliography}{9}
                \bibitem{pontryagin83} Л.~С.~Понтрягин, В.~Г.~Болтянский, Р.~В.~Гамрелидзе, Е.~Ф.~Мищенко. Математическая теория оптимальных процеccов. М.: Наука, 1983.
                \bibitem{li72} Э.~Б.~Ли, Л.~Маркус. Основы теории оптимального управления. М: Наука, 1972.
        \end{thebibliography} 
\end{document}
        \documentclass[a4paper, 11pt]{article}


\usepackage{amsmath}
\usepackage{amssymb}
\usepackage{hyperref}
\usepackage{url}
\usepackage{a4wide}
\usepackage[utf8]{inputenc}
\usepackage[main = russian, english]{babel}
\usepackage[pdftex]{graphicx}
\usepackage{float}
\usepackage{subcaption}
\usepackage{indentfirst}

% Красивый внешний вид теорем, определений и доказательств
\usepackage{amsthm}


\newenvironment{compactlist}{
        \begin{list}{{$\bullet$}}{
                        \setlength\partopsep{0pt}
                        \setlength\parskip{0pt}
                        \setlength\parsep{0pt}
                        \setlength\topsep{0pt}
                        \setlength\itemsep{0pt}
                }
        }{
        \end{list}
}
\theoremstyle{definition}
\newtheorem{definition}{Определение}

\theoremstyle{plane}
\newtheorem{theorem}{Теорема}
\newtheorem{assertion}{Утверждение}

\theoremstyle{remark}
\newtheorem{remark}{Замечание}

\renewcommand*{\proofname}{Доказательство}
\renewcommand\qedsymbol{$\blacksquare$}

\newcommand{\R}{\mathbb{R}}
\newcommand{\N}{\mathbb{N}}
\DeclareMathOperator{\sgn}{sgn}

\begin{document}
        \include{title_page/doc}

        \tableofcontents
        \clearpage
        
        \include{formulation_of_the_problem/doc}
        \include{research_of_the_system/doc}
        \include{algorithm/doc}
        \include{examples/doc}

        \begin{thebibliography}{9}
                \bibitem{pontryagin83} Л.~С.~Понтрягин, В.~Г.~Болтянский, Р.~В.~Гамрелидзе, Е.~Ф.~Мищенко. Математическая теория оптимальных процеccов. М.: Наука, 1983.
                \bibitem{li72} Э.~Б.~Ли, Л.~Маркус. Основы теории оптимального управления. М: Наука, 1972.
        \end{thebibliography} 
\end{document}
        \documentclass[a4paper, 11pt]{article}


\usepackage{amsmath}
\usepackage{amssymb}
\usepackage{hyperref}
\usepackage{url}
\usepackage{a4wide}
\usepackage[utf8]{inputenc}
\usepackage[main = russian, english]{babel}
\usepackage[pdftex]{graphicx}
\usepackage{float}
\usepackage{subcaption}
\usepackage{indentfirst}

% Красивый внешний вид теорем, определений и доказательств
\usepackage{amsthm}


\newenvironment{compactlist}{
        \begin{list}{{$\bullet$}}{
                        \setlength\partopsep{0pt}
                        \setlength\parskip{0pt}
                        \setlength\parsep{0pt}
                        \setlength\topsep{0pt}
                        \setlength\itemsep{0pt}
                }
        }{
        \end{list}
}
\theoremstyle{definition}
\newtheorem{definition}{Определение}

\theoremstyle{plane}
\newtheorem{theorem}{Теорема}
\newtheorem{assertion}{Утверждение}

\theoremstyle{remark}
\newtheorem{remark}{Замечание}

\renewcommand*{\proofname}{Доказательство}
\renewcommand\qedsymbol{$\blacksquare$}

\newcommand{\R}{\mathbb{R}}
\newcommand{\N}{\mathbb{N}}
\DeclareMathOperator{\sgn}{sgn}

\begin{document}
        \include{title_page/doc}

        \tableofcontents
        \clearpage
        
        \include{formulation_of_the_problem/doc}
        \include{research_of_the_system/doc}
        \include{algorithm/doc}
        \include{examples/doc}

        \begin{thebibliography}{9}
                \bibitem{pontryagin83} Л.~С.~Понтрягин, В.~Г.~Болтянский, Р.~В.~Гамрелидзе, Е.~Ф.~Мищенко. Математическая теория оптимальных процеccов. М.: Наука, 1983.
                \bibitem{li72} Э.~Б.~Ли, Л.~Маркус. Основы теории оптимального управления. М: Наука, 1972.
        \end{thebibliography} 
\end{document}

        \begin{thebibliography}{9}
                \bibitem{pontryagin83} Л.~С.~Понтрягин, В.~Г.~Болтянский, Р.~В.~Гамрелидзе, Е.~Ф.~Мищенко. Математическая теория оптимальных процеccов. М.: Наука, 1983.
                \bibitem{li72} Э.~Б.~Ли, Л.~Маркус. Основы теории оптимального управления. М: Наука, 1972.
        \end{thebibliography} 
\end{document}

        \tableofcontents
        \clearpage
        
        \documentclass[a4paper, 11pt]{article}


\usepackage{amsmath}
\usepackage{amssymb}
\usepackage{hyperref}
\usepackage{url}
\usepackage{a4wide}
\usepackage[utf8]{inputenc}
\usepackage[main = russian, english]{babel}
\usepackage[pdftex]{graphicx}
\usepackage{float}
\usepackage{subcaption}
\usepackage{indentfirst}

% Красивый внешний вид теорем, определений и доказательств
\usepackage{amsthm}


\newenvironment{compactlist}{
        \begin{list}{{$\bullet$}}{
                        \setlength\partopsep{0pt}
                        \setlength\parskip{0pt}
                        \setlength\parsep{0pt}
                        \setlength\topsep{0pt}
                        \setlength\itemsep{0pt}
                }
        }{
        \end{list}
}
\theoremstyle{definition}
\newtheorem{definition}{Определение}

\theoremstyle{plane}
\newtheorem{theorem}{Теорема}
\newtheorem{assertion}{Утверждение}

\theoremstyle{remark}
\newtheorem{remark}{Замечание}

\renewcommand*{\proofname}{Доказательство}
\renewcommand\qedsymbol{$\blacksquare$}

\newcommand{\R}{\mathbb{R}}
\newcommand{\N}{\mathbb{N}}
\DeclareMathOperator{\sgn}{sgn}

\begin{document}
        \documentclass[a4paper, 11pt]{article}


\usepackage{amsmath}
\usepackage{amssymb}
\usepackage{hyperref}
\usepackage{url}
\usepackage{a4wide}
\usepackage[utf8]{inputenc}
\usepackage[main = russian, english]{babel}
\usepackage[pdftex]{graphicx}
\usepackage{float}
\usepackage{subcaption}
\usepackage{indentfirst}

% Красивый внешний вид теорем, определений и доказательств
\usepackage{amsthm}


\newenvironment{compactlist}{
        \begin{list}{{$\bullet$}}{
                        \setlength\partopsep{0pt}
                        \setlength\parskip{0pt}
                        \setlength\parsep{0pt}
                        \setlength\topsep{0pt}
                        \setlength\itemsep{0pt}
                }
        }{
        \end{list}
}
\theoremstyle{definition}
\newtheorem{definition}{Определение}

\theoremstyle{plane}
\newtheorem{theorem}{Теорема}
\newtheorem{assertion}{Утверждение}

\theoremstyle{remark}
\newtheorem{remark}{Замечание}

\renewcommand*{\proofname}{Доказательство}
\renewcommand\qedsymbol{$\blacksquare$}

\newcommand{\R}{\mathbb{R}}
\newcommand{\N}{\mathbb{N}}
\DeclareMathOperator{\sgn}{sgn}

\begin{document}
        \include{title_page/doc}

        \tableofcontents
        \clearpage
        
        \include{formulation_of_the_problem/doc}
        \include{research_of_the_system/doc}
        \include{algorithm/doc}
        \include{examples/doc}

        \begin{thebibliography}{9}
                \bibitem{pontryagin83} Л.~С.~Понтрягин, В.~Г.~Болтянский, Р.~В.~Гамрелидзе, Е.~Ф.~Мищенко. Математическая теория оптимальных процеccов. М.: Наука, 1983.
                \bibitem{li72} Э.~Б.~Ли, Л.~Маркус. Основы теории оптимального управления. М: Наука, 1972.
        \end{thebibliography} 
\end{document}

        \tableofcontents
        \clearpage
        
        \documentclass[a4paper, 11pt]{article}


\usepackage{amsmath}
\usepackage{amssymb}
\usepackage{hyperref}
\usepackage{url}
\usepackage{a4wide}
\usepackage[utf8]{inputenc}
\usepackage[main = russian, english]{babel}
\usepackage[pdftex]{graphicx}
\usepackage{float}
\usepackage{subcaption}
\usepackage{indentfirst}

% Красивый внешний вид теорем, определений и доказательств
\usepackage{amsthm}


\newenvironment{compactlist}{
        \begin{list}{{$\bullet$}}{
                        \setlength\partopsep{0pt}
                        \setlength\parskip{0pt}
                        \setlength\parsep{0pt}
                        \setlength\topsep{0pt}
                        \setlength\itemsep{0pt}
                }
        }{
        \end{list}
}
\theoremstyle{definition}
\newtheorem{definition}{Определение}

\theoremstyle{plane}
\newtheorem{theorem}{Теорема}
\newtheorem{assertion}{Утверждение}

\theoremstyle{remark}
\newtheorem{remark}{Замечание}

\renewcommand*{\proofname}{Доказательство}
\renewcommand\qedsymbol{$\blacksquare$}

\newcommand{\R}{\mathbb{R}}
\newcommand{\N}{\mathbb{N}}
\DeclareMathOperator{\sgn}{sgn}

\begin{document}
        \include{title_page/doc}

        \tableofcontents
        \clearpage
        
        \include{formulation_of_the_problem/doc}
        \include{research_of_the_system/doc}
        \include{algorithm/doc}
        \include{examples/doc}

        \begin{thebibliography}{9}
                \bibitem{pontryagin83} Л.~С.~Понтрягин, В.~Г.~Болтянский, Р.~В.~Гамрелидзе, Е.~Ф.~Мищенко. Математическая теория оптимальных процеccов. М.: Наука, 1983.
                \bibitem{li72} Э.~Б.~Ли, Л.~Маркус. Основы теории оптимального управления. М: Наука, 1972.
        \end{thebibliography} 
\end{document}
        \documentclass[a4paper, 11pt]{article}


\usepackage{amsmath}
\usepackage{amssymb}
\usepackage{hyperref}
\usepackage{url}
\usepackage{a4wide}
\usepackage[utf8]{inputenc}
\usepackage[main = russian, english]{babel}
\usepackage[pdftex]{graphicx}
\usepackage{float}
\usepackage{subcaption}
\usepackage{indentfirst}

% Красивый внешний вид теорем, определений и доказательств
\usepackage{amsthm}


\newenvironment{compactlist}{
        \begin{list}{{$\bullet$}}{
                        \setlength\partopsep{0pt}
                        \setlength\parskip{0pt}
                        \setlength\parsep{0pt}
                        \setlength\topsep{0pt}
                        \setlength\itemsep{0pt}
                }
        }{
        \end{list}
}
\theoremstyle{definition}
\newtheorem{definition}{Определение}

\theoremstyle{plane}
\newtheorem{theorem}{Теорема}
\newtheorem{assertion}{Утверждение}

\theoremstyle{remark}
\newtheorem{remark}{Замечание}

\renewcommand*{\proofname}{Доказательство}
\renewcommand\qedsymbol{$\blacksquare$}

\newcommand{\R}{\mathbb{R}}
\newcommand{\N}{\mathbb{N}}
\DeclareMathOperator{\sgn}{sgn}

\begin{document}
        \include{title_page/doc}

        \tableofcontents
        \clearpage
        
        \include{formulation_of_the_problem/doc}
        \include{research_of_the_system/doc}
        \include{algorithm/doc}
        \include{examples/doc}

        \begin{thebibliography}{9}
                \bibitem{pontryagin83} Л.~С.~Понтрягин, В.~Г.~Болтянский, Р.~В.~Гамрелидзе, Е.~Ф.~Мищенко. Математическая теория оптимальных процеccов. М.: Наука, 1983.
                \bibitem{li72} Э.~Б.~Ли, Л.~Маркус. Основы теории оптимального управления. М: Наука, 1972.
        \end{thebibliography} 
\end{document}
        \documentclass[a4paper, 11pt]{article}


\usepackage{amsmath}
\usepackage{amssymb}
\usepackage{hyperref}
\usepackage{url}
\usepackage{a4wide}
\usepackage[utf8]{inputenc}
\usepackage[main = russian, english]{babel}
\usepackage[pdftex]{graphicx}
\usepackage{float}
\usepackage{subcaption}
\usepackage{indentfirst}

% Красивый внешний вид теорем, определений и доказательств
\usepackage{amsthm}


\newenvironment{compactlist}{
        \begin{list}{{$\bullet$}}{
                        \setlength\partopsep{0pt}
                        \setlength\parskip{0pt}
                        \setlength\parsep{0pt}
                        \setlength\topsep{0pt}
                        \setlength\itemsep{0pt}
                }
        }{
        \end{list}
}
\theoremstyle{definition}
\newtheorem{definition}{Определение}

\theoremstyle{plane}
\newtheorem{theorem}{Теорема}
\newtheorem{assertion}{Утверждение}

\theoremstyle{remark}
\newtheorem{remark}{Замечание}

\renewcommand*{\proofname}{Доказательство}
\renewcommand\qedsymbol{$\blacksquare$}

\newcommand{\R}{\mathbb{R}}
\newcommand{\N}{\mathbb{N}}
\DeclareMathOperator{\sgn}{sgn}

\begin{document}
        \include{title_page/doc}

        \tableofcontents
        \clearpage
        
        \include{formulation_of_the_problem/doc}
        \include{research_of_the_system/doc}
        \include{algorithm/doc}
        \include{examples/doc}

        \begin{thebibliography}{9}
                \bibitem{pontryagin83} Л.~С.~Понтрягин, В.~Г.~Болтянский, Р.~В.~Гамрелидзе, Е.~Ф.~Мищенко. Математическая теория оптимальных процеccов. М.: Наука, 1983.
                \bibitem{li72} Э.~Б.~Ли, Л.~Маркус. Основы теории оптимального управления. М: Наука, 1972.
        \end{thebibliography} 
\end{document}
        \documentclass[a4paper, 11pt]{article}


\usepackage{amsmath}
\usepackage{amssymb}
\usepackage{hyperref}
\usepackage{url}
\usepackage{a4wide}
\usepackage[utf8]{inputenc}
\usepackage[main = russian, english]{babel}
\usepackage[pdftex]{graphicx}
\usepackage{float}
\usepackage{subcaption}
\usepackage{indentfirst}

% Красивый внешний вид теорем, определений и доказательств
\usepackage{amsthm}


\newenvironment{compactlist}{
        \begin{list}{{$\bullet$}}{
                        \setlength\partopsep{0pt}
                        \setlength\parskip{0pt}
                        \setlength\parsep{0pt}
                        \setlength\topsep{0pt}
                        \setlength\itemsep{0pt}
                }
        }{
        \end{list}
}
\theoremstyle{definition}
\newtheorem{definition}{Определение}

\theoremstyle{plane}
\newtheorem{theorem}{Теорема}
\newtheorem{assertion}{Утверждение}

\theoremstyle{remark}
\newtheorem{remark}{Замечание}

\renewcommand*{\proofname}{Доказательство}
\renewcommand\qedsymbol{$\blacksquare$}

\newcommand{\R}{\mathbb{R}}
\newcommand{\N}{\mathbb{N}}
\DeclareMathOperator{\sgn}{sgn}

\begin{document}
        \include{title_page/doc}

        \tableofcontents
        \clearpage
        
        \include{formulation_of_the_problem/doc}
        \include{research_of_the_system/doc}
        \include{algorithm/doc}
        \include{examples/doc}

        \begin{thebibliography}{9}
                \bibitem{pontryagin83} Л.~С.~Понтрягин, В.~Г.~Болтянский, Р.~В.~Гамрелидзе, Е.~Ф.~Мищенко. Математическая теория оптимальных процеccов. М.: Наука, 1983.
                \bibitem{li72} Э.~Б.~Ли, Л.~Маркус. Основы теории оптимального управления. М: Наука, 1972.
        \end{thebibliography} 
\end{document}

        \begin{thebibliography}{9}
                \bibitem{pontryagin83} Л.~С.~Понтрягин, В.~Г.~Болтянский, Р.~В.~Гамрелидзе, Е.~Ф.~Мищенко. Математическая теория оптимальных процеccов. М.: Наука, 1983.
                \bibitem{li72} Э.~Б.~Ли, Л.~Маркус. Основы теории оптимального управления. М: Наука, 1972.
        \end{thebibliography} 
\end{document}
        \documentclass[a4paper, 11pt]{article}


\usepackage{amsmath}
\usepackage{amssymb}
\usepackage{hyperref}
\usepackage{url}
\usepackage{a4wide}
\usepackage[utf8]{inputenc}
\usepackage[main = russian, english]{babel}
\usepackage[pdftex]{graphicx}
\usepackage{float}
\usepackage{subcaption}
\usepackage{indentfirst}

% Красивый внешний вид теорем, определений и доказательств
\usepackage{amsthm}


\newenvironment{compactlist}{
        \begin{list}{{$\bullet$}}{
                        \setlength\partopsep{0pt}
                        \setlength\parskip{0pt}
                        \setlength\parsep{0pt}
                        \setlength\topsep{0pt}
                        \setlength\itemsep{0pt}
                }
        }{
        \end{list}
}
\theoremstyle{definition}
\newtheorem{definition}{Определение}

\theoremstyle{plane}
\newtheorem{theorem}{Теорема}
\newtheorem{assertion}{Утверждение}

\theoremstyle{remark}
\newtheorem{remark}{Замечание}

\renewcommand*{\proofname}{Доказательство}
\renewcommand\qedsymbol{$\blacksquare$}

\newcommand{\R}{\mathbb{R}}
\newcommand{\N}{\mathbb{N}}
\DeclareMathOperator{\sgn}{sgn}

\begin{document}
        \documentclass[a4paper, 11pt]{article}


\usepackage{amsmath}
\usepackage{amssymb}
\usepackage{hyperref}
\usepackage{url}
\usepackage{a4wide}
\usepackage[utf8]{inputenc}
\usepackage[main = russian, english]{babel}
\usepackage[pdftex]{graphicx}
\usepackage{float}
\usepackage{subcaption}
\usepackage{indentfirst}

% Красивый внешний вид теорем, определений и доказательств
\usepackage{amsthm}


\newenvironment{compactlist}{
        \begin{list}{{$\bullet$}}{
                        \setlength\partopsep{0pt}
                        \setlength\parskip{0pt}
                        \setlength\parsep{0pt}
                        \setlength\topsep{0pt}
                        \setlength\itemsep{0pt}
                }
        }{
        \end{list}
}
\theoremstyle{definition}
\newtheorem{definition}{Определение}

\theoremstyle{plane}
\newtheorem{theorem}{Теорема}
\newtheorem{assertion}{Утверждение}

\theoremstyle{remark}
\newtheorem{remark}{Замечание}

\renewcommand*{\proofname}{Доказательство}
\renewcommand\qedsymbol{$\blacksquare$}

\newcommand{\R}{\mathbb{R}}
\newcommand{\N}{\mathbb{N}}
\DeclareMathOperator{\sgn}{sgn}

\begin{document}
        \include{title_page/doc}

        \tableofcontents
        \clearpage
        
        \include{formulation_of_the_problem/doc}
        \include{research_of_the_system/doc}
        \include{algorithm/doc}
        \include{examples/doc}

        \begin{thebibliography}{9}
                \bibitem{pontryagin83} Л.~С.~Понтрягин, В.~Г.~Болтянский, Р.~В.~Гамрелидзе, Е.~Ф.~Мищенко. Математическая теория оптимальных процеccов. М.: Наука, 1983.
                \bibitem{li72} Э.~Б.~Ли, Л.~Маркус. Основы теории оптимального управления. М: Наука, 1972.
        \end{thebibliography} 
\end{document}

        \tableofcontents
        \clearpage
        
        \documentclass[a4paper, 11pt]{article}


\usepackage{amsmath}
\usepackage{amssymb}
\usepackage{hyperref}
\usepackage{url}
\usepackage{a4wide}
\usepackage[utf8]{inputenc}
\usepackage[main = russian, english]{babel}
\usepackage[pdftex]{graphicx}
\usepackage{float}
\usepackage{subcaption}
\usepackage{indentfirst}

% Красивый внешний вид теорем, определений и доказательств
\usepackage{amsthm}


\newenvironment{compactlist}{
        \begin{list}{{$\bullet$}}{
                        \setlength\partopsep{0pt}
                        \setlength\parskip{0pt}
                        \setlength\parsep{0pt}
                        \setlength\topsep{0pt}
                        \setlength\itemsep{0pt}
                }
        }{
        \end{list}
}
\theoremstyle{definition}
\newtheorem{definition}{Определение}

\theoremstyle{plane}
\newtheorem{theorem}{Теорема}
\newtheorem{assertion}{Утверждение}

\theoremstyle{remark}
\newtheorem{remark}{Замечание}

\renewcommand*{\proofname}{Доказательство}
\renewcommand\qedsymbol{$\blacksquare$}

\newcommand{\R}{\mathbb{R}}
\newcommand{\N}{\mathbb{N}}
\DeclareMathOperator{\sgn}{sgn}

\begin{document}
        \include{title_page/doc}

        \tableofcontents
        \clearpage
        
        \include{formulation_of_the_problem/doc}
        \include{research_of_the_system/doc}
        \include{algorithm/doc}
        \include{examples/doc}

        \begin{thebibliography}{9}
                \bibitem{pontryagin83} Л.~С.~Понтрягин, В.~Г.~Болтянский, Р.~В.~Гамрелидзе, Е.~Ф.~Мищенко. Математическая теория оптимальных процеccов. М.: Наука, 1983.
                \bibitem{li72} Э.~Б.~Ли, Л.~Маркус. Основы теории оптимального управления. М: Наука, 1972.
        \end{thebibliography} 
\end{document}
        \documentclass[a4paper, 11pt]{article}


\usepackage{amsmath}
\usepackage{amssymb}
\usepackage{hyperref}
\usepackage{url}
\usepackage{a4wide}
\usepackage[utf8]{inputenc}
\usepackage[main = russian, english]{babel}
\usepackage[pdftex]{graphicx}
\usepackage{float}
\usepackage{subcaption}
\usepackage{indentfirst}

% Красивый внешний вид теорем, определений и доказательств
\usepackage{amsthm}


\newenvironment{compactlist}{
        \begin{list}{{$\bullet$}}{
                        \setlength\partopsep{0pt}
                        \setlength\parskip{0pt}
                        \setlength\parsep{0pt}
                        \setlength\topsep{0pt}
                        \setlength\itemsep{0pt}
                }
        }{
        \end{list}
}
\theoremstyle{definition}
\newtheorem{definition}{Определение}

\theoremstyle{plane}
\newtheorem{theorem}{Теорема}
\newtheorem{assertion}{Утверждение}

\theoremstyle{remark}
\newtheorem{remark}{Замечание}

\renewcommand*{\proofname}{Доказательство}
\renewcommand\qedsymbol{$\blacksquare$}

\newcommand{\R}{\mathbb{R}}
\newcommand{\N}{\mathbb{N}}
\DeclareMathOperator{\sgn}{sgn}

\begin{document}
        \include{title_page/doc}

        \tableofcontents
        \clearpage
        
        \include{formulation_of_the_problem/doc}
        \include{research_of_the_system/doc}
        \include{algorithm/doc}
        \include{examples/doc}

        \begin{thebibliography}{9}
                \bibitem{pontryagin83} Л.~С.~Понтрягин, В.~Г.~Болтянский, Р.~В.~Гамрелидзе, Е.~Ф.~Мищенко. Математическая теория оптимальных процеccов. М.: Наука, 1983.
                \bibitem{li72} Э.~Б.~Ли, Л.~Маркус. Основы теории оптимального управления. М: Наука, 1972.
        \end{thebibliography} 
\end{document}
        \documentclass[a4paper, 11pt]{article}


\usepackage{amsmath}
\usepackage{amssymb}
\usepackage{hyperref}
\usepackage{url}
\usepackage{a4wide}
\usepackage[utf8]{inputenc}
\usepackage[main = russian, english]{babel}
\usepackage[pdftex]{graphicx}
\usepackage{float}
\usepackage{subcaption}
\usepackage{indentfirst}

% Красивый внешний вид теорем, определений и доказательств
\usepackage{amsthm}


\newenvironment{compactlist}{
        \begin{list}{{$\bullet$}}{
                        \setlength\partopsep{0pt}
                        \setlength\parskip{0pt}
                        \setlength\parsep{0pt}
                        \setlength\topsep{0pt}
                        \setlength\itemsep{0pt}
                }
        }{
        \end{list}
}
\theoremstyle{definition}
\newtheorem{definition}{Определение}

\theoremstyle{plane}
\newtheorem{theorem}{Теорема}
\newtheorem{assertion}{Утверждение}

\theoremstyle{remark}
\newtheorem{remark}{Замечание}

\renewcommand*{\proofname}{Доказательство}
\renewcommand\qedsymbol{$\blacksquare$}

\newcommand{\R}{\mathbb{R}}
\newcommand{\N}{\mathbb{N}}
\DeclareMathOperator{\sgn}{sgn}

\begin{document}
        \include{title_page/doc}

        \tableofcontents
        \clearpage
        
        \include{formulation_of_the_problem/doc}
        \include{research_of_the_system/doc}
        \include{algorithm/doc}
        \include{examples/doc}

        \begin{thebibliography}{9}
                \bibitem{pontryagin83} Л.~С.~Понтрягин, В.~Г.~Болтянский, Р.~В.~Гамрелидзе, Е.~Ф.~Мищенко. Математическая теория оптимальных процеccов. М.: Наука, 1983.
                \bibitem{li72} Э.~Б.~Ли, Л.~Маркус. Основы теории оптимального управления. М: Наука, 1972.
        \end{thebibliography} 
\end{document}
        \documentclass[a4paper, 11pt]{article}


\usepackage{amsmath}
\usepackage{amssymb}
\usepackage{hyperref}
\usepackage{url}
\usepackage{a4wide}
\usepackage[utf8]{inputenc}
\usepackage[main = russian, english]{babel}
\usepackage[pdftex]{graphicx}
\usepackage{float}
\usepackage{subcaption}
\usepackage{indentfirst}

% Красивый внешний вид теорем, определений и доказательств
\usepackage{amsthm}


\newenvironment{compactlist}{
        \begin{list}{{$\bullet$}}{
                        \setlength\partopsep{0pt}
                        \setlength\parskip{0pt}
                        \setlength\parsep{0pt}
                        \setlength\topsep{0pt}
                        \setlength\itemsep{0pt}
                }
        }{
        \end{list}
}
\theoremstyle{definition}
\newtheorem{definition}{Определение}

\theoremstyle{plane}
\newtheorem{theorem}{Теорема}
\newtheorem{assertion}{Утверждение}

\theoremstyle{remark}
\newtheorem{remark}{Замечание}

\renewcommand*{\proofname}{Доказательство}
\renewcommand\qedsymbol{$\blacksquare$}

\newcommand{\R}{\mathbb{R}}
\newcommand{\N}{\mathbb{N}}
\DeclareMathOperator{\sgn}{sgn}

\begin{document}
        \include{title_page/doc}

        \tableofcontents
        \clearpage
        
        \include{formulation_of_the_problem/doc}
        \include{research_of_the_system/doc}
        \include{algorithm/doc}
        \include{examples/doc}

        \begin{thebibliography}{9}
                \bibitem{pontryagin83} Л.~С.~Понтрягин, В.~Г.~Болтянский, Р.~В.~Гамрелидзе, Е.~Ф.~Мищенко. Математическая теория оптимальных процеccов. М.: Наука, 1983.
                \bibitem{li72} Э.~Б.~Ли, Л.~Маркус. Основы теории оптимального управления. М: Наука, 1972.
        \end{thebibliography} 
\end{document}

        \begin{thebibliography}{9}
                \bibitem{pontryagin83} Л.~С.~Понтрягин, В.~Г.~Болтянский, Р.~В.~Гамрелидзе, Е.~Ф.~Мищенко. Математическая теория оптимальных процеccов. М.: Наука, 1983.
                \bibitem{li72} Э.~Б.~Ли, Л.~Маркус. Основы теории оптимального управления. М: Наука, 1972.
        \end{thebibliography} 
\end{document}
        \documentclass[a4paper, 11pt]{article}


\usepackage{amsmath}
\usepackage{amssymb}
\usepackage{hyperref}
\usepackage{url}
\usepackage{a4wide}
\usepackage[utf8]{inputenc}
\usepackage[main = russian, english]{babel}
\usepackage[pdftex]{graphicx}
\usepackage{float}
\usepackage{subcaption}
\usepackage{indentfirst}

% Красивый внешний вид теорем, определений и доказательств
\usepackage{amsthm}


\newenvironment{compactlist}{
        \begin{list}{{$\bullet$}}{
                        \setlength\partopsep{0pt}
                        \setlength\parskip{0pt}
                        \setlength\parsep{0pt}
                        \setlength\topsep{0pt}
                        \setlength\itemsep{0pt}
                }
        }{
        \end{list}
}
\theoremstyle{definition}
\newtheorem{definition}{Определение}

\theoremstyle{plane}
\newtheorem{theorem}{Теорема}
\newtheorem{assertion}{Утверждение}

\theoremstyle{remark}
\newtheorem{remark}{Замечание}

\renewcommand*{\proofname}{Доказательство}
\renewcommand\qedsymbol{$\blacksquare$}

\newcommand{\R}{\mathbb{R}}
\newcommand{\N}{\mathbb{N}}
\DeclareMathOperator{\sgn}{sgn}

\begin{document}
        \documentclass[a4paper, 11pt]{article}


\usepackage{amsmath}
\usepackage{amssymb}
\usepackage{hyperref}
\usepackage{url}
\usepackage{a4wide}
\usepackage[utf8]{inputenc}
\usepackage[main = russian, english]{babel}
\usepackage[pdftex]{graphicx}
\usepackage{float}
\usepackage{subcaption}
\usepackage{indentfirst}

% Красивый внешний вид теорем, определений и доказательств
\usepackage{amsthm}


\newenvironment{compactlist}{
        \begin{list}{{$\bullet$}}{
                        \setlength\partopsep{0pt}
                        \setlength\parskip{0pt}
                        \setlength\parsep{0pt}
                        \setlength\topsep{0pt}
                        \setlength\itemsep{0pt}
                }
        }{
        \end{list}
}
\theoremstyle{definition}
\newtheorem{definition}{Определение}

\theoremstyle{plane}
\newtheorem{theorem}{Теорема}
\newtheorem{assertion}{Утверждение}

\theoremstyle{remark}
\newtheorem{remark}{Замечание}

\renewcommand*{\proofname}{Доказательство}
\renewcommand\qedsymbol{$\blacksquare$}

\newcommand{\R}{\mathbb{R}}
\newcommand{\N}{\mathbb{N}}
\DeclareMathOperator{\sgn}{sgn}

\begin{document}
        \include{title_page/doc}

        \tableofcontents
        \clearpage
        
        \include{formulation_of_the_problem/doc}
        \include{research_of_the_system/doc}
        \include{algorithm/doc}
        \include{examples/doc}

        \begin{thebibliography}{9}
                \bibitem{pontryagin83} Л.~С.~Понтрягин, В.~Г.~Болтянский, Р.~В.~Гамрелидзе, Е.~Ф.~Мищенко. Математическая теория оптимальных процеccов. М.: Наука, 1983.
                \bibitem{li72} Э.~Б.~Ли, Л.~Маркус. Основы теории оптимального управления. М: Наука, 1972.
        \end{thebibliography} 
\end{document}

        \tableofcontents
        \clearpage
        
        \documentclass[a4paper, 11pt]{article}


\usepackage{amsmath}
\usepackage{amssymb}
\usepackage{hyperref}
\usepackage{url}
\usepackage{a4wide}
\usepackage[utf8]{inputenc}
\usepackage[main = russian, english]{babel}
\usepackage[pdftex]{graphicx}
\usepackage{float}
\usepackage{subcaption}
\usepackage{indentfirst}

% Красивый внешний вид теорем, определений и доказательств
\usepackage{amsthm}


\newenvironment{compactlist}{
        \begin{list}{{$\bullet$}}{
                        \setlength\partopsep{0pt}
                        \setlength\parskip{0pt}
                        \setlength\parsep{0pt}
                        \setlength\topsep{0pt}
                        \setlength\itemsep{0pt}
                }
        }{
        \end{list}
}
\theoremstyle{definition}
\newtheorem{definition}{Определение}

\theoremstyle{plane}
\newtheorem{theorem}{Теорема}
\newtheorem{assertion}{Утверждение}

\theoremstyle{remark}
\newtheorem{remark}{Замечание}

\renewcommand*{\proofname}{Доказательство}
\renewcommand\qedsymbol{$\blacksquare$}

\newcommand{\R}{\mathbb{R}}
\newcommand{\N}{\mathbb{N}}
\DeclareMathOperator{\sgn}{sgn}

\begin{document}
        \include{title_page/doc}

        \tableofcontents
        \clearpage
        
        \include{formulation_of_the_problem/doc}
        \include{research_of_the_system/doc}
        \include{algorithm/doc}
        \include{examples/doc}

        \begin{thebibliography}{9}
                \bibitem{pontryagin83} Л.~С.~Понтрягин, В.~Г.~Болтянский, Р.~В.~Гамрелидзе, Е.~Ф.~Мищенко. Математическая теория оптимальных процеccов. М.: Наука, 1983.
                \bibitem{li72} Э.~Б.~Ли, Л.~Маркус. Основы теории оптимального управления. М: Наука, 1972.
        \end{thebibliography} 
\end{document}
        \documentclass[a4paper, 11pt]{article}


\usepackage{amsmath}
\usepackage{amssymb}
\usepackage{hyperref}
\usepackage{url}
\usepackage{a4wide}
\usepackage[utf8]{inputenc}
\usepackage[main = russian, english]{babel}
\usepackage[pdftex]{graphicx}
\usepackage{float}
\usepackage{subcaption}
\usepackage{indentfirst}

% Красивый внешний вид теорем, определений и доказательств
\usepackage{amsthm}


\newenvironment{compactlist}{
        \begin{list}{{$\bullet$}}{
                        \setlength\partopsep{0pt}
                        \setlength\parskip{0pt}
                        \setlength\parsep{0pt}
                        \setlength\topsep{0pt}
                        \setlength\itemsep{0pt}
                }
        }{
        \end{list}
}
\theoremstyle{definition}
\newtheorem{definition}{Определение}

\theoremstyle{plane}
\newtheorem{theorem}{Теорема}
\newtheorem{assertion}{Утверждение}

\theoremstyle{remark}
\newtheorem{remark}{Замечание}

\renewcommand*{\proofname}{Доказательство}
\renewcommand\qedsymbol{$\blacksquare$}

\newcommand{\R}{\mathbb{R}}
\newcommand{\N}{\mathbb{N}}
\DeclareMathOperator{\sgn}{sgn}

\begin{document}
        \include{title_page/doc}

        \tableofcontents
        \clearpage
        
        \include{formulation_of_the_problem/doc}
        \include{research_of_the_system/doc}
        \include{algorithm/doc}
        \include{examples/doc}

        \begin{thebibliography}{9}
                \bibitem{pontryagin83} Л.~С.~Понтрягин, В.~Г.~Болтянский, Р.~В.~Гамрелидзе, Е.~Ф.~Мищенко. Математическая теория оптимальных процеccов. М.: Наука, 1983.
                \bibitem{li72} Э.~Б.~Ли, Л.~Маркус. Основы теории оптимального управления. М: Наука, 1972.
        \end{thebibliography} 
\end{document}
        \documentclass[a4paper, 11pt]{article}


\usepackage{amsmath}
\usepackage{amssymb}
\usepackage{hyperref}
\usepackage{url}
\usepackage{a4wide}
\usepackage[utf8]{inputenc}
\usepackage[main = russian, english]{babel}
\usepackage[pdftex]{graphicx}
\usepackage{float}
\usepackage{subcaption}
\usepackage{indentfirst}

% Красивый внешний вид теорем, определений и доказательств
\usepackage{amsthm}


\newenvironment{compactlist}{
        \begin{list}{{$\bullet$}}{
                        \setlength\partopsep{0pt}
                        \setlength\parskip{0pt}
                        \setlength\parsep{0pt}
                        \setlength\topsep{0pt}
                        \setlength\itemsep{0pt}
                }
        }{
        \end{list}
}
\theoremstyle{definition}
\newtheorem{definition}{Определение}

\theoremstyle{plane}
\newtheorem{theorem}{Теорема}
\newtheorem{assertion}{Утверждение}

\theoremstyle{remark}
\newtheorem{remark}{Замечание}

\renewcommand*{\proofname}{Доказательство}
\renewcommand\qedsymbol{$\blacksquare$}

\newcommand{\R}{\mathbb{R}}
\newcommand{\N}{\mathbb{N}}
\DeclareMathOperator{\sgn}{sgn}

\begin{document}
        \include{title_page/doc}

        \tableofcontents
        \clearpage
        
        \include{formulation_of_the_problem/doc}
        \include{research_of_the_system/doc}
        \include{algorithm/doc}
        \include{examples/doc}

        \begin{thebibliography}{9}
                \bibitem{pontryagin83} Л.~С.~Понтрягин, В.~Г.~Болтянский, Р.~В.~Гамрелидзе, Е.~Ф.~Мищенко. Математическая теория оптимальных процеccов. М.: Наука, 1983.
                \bibitem{li72} Э.~Б.~Ли, Л.~Маркус. Основы теории оптимального управления. М: Наука, 1972.
        \end{thebibliography} 
\end{document}
        \documentclass[a4paper, 11pt]{article}


\usepackage{amsmath}
\usepackage{amssymb}
\usepackage{hyperref}
\usepackage{url}
\usepackage{a4wide}
\usepackage[utf8]{inputenc}
\usepackage[main = russian, english]{babel}
\usepackage[pdftex]{graphicx}
\usepackage{float}
\usepackage{subcaption}
\usepackage{indentfirst}

% Красивый внешний вид теорем, определений и доказательств
\usepackage{amsthm}


\newenvironment{compactlist}{
        \begin{list}{{$\bullet$}}{
                        \setlength\partopsep{0pt}
                        \setlength\parskip{0pt}
                        \setlength\parsep{0pt}
                        \setlength\topsep{0pt}
                        \setlength\itemsep{0pt}
                }
        }{
        \end{list}
}
\theoremstyle{definition}
\newtheorem{definition}{Определение}

\theoremstyle{plane}
\newtheorem{theorem}{Теорема}
\newtheorem{assertion}{Утверждение}

\theoremstyle{remark}
\newtheorem{remark}{Замечание}

\renewcommand*{\proofname}{Доказательство}
\renewcommand\qedsymbol{$\blacksquare$}

\newcommand{\R}{\mathbb{R}}
\newcommand{\N}{\mathbb{N}}
\DeclareMathOperator{\sgn}{sgn}

\begin{document}
        \include{title_page/doc}

        \tableofcontents
        \clearpage
        
        \include{formulation_of_the_problem/doc}
        \include{research_of_the_system/doc}
        \include{algorithm/doc}
        \include{examples/doc}

        \begin{thebibliography}{9}
                \bibitem{pontryagin83} Л.~С.~Понтрягин, В.~Г.~Болтянский, Р.~В.~Гамрелидзе, Е.~Ф.~Мищенко. Математическая теория оптимальных процеccов. М.: Наука, 1983.
                \bibitem{li72} Э.~Б.~Ли, Л.~Маркус. Основы теории оптимального управления. М: Наука, 1972.
        \end{thebibliography} 
\end{document}

        \begin{thebibliography}{9}
                \bibitem{pontryagin83} Л.~С.~Понтрягин, В.~Г.~Болтянский, Р.~В.~Гамрелидзе, Е.~Ф.~Мищенко. Математическая теория оптимальных процеccов. М.: Наука, 1983.
                \bibitem{li72} Э.~Б.~Ли, Л.~Маркус. Основы теории оптимального управления. М: Наука, 1972.
        \end{thebibliography} 
\end{document}
        \documentclass[a4paper, 11pt]{article}


\usepackage{amsmath}
\usepackage{amssymb}
\usepackage{hyperref}
\usepackage{url}
\usepackage{a4wide}
\usepackage[utf8]{inputenc}
\usepackage[main = russian, english]{babel}
\usepackage[pdftex]{graphicx}
\usepackage{float}
\usepackage{subcaption}
\usepackage{indentfirst}

% Красивый внешний вид теорем, определений и доказательств
\usepackage{amsthm}


\newenvironment{compactlist}{
        \begin{list}{{$\bullet$}}{
                        \setlength\partopsep{0pt}
                        \setlength\parskip{0pt}
                        \setlength\parsep{0pt}
                        \setlength\topsep{0pt}
                        \setlength\itemsep{0pt}
                }
        }{
        \end{list}
}
\theoremstyle{definition}
\newtheorem{definition}{Определение}

\theoremstyle{plane}
\newtheorem{theorem}{Теорема}
\newtheorem{assertion}{Утверждение}

\theoremstyle{remark}
\newtheorem{remark}{Замечание}

\renewcommand*{\proofname}{Доказательство}
\renewcommand\qedsymbol{$\blacksquare$}

\newcommand{\R}{\mathbb{R}}
\newcommand{\N}{\mathbb{N}}
\DeclareMathOperator{\sgn}{sgn}

\begin{document}
        \documentclass[a4paper, 11pt]{article}


\usepackage{amsmath}
\usepackage{amssymb}
\usepackage{hyperref}
\usepackage{url}
\usepackage{a4wide}
\usepackage[utf8]{inputenc}
\usepackage[main = russian, english]{babel}
\usepackage[pdftex]{graphicx}
\usepackage{float}
\usepackage{subcaption}
\usepackage{indentfirst}

% Красивый внешний вид теорем, определений и доказательств
\usepackage{amsthm}


\newenvironment{compactlist}{
        \begin{list}{{$\bullet$}}{
                        \setlength\partopsep{0pt}
                        \setlength\parskip{0pt}
                        \setlength\parsep{0pt}
                        \setlength\topsep{0pt}
                        \setlength\itemsep{0pt}
                }
        }{
        \end{list}
}
\theoremstyle{definition}
\newtheorem{definition}{Определение}

\theoremstyle{plane}
\newtheorem{theorem}{Теорема}
\newtheorem{assertion}{Утверждение}

\theoremstyle{remark}
\newtheorem{remark}{Замечание}

\renewcommand*{\proofname}{Доказательство}
\renewcommand\qedsymbol{$\blacksquare$}

\newcommand{\R}{\mathbb{R}}
\newcommand{\N}{\mathbb{N}}
\DeclareMathOperator{\sgn}{sgn}

\begin{document}
        \include{title_page/doc}

        \tableofcontents
        \clearpage
        
        \include{formulation_of_the_problem/doc}
        \include{research_of_the_system/doc}
        \include{algorithm/doc}
        \include{examples/doc}

        \begin{thebibliography}{9}
                \bibitem{pontryagin83} Л.~С.~Понтрягин, В.~Г.~Болтянский, Р.~В.~Гамрелидзе, Е.~Ф.~Мищенко. Математическая теория оптимальных процеccов. М.: Наука, 1983.
                \bibitem{li72} Э.~Б.~Ли, Л.~Маркус. Основы теории оптимального управления. М: Наука, 1972.
        \end{thebibliography} 
\end{document}

        \tableofcontents
        \clearpage
        
        \documentclass[a4paper, 11pt]{article}


\usepackage{amsmath}
\usepackage{amssymb}
\usepackage{hyperref}
\usepackage{url}
\usepackage{a4wide}
\usepackage[utf8]{inputenc}
\usepackage[main = russian, english]{babel}
\usepackage[pdftex]{graphicx}
\usepackage{float}
\usepackage{subcaption}
\usepackage{indentfirst}

% Красивый внешний вид теорем, определений и доказательств
\usepackage{amsthm}


\newenvironment{compactlist}{
        \begin{list}{{$\bullet$}}{
                        \setlength\partopsep{0pt}
                        \setlength\parskip{0pt}
                        \setlength\parsep{0pt}
                        \setlength\topsep{0pt}
                        \setlength\itemsep{0pt}
                }
        }{
        \end{list}
}
\theoremstyle{definition}
\newtheorem{definition}{Определение}

\theoremstyle{plane}
\newtheorem{theorem}{Теорема}
\newtheorem{assertion}{Утверждение}

\theoremstyle{remark}
\newtheorem{remark}{Замечание}

\renewcommand*{\proofname}{Доказательство}
\renewcommand\qedsymbol{$\blacksquare$}

\newcommand{\R}{\mathbb{R}}
\newcommand{\N}{\mathbb{N}}
\DeclareMathOperator{\sgn}{sgn}

\begin{document}
        \include{title_page/doc}

        \tableofcontents
        \clearpage
        
        \include{formulation_of_the_problem/doc}
        \include{research_of_the_system/doc}
        \include{algorithm/doc}
        \include{examples/doc}

        \begin{thebibliography}{9}
                \bibitem{pontryagin83} Л.~С.~Понтрягин, В.~Г.~Болтянский, Р.~В.~Гамрелидзе, Е.~Ф.~Мищенко. Математическая теория оптимальных процеccов. М.: Наука, 1983.
                \bibitem{li72} Э.~Б.~Ли, Л.~Маркус. Основы теории оптимального управления. М: Наука, 1972.
        \end{thebibliography} 
\end{document}
        \documentclass[a4paper, 11pt]{article}


\usepackage{amsmath}
\usepackage{amssymb}
\usepackage{hyperref}
\usepackage{url}
\usepackage{a4wide}
\usepackage[utf8]{inputenc}
\usepackage[main = russian, english]{babel}
\usepackage[pdftex]{graphicx}
\usepackage{float}
\usepackage{subcaption}
\usepackage{indentfirst}

% Красивый внешний вид теорем, определений и доказательств
\usepackage{amsthm}


\newenvironment{compactlist}{
        \begin{list}{{$\bullet$}}{
                        \setlength\partopsep{0pt}
                        \setlength\parskip{0pt}
                        \setlength\parsep{0pt}
                        \setlength\topsep{0pt}
                        \setlength\itemsep{0pt}
                }
        }{
        \end{list}
}
\theoremstyle{definition}
\newtheorem{definition}{Определение}

\theoremstyle{plane}
\newtheorem{theorem}{Теорема}
\newtheorem{assertion}{Утверждение}

\theoremstyle{remark}
\newtheorem{remark}{Замечание}

\renewcommand*{\proofname}{Доказательство}
\renewcommand\qedsymbol{$\blacksquare$}

\newcommand{\R}{\mathbb{R}}
\newcommand{\N}{\mathbb{N}}
\DeclareMathOperator{\sgn}{sgn}

\begin{document}
        \include{title_page/doc}

        \tableofcontents
        \clearpage
        
        \include{formulation_of_the_problem/doc}
        \include{research_of_the_system/doc}
        \include{algorithm/doc}
        \include{examples/doc}

        \begin{thebibliography}{9}
                \bibitem{pontryagin83} Л.~С.~Понтрягин, В.~Г.~Болтянский, Р.~В.~Гамрелидзе, Е.~Ф.~Мищенко. Математическая теория оптимальных процеccов. М.: Наука, 1983.
                \bibitem{li72} Э.~Б.~Ли, Л.~Маркус. Основы теории оптимального управления. М: Наука, 1972.
        \end{thebibliography} 
\end{document}
        \documentclass[a4paper, 11pt]{article}


\usepackage{amsmath}
\usepackage{amssymb}
\usepackage{hyperref}
\usepackage{url}
\usepackage{a4wide}
\usepackage[utf8]{inputenc}
\usepackage[main = russian, english]{babel}
\usepackage[pdftex]{graphicx}
\usepackage{float}
\usepackage{subcaption}
\usepackage{indentfirst}

% Красивый внешний вид теорем, определений и доказательств
\usepackage{amsthm}


\newenvironment{compactlist}{
        \begin{list}{{$\bullet$}}{
                        \setlength\partopsep{0pt}
                        \setlength\parskip{0pt}
                        \setlength\parsep{0pt}
                        \setlength\topsep{0pt}
                        \setlength\itemsep{0pt}
                }
        }{
        \end{list}
}
\theoremstyle{definition}
\newtheorem{definition}{Определение}

\theoremstyle{plane}
\newtheorem{theorem}{Теорема}
\newtheorem{assertion}{Утверждение}

\theoremstyle{remark}
\newtheorem{remark}{Замечание}

\renewcommand*{\proofname}{Доказательство}
\renewcommand\qedsymbol{$\blacksquare$}

\newcommand{\R}{\mathbb{R}}
\newcommand{\N}{\mathbb{N}}
\DeclareMathOperator{\sgn}{sgn}

\begin{document}
        \include{title_page/doc}

        \tableofcontents
        \clearpage
        
        \include{formulation_of_the_problem/doc}
        \include{research_of_the_system/doc}
        \include{algorithm/doc}
        \include{examples/doc}

        \begin{thebibliography}{9}
                \bibitem{pontryagin83} Л.~С.~Понтрягин, В.~Г.~Болтянский, Р.~В.~Гамрелидзе, Е.~Ф.~Мищенко. Математическая теория оптимальных процеccов. М.: Наука, 1983.
                \bibitem{li72} Э.~Б.~Ли, Л.~Маркус. Основы теории оптимального управления. М: Наука, 1972.
        \end{thebibliography} 
\end{document}
        \documentclass[a4paper, 11pt]{article}


\usepackage{amsmath}
\usepackage{amssymb}
\usepackage{hyperref}
\usepackage{url}
\usepackage{a4wide}
\usepackage[utf8]{inputenc}
\usepackage[main = russian, english]{babel}
\usepackage[pdftex]{graphicx}
\usepackage{float}
\usepackage{subcaption}
\usepackage{indentfirst}

% Красивый внешний вид теорем, определений и доказательств
\usepackage{amsthm}


\newenvironment{compactlist}{
        \begin{list}{{$\bullet$}}{
                        \setlength\partopsep{0pt}
                        \setlength\parskip{0pt}
                        \setlength\parsep{0pt}
                        \setlength\topsep{0pt}
                        \setlength\itemsep{0pt}
                }
        }{
        \end{list}
}
\theoremstyle{definition}
\newtheorem{definition}{Определение}

\theoremstyle{plane}
\newtheorem{theorem}{Теорема}
\newtheorem{assertion}{Утверждение}

\theoremstyle{remark}
\newtheorem{remark}{Замечание}

\renewcommand*{\proofname}{Доказательство}
\renewcommand\qedsymbol{$\blacksquare$}

\newcommand{\R}{\mathbb{R}}
\newcommand{\N}{\mathbb{N}}
\DeclareMathOperator{\sgn}{sgn}

\begin{document}
        \include{title_page/doc}

        \tableofcontents
        \clearpage
        
        \include{formulation_of_the_problem/doc}
        \include{research_of_the_system/doc}
        \include{algorithm/doc}
        \include{examples/doc}

        \begin{thebibliography}{9}
                \bibitem{pontryagin83} Л.~С.~Понтрягин, В.~Г.~Болтянский, Р.~В.~Гамрелидзе, Е.~Ф.~Мищенко. Математическая теория оптимальных процеccов. М.: Наука, 1983.
                \bibitem{li72} Э.~Б.~Ли, Л.~Маркус. Основы теории оптимального управления. М: Наука, 1972.
        \end{thebibliography} 
\end{document}

        \begin{thebibliography}{9}
                \bibitem{pontryagin83} Л.~С.~Понтрягин, В.~Г.~Болтянский, Р.~В.~Гамрелидзе, Е.~Ф.~Мищенко. Математическая теория оптимальных процеccов. М.: Наука, 1983.
                \bibitem{li72} Э.~Б.~Ли, Л.~Маркус. Основы теории оптимального управления. М: Наука, 1972.
        \end{thebibliography} 
\end{document}

        \begin{thebibliography}{9}
                \bibitem{pontryagin83} Л.~С.~Понтрягин, В.~Г.~Болтянский, Р.~В.~Гамрелидзе, Е.~Ф.~Мищенко. Математическая теория оптимальных процеccов. М.: Наука, 1983.
                \bibitem{li72} Э.~Б.~Ли, Л.~Маркус. Основы теории оптимального управления. М: Наука, 1972.
        \end{thebibliography} 
\end{document}