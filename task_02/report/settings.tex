\documentclass[a4paper, 11pt]{article}

% Внешние пакеты
\usepackage{amsmath}
\usepackage{amssymb}
\usepackage{hyperref}
\usepackage{url}
\usepackage{a4wide}
\usepackage[utf8]{inputenc}
\usepackage[main = russian, english]{babel}
\usepackage[pdftex]{graphicx}
\usepackage{float}
\usepackage{subcaption}
\usepackage{indentfirst}
\usepackage{amsthm}                  % Красивый внешний вид теорем, определений и доказательств
% \usepackage[integrals]{wasysym}    % Делает интегралы прямыми, но некрасивыми
\usepackage{bigints}                 % Позволяет делать большущие интегралы

% Красивый внешний вид теорем, определений и доказательств
\theoremstyle{definition}
\newtheorem{definition}{Определение}

\theoremstyle{plane}
\newtheorem{theorem}{Теорема}
\newtheorem{lemma}{Лемма}
\newtheorem{assertion}{Утверждение}

\theoremstyle{remark}
\newtheorem{remark}{Замечание}

\renewcommand*{\proofname}{Доказательство}
\renewcommand\qedsymbol{$\blacksquare$}

% Переопределение математических штук
\newcommand{\R}{\mathbb{R}}
\newcommand{\N}{\mathbb{N}}
\newcommand{\T}{\mathrm{T}}
\DeclareMathOperator{\sgn}{sgn}
\DeclareMathOperator{\const}{const}
