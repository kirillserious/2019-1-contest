\section{Решение задачи при втором типе ограничения на управление}

В этом разделе рассматриваются ограничения на значения управляющих параметров вида
$$
        u(t) \in \Omega = \{\,[u_1,\,u_2]^\T\;:\; u_1 \in \R,\; u_2\in[k_1,\,k_2]\,\}.
$$
Функция Гамильтона"--~Понтрягина принимает вид:
$$
        \psi_0(u_1^2 + \alpha|u_1|) + \psi_1 x_2 + \psi_2(u_1 - x_2(1 + u_2)), \mbox{ где $\psi_0 < 0$}.
$$
Задача ее максимизации равносильна
$$
        -\psi_0(u_1^2 + \alpha|u_1|) + \psi_2(u_1 - x_2(1+u_2)) \to \max\limits_{u}.
$$
Аналогично первому случаю можно считать, что $\psi_0 = \frac{1}{2}$. Тогда задача максимизации распадается на две:
$$
\begin{aligned}
        u_1^2 + \alpha|u_1| + \psi_2 u_1 \to \max\limits_{u_1\in\R}, \\
        -\psi_2 x_2 u_2 \to \max\limits_{u_2\in[k_1,\,k_2]}.
\end{aligned}
$$
Эти задачи имеют следующие решения:
$$
\begin{aligned}
        u_1^* &= 
        \begin{cases}
                \psi_2 + \frac{\alpha}{2}, &\mbox{при $\psi_2 < -\frac{\alpha}{2}$}\\
                0, &\mbox{при $-\frac{\alpha}{2} \leqslant \psi_2 \leqslant \frac{\alpha}{2}$}\\
                \psi_2 - \frac{\alpha}{2}, &\mbox{при $\psi_2 > \frac{\alpha}{2}$},
        \end{cases}
        \\
        u_2^* &=
        \begin{cases}
                k_2, &\mbox{при $\psi_2 x_2 < 0$}\\
                [k_1,\,k_2], &\mbox{при $\psi_2 x_2 = 0$}\\
                k_1, &\mbox{при $\psi_2 x_2 > 0$}.
        \end{cases}
\end{aligned}
$$

Теорема \ref{th:first} требует, чтобы
$\psi_2^0 > \frac{\alpha}{2}$. Аналогично в некоторой окрестности нуля систему описывают соотношения (\ref{eq:firstlim_psi2}),(\ref{firstlim_x2}),(\ref{firstlim_x1}). Все так же нас будут интересовать моменты $t_1$ и $t_2$, в которые
$\psi_2(t_1) = \frac{\alpha}{2}$
и
$\psi_2(t_2) = 0$,
а также момент времени $t_3$ такой, что
$\psi_2(t_3) = -\frac{\alpha}{2}$.
Введем классификацию рассматриваемых управлений.
\begin{enumerate}
        \item Пусть $A \geqslant 0$. Тогда будем говорить, что управление реализует \textit{режим акселерации}.
        \item Пусть $A < 0$.
        \begin{enumerate}
                \item Если $t_1 > T$, то будем говорить, что управление реализует \textit{режим отсутствия торможения}.
                \item Если $t_1 < T, \; t_2 > T$, то будем говорить, что управление реализует \textit{режим слабого торможения}.
                \item Если $t_2 < T,\; t_3 > T$, то будем говорить, что управление реализует \textit{режим сильного торможения}.
                \item Если $t_3 < T$, то будем говорить, что управление реализует \textit{режим интенсивного торможения}.
        \end{enumerate}
\end{enumerate}

При этом видно, что режимы акселерации, отсутствия и слабого торможения дословно переносятся на случай второго типа ограничений на управления. В случае же сильного торможения необходимо добавить единственную проверку.

\documentclass[a4paper, 11pt]{article}


\usepackage{amsmath}
\usepackage{amssymb}
\usepackage{hyperref}
\usepackage{url}
\usepackage{a4wide}
\usepackage[utf8]{inputenc}
\usepackage[main = russian, english]{babel}
\usepackage[pdftex]{graphicx}
\usepackage{float}
\usepackage{subcaption}
\usepackage{indentfirst}

% Красивый внешний вид теорем, определений и доказательств
\usepackage{amsthm}


\newenvironment{compactlist}{
        \begin{list}{{$\bullet$}}{
                        \setlength\partopsep{0pt}
                        \setlength\parskip{0pt}
                        \setlength\parsep{0pt}
                        \setlength\topsep{0pt}
                        \setlength\itemsep{0pt}
                }
        }{
        \end{list}
}
\theoremstyle{definition}
\newtheorem{definition}{Определение}

\theoremstyle{plane}
\newtheorem{theorem}{Теорема}
\newtheorem{assertion}{Утверждение}

\theoremstyle{remark}
\newtheorem{remark}{Замечание}

\renewcommand*{\proofname}{Доказательство}
\renewcommand\qedsymbol{$\blacksquare$}

\newcommand{\R}{\mathbb{R}}
\newcommand{\N}{\mathbb{N}}
\DeclareMathOperator{\sgn}{sgn}

\begin{document}
        \documentclass[a4paper, 11pt]{article}


\usepackage{amsmath}
\usepackage{amssymb}
\usepackage{hyperref}
\usepackage{url}
\usepackage{a4wide}
\usepackage[utf8]{inputenc}
\usepackage[main = russian, english]{babel}
\usepackage[pdftex]{graphicx}
\usepackage{float}
\usepackage{subcaption}
\usepackage{indentfirst}

% Красивый внешний вид теорем, определений и доказательств
\usepackage{amsthm}


\newenvironment{compactlist}{
        \begin{list}{{$\bullet$}}{
                        \setlength\partopsep{0pt}
                        \setlength\parskip{0pt}
                        \setlength\parsep{0pt}
                        \setlength\topsep{0pt}
                        \setlength\itemsep{0pt}
                }
        }{
        \end{list}
}
\theoremstyle{definition}
\newtheorem{definition}{Определение}

\theoremstyle{plane}
\newtheorem{theorem}{Теорема}
\newtheorem{assertion}{Утверждение}

\theoremstyle{remark}
\newtheorem{remark}{Замечание}

\renewcommand*{\proofname}{Доказательство}
\renewcommand\qedsymbol{$\blacksquare$}

\newcommand{\R}{\mathbb{R}}
\newcommand{\N}{\mathbb{N}}
\DeclareMathOperator{\sgn}{sgn}

\begin{document}
        \documentclass[a4paper, 11pt]{article}


\usepackage{amsmath}
\usepackage{amssymb}
\usepackage{hyperref}
\usepackage{url}
\usepackage{a4wide}
\usepackage[utf8]{inputenc}
\usepackage[main = russian, english]{babel}
\usepackage[pdftex]{graphicx}
\usepackage{float}
\usepackage{subcaption}
\usepackage{indentfirst}

% Красивый внешний вид теорем, определений и доказательств
\usepackage{amsthm}


\newenvironment{compactlist}{
        \begin{list}{{$\bullet$}}{
                        \setlength\partopsep{0pt}
                        \setlength\parskip{0pt}
                        \setlength\parsep{0pt}
                        \setlength\topsep{0pt}
                        \setlength\itemsep{0pt}
                }
        }{
        \end{list}
}
\theoremstyle{definition}
\newtheorem{definition}{Определение}

\theoremstyle{plane}
\newtheorem{theorem}{Теорема}
\newtheorem{assertion}{Утверждение}

\theoremstyle{remark}
\newtheorem{remark}{Замечание}

\renewcommand*{\proofname}{Доказательство}
\renewcommand\qedsymbol{$\blacksquare$}

\newcommand{\R}{\mathbb{R}}
\newcommand{\N}{\mathbb{N}}
\DeclareMathOperator{\sgn}{sgn}

\begin{document}
        \include{title_page/doc}

        \tableofcontents
        \clearpage
        
        \include{formulation_of_the_problem/doc}
        \include{research_of_the_system/doc}
        \include{algorithm/doc}
        \include{examples/doc}

        \begin{thebibliography}{9}
                \bibitem{pontryagin83} Л.~С.~Понтрягин, В.~Г.~Болтянский, Р.~В.~Гамрелидзе, Е.~Ф.~Мищенко. Математическая теория оптимальных процеccов. М.: Наука, 1983.
                \bibitem{li72} Э.~Б.~Ли, Л.~Маркус. Основы теории оптимального управления. М: Наука, 1972.
        \end{thebibliography} 
\end{document}

        \tableofcontents
        \clearpage
        
        \documentclass[a4paper, 11pt]{article}


\usepackage{amsmath}
\usepackage{amssymb}
\usepackage{hyperref}
\usepackage{url}
\usepackage{a4wide}
\usepackage[utf8]{inputenc}
\usepackage[main = russian, english]{babel}
\usepackage[pdftex]{graphicx}
\usepackage{float}
\usepackage{subcaption}
\usepackage{indentfirst}

% Красивый внешний вид теорем, определений и доказательств
\usepackage{amsthm}


\newenvironment{compactlist}{
        \begin{list}{{$\bullet$}}{
                        \setlength\partopsep{0pt}
                        \setlength\parskip{0pt}
                        \setlength\parsep{0pt}
                        \setlength\topsep{0pt}
                        \setlength\itemsep{0pt}
                }
        }{
        \end{list}
}
\theoremstyle{definition}
\newtheorem{definition}{Определение}

\theoremstyle{plane}
\newtheorem{theorem}{Теорема}
\newtheorem{assertion}{Утверждение}

\theoremstyle{remark}
\newtheorem{remark}{Замечание}

\renewcommand*{\proofname}{Доказательство}
\renewcommand\qedsymbol{$\blacksquare$}

\newcommand{\R}{\mathbb{R}}
\newcommand{\N}{\mathbb{N}}
\DeclareMathOperator{\sgn}{sgn}

\begin{document}
        \include{title_page/doc}

        \tableofcontents
        \clearpage
        
        \include{formulation_of_the_problem/doc}
        \include{research_of_the_system/doc}
        \include{algorithm/doc}
        \include{examples/doc}

        \begin{thebibliography}{9}
                \bibitem{pontryagin83} Л.~С.~Понтрягин, В.~Г.~Болтянский, Р.~В.~Гамрелидзе, Е.~Ф.~Мищенко. Математическая теория оптимальных процеccов. М.: Наука, 1983.
                \bibitem{li72} Э.~Б.~Ли, Л.~Маркус. Основы теории оптимального управления. М: Наука, 1972.
        \end{thebibliography} 
\end{document}
        \documentclass[a4paper, 11pt]{article}


\usepackage{amsmath}
\usepackage{amssymb}
\usepackage{hyperref}
\usepackage{url}
\usepackage{a4wide}
\usepackage[utf8]{inputenc}
\usepackage[main = russian, english]{babel}
\usepackage[pdftex]{graphicx}
\usepackage{float}
\usepackage{subcaption}
\usepackage{indentfirst}

% Красивый внешний вид теорем, определений и доказательств
\usepackage{amsthm}


\newenvironment{compactlist}{
        \begin{list}{{$\bullet$}}{
                        \setlength\partopsep{0pt}
                        \setlength\parskip{0pt}
                        \setlength\parsep{0pt}
                        \setlength\topsep{0pt}
                        \setlength\itemsep{0pt}
                }
        }{
        \end{list}
}
\theoremstyle{definition}
\newtheorem{definition}{Определение}

\theoremstyle{plane}
\newtheorem{theorem}{Теорема}
\newtheorem{assertion}{Утверждение}

\theoremstyle{remark}
\newtheorem{remark}{Замечание}

\renewcommand*{\proofname}{Доказательство}
\renewcommand\qedsymbol{$\blacksquare$}

\newcommand{\R}{\mathbb{R}}
\newcommand{\N}{\mathbb{N}}
\DeclareMathOperator{\sgn}{sgn}

\begin{document}
        \include{title_page/doc}

        \tableofcontents
        \clearpage
        
        \include{formulation_of_the_problem/doc}
        \include{research_of_the_system/doc}
        \include{algorithm/doc}
        \include{examples/doc}

        \begin{thebibliography}{9}
                \bibitem{pontryagin83} Л.~С.~Понтрягин, В.~Г.~Болтянский, Р.~В.~Гамрелидзе, Е.~Ф.~Мищенко. Математическая теория оптимальных процеccов. М.: Наука, 1983.
                \bibitem{li72} Э.~Б.~Ли, Л.~Маркус. Основы теории оптимального управления. М: Наука, 1972.
        \end{thebibliography} 
\end{document}
        \documentclass[a4paper, 11pt]{article}


\usepackage{amsmath}
\usepackage{amssymb}
\usepackage{hyperref}
\usepackage{url}
\usepackage{a4wide}
\usepackage[utf8]{inputenc}
\usepackage[main = russian, english]{babel}
\usepackage[pdftex]{graphicx}
\usepackage{float}
\usepackage{subcaption}
\usepackage{indentfirst}

% Красивый внешний вид теорем, определений и доказательств
\usepackage{amsthm}


\newenvironment{compactlist}{
        \begin{list}{{$\bullet$}}{
                        \setlength\partopsep{0pt}
                        \setlength\parskip{0pt}
                        \setlength\parsep{0pt}
                        \setlength\topsep{0pt}
                        \setlength\itemsep{0pt}
                }
        }{
        \end{list}
}
\theoremstyle{definition}
\newtheorem{definition}{Определение}

\theoremstyle{plane}
\newtheorem{theorem}{Теорема}
\newtheorem{assertion}{Утверждение}

\theoremstyle{remark}
\newtheorem{remark}{Замечание}

\renewcommand*{\proofname}{Доказательство}
\renewcommand\qedsymbol{$\blacksquare$}

\newcommand{\R}{\mathbb{R}}
\newcommand{\N}{\mathbb{N}}
\DeclareMathOperator{\sgn}{sgn}

\begin{document}
        \include{title_page/doc}

        \tableofcontents
        \clearpage
        
        \include{formulation_of_the_problem/doc}
        \include{research_of_the_system/doc}
        \include{algorithm/doc}
        \include{examples/doc}

        \begin{thebibliography}{9}
                \bibitem{pontryagin83} Л.~С.~Понтрягин, В.~Г.~Болтянский, Р.~В.~Гамрелидзе, Е.~Ф.~Мищенко. Математическая теория оптимальных процеccов. М.: Наука, 1983.
                \bibitem{li72} Э.~Б.~Ли, Л.~Маркус. Основы теории оптимального управления. М: Наука, 1972.
        \end{thebibliography} 
\end{document}
        \documentclass[a4paper, 11pt]{article}


\usepackage{amsmath}
\usepackage{amssymb}
\usepackage{hyperref}
\usepackage{url}
\usepackage{a4wide}
\usepackage[utf8]{inputenc}
\usepackage[main = russian, english]{babel}
\usepackage[pdftex]{graphicx}
\usepackage{float}
\usepackage{subcaption}
\usepackage{indentfirst}

% Красивый внешний вид теорем, определений и доказательств
\usepackage{amsthm}


\newenvironment{compactlist}{
        \begin{list}{{$\bullet$}}{
                        \setlength\partopsep{0pt}
                        \setlength\parskip{0pt}
                        \setlength\parsep{0pt}
                        \setlength\topsep{0pt}
                        \setlength\itemsep{0pt}
                }
        }{
        \end{list}
}
\theoremstyle{definition}
\newtheorem{definition}{Определение}

\theoremstyle{plane}
\newtheorem{theorem}{Теорема}
\newtheorem{assertion}{Утверждение}

\theoremstyle{remark}
\newtheorem{remark}{Замечание}

\renewcommand*{\proofname}{Доказательство}
\renewcommand\qedsymbol{$\blacksquare$}

\newcommand{\R}{\mathbb{R}}
\newcommand{\N}{\mathbb{N}}
\DeclareMathOperator{\sgn}{sgn}

\begin{document}
        \include{title_page/doc}

        \tableofcontents
        \clearpage
        
        \include{formulation_of_the_problem/doc}
        \include{research_of_the_system/doc}
        \include{algorithm/doc}
        \include{examples/doc}

        \begin{thebibliography}{9}
                \bibitem{pontryagin83} Л.~С.~Понтрягин, В.~Г.~Болтянский, Р.~В.~Гамрелидзе, Е.~Ф.~Мищенко. Математическая теория оптимальных процеccов. М.: Наука, 1983.
                \bibitem{li72} Э.~Б.~Ли, Л.~Маркус. Основы теории оптимального управления. М: Наука, 1972.
        \end{thebibliography} 
\end{document}

        \begin{thebibliography}{9}
                \bibitem{pontryagin83} Л.~С.~Понтрягин, В.~Г.~Болтянский, Р.~В.~Гамрелидзе, Е.~Ф.~Мищенко. Математическая теория оптимальных процеccов. М.: Наука, 1983.
                \bibitem{li72} Э.~Б.~Ли, Л.~Маркус. Основы теории оптимального управления. М: Наука, 1972.
        \end{thebibliography} 
\end{document}

        \tableofcontents
        \clearpage
        
        \documentclass[a4paper, 11pt]{article}


\usepackage{amsmath}
\usepackage{amssymb}
\usepackage{hyperref}
\usepackage{url}
\usepackage{a4wide}
\usepackage[utf8]{inputenc}
\usepackage[main = russian, english]{babel}
\usepackage[pdftex]{graphicx}
\usepackage{float}
\usepackage{subcaption}
\usepackage{indentfirst}

% Красивый внешний вид теорем, определений и доказательств
\usepackage{amsthm}


\newenvironment{compactlist}{
        \begin{list}{{$\bullet$}}{
                        \setlength\partopsep{0pt}
                        \setlength\parskip{0pt}
                        \setlength\parsep{0pt}
                        \setlength\topsep{0pt}
                        \setlength\itemsep{0pt}
                }
        }{
        \end{list}
}
\theoremstyle{definition}
\newtheorem{definition}{Определение}

\theoremstyle{plane}
\newtheorem{theorem}{Теорема}
\newtheorem{assertion}{Утверждение}

\theoremstyle{remark}
\newtheorem{remark}{Замечание}

\renewcommand*{\proofname}{Доказательство}
\renewcommand\qedsymbol{$\blacksquare$}

\newcommand{\R}{\mathbb{R}}
\newcommand{\N}{\mathbb{N}}
\DeclareMathOperator{\sgn}{sgn}

\begin{document}
        \documentclass[a4paper, 11pt]{article}


\usepackage{amsmath}
\usepackage{amssymb}
\usepackage{hyperref}
\usepackage{url}
\usepackage{a4wide}
\usepackage[utf8]{inputenc}
\usepackage[main = russian, english]{babel}
\usepackage[pdftex]{graphicx}
\usepackage{float}
\usepackage{subcaption}
\usepackage{indentfirst}

% Красивый внешний вид теорем, определений и доказательств
\usepackage{amsthm}


\newenvironment{compactlist}{
        \begin{list}{{$\bullet$}}{
                        \setlength\partopsep{0pt}
                        \setlength\parskip{0pt}
                        \setlength\parsep{0pt}
                        \setlength\topsep{0pt}
                        \setlength\itemsep{0pt}
                }
        }{
        \end{list}
}
\theoremstyle{definition}
\newtheorem{definition}{Определение}

\theoremstyle{plane}
\newtheorem{theorem}{Теорема}
\newtheorem{assertion}{Утверждение}

\theoremstyle{remark}
\newtheorem{remark}{Замечание}

\renewcommand*{\proofname}{Доказательство}
\renewcommand\qedsymbol{$\blacksquare$}

\newcommand{\R}{\mathbb{R}}
\newcommand{\N}{\mathbb{N}}
\DeclareMathOperator{\sgn}{sgn}

\begin{document}
        \include{title_page/doc}

        \tableofcontents
        \clearpage
        
        \include{formulation_of_the_problem/doc}
        \include{research_of_the_system/doc}
        \include{algorithm/doc}
        \include{examples/doc}

        \begin{thebibliography}{9}
                \bibitem{pontryagin83} Л.~С.~Понтрягин, В.~Г.~Болтянский, Р.~В.~Гамрелидзе, Е.~Ф.~Мищенко. Математическая теория оптимальных процеccов. М.: Наука, 1983.
                \bibitem{li72} Э.~Б.~Ли, Л.~Маркус. Основы теории оптимального управления. М: Наука, 1972.
        \end{thebibliography} 
\end{document}

        \tableofcontents
        \clearpage
        
        \documentclass[a4paper, 11pt]{article}


\usepackage{amsmath}
\usepackage{amssymb}
\usepackage{hyperref}
\usepackage{url}
\usepackage{a4wide}
\usepackage[utf8]{inputenc}
\usepackage[main = russian, english]{babel}
\usepackage[pdftex]{graphicx}
\usepackage{float}
\usepackage{subcaption}
\usepackage{indentfirst}

% Красивый внешний вид теорем, определений и доказательств
\usepackage{amsthm}


\newenvironment{compactlist}{
        \begin{list}{{$\bullet$}}{
                        \setlength\partopsep{0pt}
                        \setlength\parskip{0pt}
                        \setlength\parsep{0pt}
                        \setlength\topsep{0pt}
                        \setlength\itemsep{0pt}
                }
        }{
        \end{list}
}
\theoremstyle{definition}
\newtheorem{definition}{Определение}

\theoremstyle{plane}
\newtheorem{theorem}{Теорема}
\newtheorem{assertion}{Утверждение}

\theoremstyle{remark}
\newtheorem{remark}{Замечание}

\renewcommand*{\proofname}{Доказательство}
\renewcommand\qedsymbol{$\blacksquare$}

\newcommand{\R}{\mathbb{R}}
\newcommand{\N}{\mathbb{N}}
\DeclareMathOperator{\sgn}{sgn}

\begin{document}
        \include{title_page/doc}

        \tableofcontents
        \clearpage
        
        \include{formulation_of_the_problem/doc}
        \include{research_of_the_system/doc}
        \include{algorithm/doc}
        \include{examples/doc}

        \begin{thebibliography}{9}
                \bibitem{pontryagin83} Л.~С.~Понтрягин, В.~Г.~Болтянский, Р.~В.~Гамрелидзе, Е.~Ф.~Мищенко. Математическая теория оптимальных процеccов. М.: Наука, 1983.
                \bibitem{li72} Э.~Б.~Ли, Л.~Маркус. Основы теории оптимального управления. М: Наука, 1972.
        \end{thebibliography} 
\end{document}
        \documentclass[a4paper, 11pt]{article}


\usepackage{amsmath}
\usepackage{amssymb}
\usepackage{hyperref}
\usepackage{url}
\usepackage{a4wide}
\usepackage[utf8]{inputenc}
\usepackage[main = russian, english]{babel}
\usepackage[pdftex]{graphicx}
\usepackage{float}
\usepackage{subcaption}
\usepackage{indentfirst}

% Красивый внешний вид теорем, определений и доказательств
\usepackage{amsthm}


\newenvironment{compactlist}{
        \begin{list}{{$\bullet$}}{
                        \setlength\partopsep{0pt}
                        \setlength\parskip{0pt}
                        \setlength\parsep{0pt}
                        \setlength\topsep{0pt}
                        \setlength\itemsep{0pt}
                }
        }{
        \end{list}
}
\theoremstyle{definition}
\newtheorem{definition}{Определение}

\theoremstyle{plane}
\newtheorem{theorem}{Теорема}
\newtheorem{assertion}{Утверждение}

\theoremstyle{remark}
\newtheorem{remark}{Замечание}

\renewcommand*{\proofname}{Доказательство}
\renewcommand\qedsymbol{$\blacksquare$}

\newcommand{\R}{\mathbb{R}}
\newcommand{\N}{\mathbb{N}}
\DeclareMathOperator{\sgn}{sgn}

\begin{document}
        \include{title_page/doc}

        \tableofcontents
        \clearpage
        
        \include{formulation_of_the_problem/doc}
        \include{research_of_the_system/doc}
        \include{algorithm/doc}
        \include{examples/doc}

        \begin{thebibliography}{9}
                \bibitem{pontryagin83} Л.~С.~Понтрягин, В.~Г.~Болтянский, Р.~В.~Гамрелидзе, Е.~Ф.~Мищенко. Математическая теория оптимальных процеccов. М.: Наука, 1983.
                \bibitem{li72} Э.~Б.~Ли, Л.~Маркус. Основы теории оптимального управления. М: Наука, 1972.
        \end{thebibliography} 
\end{document}
        \documentclass[a4paper, 11pt]{article}


\usepackage{amsmath}
\usepackage{amssymb}
\usepackage{hyperref}
\usepackage{url}
\usepackage{a4wide}
\usepackage[utf8]{inputenc}
\usepackage[main = russian, english]{babel}
\usepackage[pdftex]{graphicx}
\usepackage{float}
\usepackage{subcaption}
\usepackage{indentfirst}

% Красивый внешний вид теорем, определений и доказательств
\usepackage{amsthm}


\newenvironment{compactlist}{
        \begin{list}{{$\bullet$}}{
                        \setlength\partopsep{0pt}
                        \setlength\parskip{0pt}
                        \setlength\parsep{0pt}
                        \setlength\topsep{0pt}
                        \setlength\itemsep{0pt}
                }
        }{
        \end{list}
}
\theoremstyle{definition}
\newtheorem{definition}{Определение}

\theoremstyle{plane}
\newtheorem{theorem}{Теорема}
\newtheorem{assertion}{Утверждение}

\theoremstyle{remark}
\newtheorem{remark}{Замечание}

\renewcommand*{\proofname}{Доказательство}
\renewcommand\qedsymbol{$\blacksquare$}

\newcommand{\R}{\mathbb{R}}
\newcommand{\N}{\mathbb{N}}
\DeclareMathOperator{\sgn}{sgn}

\begin{document}
        \include{title_page/doc}

        \tableofcontents
        \clearpage
        
        \include{formulation_of_the_problem/doc}
        \include{research_of_the_system/doc}
        \include{algorithm/doc}
        \include{examples/doc}

        \begin{thebibliography}{9}
                \bibitem{pontryagin83} Л.~С.~Понтрягин, В.~Г.~Болтянский, Р.~В.~Гамрелидзе, Е.~Ф.~Мищенко. Математическая теория оптимальных процеccов. М.: Наука, 1983.
                \bibitem{li72} Э.~Б.~Ли, Л.~Маркус. Основы теории оптимального управления. М: Наука, 1972.
        \end{thebibliography} 
\end{document}
        \documentclass[a4paper, 11pt]{article}


\usepackage{amsmath}
\usepackage{amssymb}
\usepackage{hyperref}
\usepackage{url}
\usepackage{a4wide}
\usepackage[utf8]{inputenc}
\usepackage[main = russian, english]{babel}
\usepackage[pdftex]{graphicx}
\usepackage{float}
\usepackage{subcaption}
\usepackage{indentfirst}

% Красивый внешний вид теорем, определений и доказательств
\usepackage{amsthm}


\newenvironment{compactlist}{
        \begin{list}{{$\bullet$}}{
                        \setlength\partopsep{0pt}
                        \setlength\parskip{0pt}
                        \setlength\parsep{0pt}
                        \setlength\topsep{0pt}
                        \setlength\itemsep{0pt}
                }
        }{
        \end{list}
}
\theoremstyle{definition}
\newtheorem{definition}{Определение}

\theoremstyle{plane}
\newtheorem{theorem}{Теорема}
\newtheorem{assertion}{Утверждение}

\theoremstyle{remark}
\newtheorem{remark}{Замечание}

\renewcommand*{\proofname}{Доказательство}
\renewcommand\qedsymbol{$\blacksquare$}

\newcommand{\R}{\mathbb{R}}
\newcommand{\N}{\mathbb{N}}
\DeclareMathOperator{\sgn}{sgn}

\begin{document}
        \include{title_page/doc}

        \tableofcontents
        \clearpage
        
        \include{formulation_of_the_problem/doc}
        \include{research_of_the_system/doc}
        \include{algorithm/doc}
        \include{examples/doc}

        \begin{thebibliography}{9}
                \bibitem{pontryagin83} Л.~С.~Понтрягин, В.~Г.~Болтянский, Р.~В.~Гамрелидзе, Е.~Ф.~Мищенко. Математическая теория оптимальных процеccов. М.: Наука, 1983.
                \bibitem{li72} Э.~Б.~Ли, Л.~Маркус. Основы теории оптимального управления. М: Наука, 1972.
        \end{thebibliography} 
\end{document}

        \begin{thebibliography}{9}
                \bibitem{pontryagin83} Л.~С.~Понтрягин, В.~Г.~Болтянский, Р.~В.~Гамрелидзе, Е.~Ф.~Мищенко. Математическая теория оптимальных процеccов. М.: Наука, 1983.
                \bibitem{li72} Э.~Б.~Ли, Л.~Маркус. Основы теории оптимального управления. М: Наука, 1972.
        \end{thebibliography} 
\end{document}
        \documentclass[a4paper, 11pt]{article}


\usepackage{amsmath}
\usepackage{amssymb}
\usepackage{hyperref}
\usepackage{url}
\usepackage{a4wide}
\usepackage[utf8]{inputenc}
\usepackage[main = russian, english]{babel}
\usepackage[pdftex]{graphicx}
\usepackage{float}
\usepackage{subcaption}
\usepackage{indentfirst}

% Красивый внешний вид теорем, определений и доказательств
\usepackage{amsthm}


\newenvironment{compactlist}{
        \begin{list}{{$\bullet$}}{
                        \setlength\partopsep{0pt}
                        \setlength\parskip{0pt}
                        \setlength\parsep{0pt}
                        \setlength\topsep{0pt}
                        \setlength\itemsep{0pt}
                }
        }{
        \end{list}
}
\theoremstyle{definition}
\newtheorem{definition}{Определение}

\theoremstyle{plane}
\newtheorem{theorem}{Теорема}
\newtheorem{assertion}{Утверждение}

\theoremstyle{remark}
\newtheorem{remark}{Замечание}

\renewcommand*{\proofname}{Доказательство}
\renewcommand\qedsymbol{$\blacksquare$}

\newcommand{\R}{\mathbb{R}}
\newcommand{\N}{\mathbb{N}}
\DeclareMathOperator{\sgn}{sgn}

\begin{document}
        \documentclass[a4paper, 11pt]{article}


\usepackage{amsmath}
\usepackage{amssymb}
\usepackage{hyperref}
\usepackage{url}
\usepackage{a4wide}
\usepackage[utf8]{inputenc}
\usepackage[main = russian, english]{babel}
\usepackage[pdftex]{graphicx}
\usepackage{float}
\usepackage{subcaption}
\usepackage{indentfirst}

% Красивый внешний вид теорем, определений и доказательств
\usepackage{amsthm}


\newenvironment{compactlist}{
        \begin{list}{{$\bullet$}}{
                        \setlength\partopsep{0pt}
                        \setlength\parskip{0pt}
                        \setlength\parsep{0pt}
                        \setlength\topsep{0pt}
                        \setlength\itemsep{0pt}
                }
        }{
        \end{list}
}
\theoremstyle{definition}
\newtheorem{definition}{Определение}

\theoremstyle{plane}
\newtheorem{theorem}{Теорема}
\newtheorem{assertion}{Утверждение}

\theoremstyle{remark}
\newtheorem{remark}{Замечание}

\renewcommand*{\proofname}{Доказательство}
\renewcommand\qedsymbol{$\blacksquare$}

\newcommand{\R}{\mathbb{R}}
\newcommand{\N}{\mathbb{N}}
\DeclareMathOperator{\sgn}{sgn}

\begin{document}
        \include{title_page/doc}

        \tableofcontents
        \clearpage
        
        \include{formulation_of_the_problem/doc}
        \include{research_of_the_system/doc}
        \include{algorithm/doc}
        \include{examples/doc}

        \begin{thebibliography}{9}
                \bibitem{pontryagin83} Л.~С.~Понтрягин, В.~Г.~Болтянский, Р.~В.~Гамрелидзе, Е.~Ф.~Мищенко. Математическая теория оптимальных процеccов. М.: Наука, 1983.
                \bibitem{li72} Э.~Б.~Ли, Л.~Маркус. Основы теории оптимального управления. М: Наука, 1972.
        \end{thebibliography} 
\end{document}

        \tableofcontents
        \clearpage
        
        \documentclass[a4paper, 11pt]{article}


\usepackage{amsmath}
\usepackage{amssymb}
\usepackage{hyperref}
\usepackage{url}
\usepackage{a4wide}
\usepackage[utf8]{inputenc}
\usepackage[main = russian, english]{babel}
\usepackage[pdftex]{graphicx}
\usepackage{float}
\usepackage{subcaption}
\usepackage{indentfirst}

% Красивый внешний вид теорем, определений и доказательств
\usepackage{amsthm}


\newenvironment{compactlist}{
        \begin{list}{{$\bullet$}}{
                        \setlength\partopsep{0pt}
                        \setlength\parskip{0pt}
                        \setlength\parsep{0pt}
                        \setlength\topsep{0pt}
                        \setlength\itemsep{0pt}
                }
        }{
        \end{list}
}
\theoremstyle{definition}
\newtheorem{definition}{Определение}

\theoremstyle{plane}
\newtheorem{theorem}{Теорема}
\newtheorem{assertion}{Утверждение}

\theoremstyle{remark}
\newtheorem{remark}{Замечание}

\renewcommand*{\proofname}{Доказательство}
\renewcommand\qedsymbol{$\blacksquare$}

\newcommand{\R}{\mathbb{R}}
\newcommand{\N}{\mathbb{N}}
\DeclareMathOperator{\sgn}{sgn}

\begin{document}
        \include{title_page/doc}

        \tableofcontents
        \clearpage
        
        \include{formulation_of_the_problem/doc}
        \include{research_of_the_system/doc}
        \include{algorithm/doc}
        \include{examples/doc}

        \begin{thebibliography}{9}
                \bibitem{pontryagin83} Л.~С.~Понтрягин, В.~Г.~Болтянский, Р.~В.~Гамрелидзе, Е.~Ф.~Мищенко. Математическая теория оптимальных процеccов. М.: Наука, 1983.
                \bibitem{li72} Э.~Б.~Ли, Л.~Маркус. Основы теории оптимального управления. М: Наука, 1972.
        \end{thebibliography} 
\end{document}
        \documentclass[a4paper, 11pt]{article}


\usepackage{amsmath}
\usepackage{amssymb}
\usepackage{hyperref}
\usepackage{url}
\usepackage{a4wide}
\usepackage[utf8]{inputenc}
\usepackage[main = russian, english]{babel}
\usepackage[pdftex]{graphicx}
\usepackage{float}
\usepackage{subcaption}
\usepackage{indentfirst}

% Красивый внешний вид теорем, определений и доказательств
\usepackage{amsthm}


\newenvironment{compactlist}{
        \begin{list}{{$\bullet$}}{
                        \setlength\partopsep{0pt}
                        \setlength\parskip{0pt}
                        \setlength\parsep{0pt}
                        \setlength\topsep{0pt}
                        \setlength\itemsep{0pt}
                }
        }{
        \end{list}
}
\theoremstyle{definition}
\newtheorem{definition}{Определение}

\theoremstyle{plane}
\newtheorem{theorem}{Теорема}
\newtheorem{assertion}{Утверждение}

\theoremstyle{remark}
\newtheorem{remark}{Замечание}

\renewcommand*{\proofname}{Доказательство}
\renewcommand\qedsymbol{$\blacksquare$}

\newcommand{\R}{\mathbb{R}}
\newcommand{\N}{\mathbb{N}}
\DeclareMathOperator{\sgn}{sgn}

\begin{document}
        \include{title_page/doc}

        \tableofcontents
        \clearpage
        
        \include{formulation_of_the_problem/doc}
        \include{research_of_the_system/doc}
        \include{algorithm/doc}
        \include{examples/doc}

        \begin{thebibliography}{9}
                \bibitem{pontryagin83} Л.~С.~Понтрягин, В.~Г.~Болтянский, Р.~В.~Гамрелидзе, Е.~Ф.~Мищенко. Математическая теория оптимальных процеccов. М.: Наука, 1983.
                \bibitem{li72} Э.~Б.~Ли, Л.~Маркус. Основы теории оптимального управления. М: Наука, 1972.
        \end{thebibliography} 
\end{document}
        \documentclass[a4paper, 11pt]{article}


\usepackage{amsmath}
\usepackage{amssymb}
\usepackage{hyperref}
\usepackage{url}
\usepackage{a4wide}
\usepackage[utf8]{inputenc}
\usepackage[main = russian, english]{babel}
\usepackage[pdftex]{graphicx}
\usepackage{float}
\usepackage{subcaption}
\usepackage{indentfirst}

% Красивый внешний вид теорем, определений и доказательств
\usepackage{amsthm}


\newenvironment{compactlist}{
        \begin{list}{{$\bullet$}}{
                        \setlength\partopsep{0pt}
                        \setlength\parskip{0pt}
                        \setlength\parsep{0pt}
                        \setlength\topsep{0pt}
                        \setlength\itemsep{0pt}
                }
        }{
        \end{list}
}
\theoremstyle{definition}
\newtheorem{definition}{Определение}

\theoremstyle{plane}
\newtheorem{theorem}{Теорема}
\newtheorem{assertion}{Утверждение}

\theoremstyle{remark}
\newtheorem{remark}{Замечание}

\renewcommand*{\proofname}{Доказательство}
\renewcommand\qedsymbol{$\blacksquare$}

\newcommand{\R}{\mathbb{R}}
\newcommand{\N}{\mathbb{N}}
\DeclareMathOperator{\sgn}{sgn}

\begin{document}
        \include{title_page/doc}

        \tableofcontents
        \clearpage
        
        \include{formulation_of_the_problem/doc}
        \include{research_of_the_system/doc}
        \include{algorithm/doc}
        \include{examples/doc}

        \begin{thebibliography}{9}
                \bibitem{pontryagin83} Л.~С.~Понтрягин, В.~Г.~Болтянский, Р.~В.~Гамрелидзе, Е.~Ф.~Мищенко. Математическая теория оптимальных процеccов. М.: Наука, 1983.
                \bibitem{li72} Э.~Б.~Ли, Л.~Маркус. Основы теории оптимального управления. М: Наука, 1972.
        \end{thebibliography} 
\end{document}
        \documentclass[a4paper, 11pt]{article}


\usepackage{amsmath}
\usepackage{amssymb}
\usepackage{hyperref}
\usepackage{url}
\usepackage{a4wide}
\usepackage[utf8]{inputenc}
\usepackage[main = russian, english]{babel}
\usepackage[pdftex]{graphicx}
\usepackage{float}
\usepackage{subcaption}
\usepackage{indentfirst}

% Красивый внешний вид теорем, определений и доказательств
\usepackage{amsthm}


\newenvironment{compactlist}{
        \begin{list}{{$\bullet$}}{
                        \setlength\partopsep{0pt}
                        \setlength\parskip{0pt}
                        \setlength\parsep{0pt}
                        \setlength\topsep{0pt}
                        \setlength\itemsep{0pt}
                }
        }{
        \end{list}
}
\theoremstyle{definition}
\newtheorem{definition}{Определение}

\theoremstyle{plane}
\newtheorem{theorem}{Теорема}
\newtheorem{assertion}{Утверждение}

\theoremstyle{remark}
\newtheorem{remark}{Замечание}

\renewcommand*{\proofname}{Доказательство}
\renewcommand\qedsymbol{$\blacksquare$}

\newcommand{\R}{\mathbb{R}}
\newcommand{\N}{\mathbb{N}}
\DeclareMathOperator{\sgn}{sgn}

\begin{document}
        \include{title_page/doc}

        \tableofcontents
        \clearpage
        
        \include{formulation_of_the_problem/doc}
        \include{research_of_the_system/doc}
        \include{algorithm/doc}
        \include{examples/doc}

        \begin{thebibliography}{9}
                \bibitem{pontryagin83} Л.~С.~Понтрягин, В.~Г.~Болтянский, Р.~В.~Гамрелидзе, Е.~Ф.~Мищенко. Математическая теория оптимальных процеccов. М.: Наука, 1983.
                \bibitem{li72} Э.~Б.~Ли, Л.~Маркус. Основы теории оптимального управления. М: Наука, 1972.
        \end{thebibliography} 
\end{document}

        \begin{thebibliography}{9}
                \bibitem{pontryagin83} Л.~С.~Понтрягин, В.~Г.~Болтянский, Р.~В.~Гамрелидзе, Е.~Ф.~Мищенко. Математическая теория оптимальных процеccов. М.: Наука, 1983.
                \bibitem{li72} Э.~Б.~Ли, Л.~Маркус. Основы теории оптимального управления. М: Наука, 1972.
        \end{thebibliography} 
\end{document}
        \documentclass[a4paper, 11pt]{article}


\usepackage{amsmath}
\usepackage{amssymb}
\usepackage{hyperref}
\usepackage{url}
\usepackage{a4wide}
\usepackage[utf8]{inputenc}
\usepackage[main = russian, english]{babel}
\usepackage[pdftex]{graphicx}
\usepackage{float}
\usepackage{subcaption}
\usepackage{indentfirst}

% Красивый внешний вид теорем, определений и доказательств
\usepackage{amsthm}


\newenvironment{compactlist}{
        \begin{list}{{$\bullet$}}{
                        \setlength\partopsep{0pt}
                        \setlength\parskip{0pt}
                        \setlength\parsep{0pt}
                        \setlength\topsep{0pt}
                        \setlength\itemsep{0pt}
                }
        }{
        \end{list}
}
\theoremstyle{definition}
\newtheorem{definition}{Определение}

\theoremstyle{plane}
\newtheorem{theorem}{Теорема}
\newtheorem{assertion}{Утверждение}

\theoremstyle{remark}
\newtheorem{remark}{Замечание}

\renewcommand*{\proofname}{Доказательство}
\renewcommand\qedsymbol{$\blacksquare$}

\newcommand{\R}{\mathbb{R}}
\newcommand{\N}{\mathbb{N}}
\DeclareMathOperator{\sgn}{sgn}

\begin{document}
        \documentclass[a4paper, 11pt]{article}


\usepackage{amsmath}
\usepackage{amssymb}
\usepackage{hyperref}
\usepackage{url}
\usepackage{a4wide}
\usepackage[utf8]{inputenc}
\usepackage[main = russian, english]{babel}
\usepackage[pdftex]{graphicx}
\usepackage{float}
\usepackage{subcaption}
\usepackage{indentfirst}

% Красивый внешний вид теорем, определений и доказательств
\usepackage{amsthm}


\newenvironment{compactlist}{
        \begin{list}{{$\bullet$}}{
                        \setlength\partopsep{0pt}
                        \setlength\parskip{0pt}
                        \setlength\parsep{0pt}
                        \setlength\topsep{0pt}
                        \setlength\itemsep{0pt}
                }
        }{
        \end{list}
}
\theoremstyle{definition}
\newtheorem{definition}{Определение}

\theoremstyle{plane}
\newtheorem{theorem}{Теорема}
\newtheorem{assertion}{Утверждение}

\theoremstyle{remark}
\newtheorem{remark}{Замечание}

\renewcommand*{\proofname}{Доказательство}
\renewcommand\qedsymbol{$\blacksquare$}

\newcommand{\R}{\mathbb{R}}
\newcommand{\N}{\mathbb{N}}
\DeclareMathOperator{\sgn}{sgn}

\begin{document}
        \include{title_page/doc}

        \tableofcontents
        \clearpage
        
        \include{formulation_of_the_problem/doc}
        \include{research_of_the_system/doc}
        \include{algorithm/doc}
        \include{examples/doc}

        \begin{thebibliography}{9}
                \bibitem{pontryagin83} Л.~С.~Понтрягин, В.~Г.~Болтянский, Р.~В.~Гамрелидзе, Е.~Ф.~Мищенко. Математическая теория оптимальных процеccов. М.: Наука, 1983.
                \bibitem{li72} Э.~Б.~Ли, Л.~Маркус. Основы теории оптимального управления. М: Наука, 1972.
        \end{thebibliography} 
\end{document}

        \tableofcontents
        \clearpage
        
        \documentclass[a4paper, 11pt]{article}


\usepackage{amsmath}
\usepackage{amssymb}
\usepackage{hyperref}
\usepackage{url}
\usepackage{a4wide}
\usepackage[utf8]{inputenc}
\usepackage[main = russian, english]{babel}
\usepackage[pdftex]{graphicx}
\usepackage{float}
\usepackage{subcaption}
\usepackage{indentfirst}

% Красивый внешний вид теорем, определений и доказательств
\usepackage{amsthm}


\newenvironment{compactlist}{
        \begin{list}{{$\bullet$}}{
                        \setlength\partopsep{0pt}
                        \setlength\parskip{0pt}
                        \setlength\parsep{0pt}
                        \setlength\topsep{0pt}
                        \setlength\itemsep{0pt}
                }
        }{
        \end{list}
}
\theoremstyle{definition}
\newtheorem{definition}{Определение}

\theoremstyle{plane}
\newtheorem{theorem}{Теорема}
\newtheorem{assertion}{Утверждение}

\theoremstyle{remark}
\newtheorem{remark}{Замечание}

\renewcommand*{\proofname}{Доказательство}
\renewcommand\qedsymbol{$\blacksquare$}

\newcommand{\R}{\mathbb{R}}
\newcommand{\N}{\mathbb{N}}
\DeclareMathOperator{\sgn}{sgn}

\begin{document}
        \include{title_page/doc}

        \tableofcontents
        \clearpage
        
        \include{formulation_of_the_problem/doc}
        \include{research_of_the_system/doc}
        \include{algorithm/doc}
        \include{examples/doc}

        \begin{thebibliography}{9}
                \bibitem{pontryagin83} Л.~С.~Понтрягин, В.~Г.~Болтянский, Р.~В.~Гамрелидзе, Е.~Ф.~Мищенко. Математическая теория оптимальных процеccов. М.: Наука, 1983.
                \bibitem{li72} Э.~Б.~Ли, Л.~Маркус. Основы теории оптимального управления. М: Наука, 1972.
        \end{thebibliography} 
\end{document}
        \documentclass[a4paper, 11pt]{article}


\usepackage{amsmath}
\usepackage{amssymb}
\usepackage{hyperref}
\usepackage{url}
\usepackage{a4wide}
\usepackage[utf8]{inputenc}
\usepackage[main = russian, english]{babel}
\usepackage[pdftex]{graphicx}
\usepackage{float}
\usepackage{subcaption}
\usepackage{indentfirst}

% Красивый внешний вид теорем, определений и доказательств
\usepackage{amsthm}


\newenvironment{compactlist}{
        \begin{list}{{$\bullet$}}{
                        \setlength\partopsep{0pt}
                        \setlength\parskip{0pt}
                        \setlength\parsep{0pt}
                        \setlength\topsep{0pt}
                        \setlength\itemsep{0pt}
                }
        }{
        \end{list}
}
\theoremstyle{definition}
\newtheorem{definition}{Определение}

\theoremstyle{plane}
\newtheorem{theorem}{Теорема}
\newtheorem{assertion}{Утверждение}

\theoremstyle{remark}
\newtheorem{remark}{Замечание}

\renewcommand*{\proofname}{Доказательство}
\renewcommand\qedsymbol{$\blacksquare$}

\newcommand{\R}{\mathbb{R}}
\newcommand{\N}{\mathbb{N}}
\DeclareMathOperator{\sgn}{sgn}

\begin{document}
        \include{title_page/doc}

        \tableofcontents
        \clearpage
        
        \include{formulation_of_the_problem/doc}
        \include{research_of_the_system/doc}
        \include{algorithm/doc}
        \include{examples/doc}

        \begin{thebibliography}{9}
                \bibitem{pontryagin83} Л.~С.~Понтрягин, В.~Г.~Болтянский, Р.~В.~Гамрелидзе, Е.~Ф.~Мищенко. Математическая теория оптимальных процеccов. М.: Наука, 1983.
                \bibitem{li72} Э.~Б.~Ли, Л.~Маркус. Основы теории оптимального управления. М: Наука, 1972.
        \end{thebibliography} 
\end{document}
        \documentclass[a4paper, 11pt]{article}


\usepackage{amsmath}
\usepackage{amssymb}
\usepackage{hyperref}
\usepackage{url}
\usepackage{a4wide}
\usepackage[utf8]{inputenc}
\usepackage[main = russian, english]{babel}
\usepackage[pdftex]{graphicx}
\usepackage{float}
\usepackage{subcaption}
\usepackage{indentfirst}

% Красивый внешний вид теорем, определений и доказательств
\usepackage{amsthm}


\newenvironment{compactlist}{
        \begin{list}{{$\bullet$}}{
                        \setlength\partopsep{0pt}
                        \setlength\parskip{0pt}
                        \setlength\parsep{0pt}
                        \setlength\topsep{0pt}
                        \setlength\itemsep{0pt}
                }
        }{
        \end{list}
}
\theoremstyle{definition}
\newtheorem{definition}{Определение}

\theoremstyle{plane}
\newtheorem{theorem}{Теорема}
\newtheorem{assertion}{Утверждение}

\theoremstyle{remark}
\newtheorem{remark}{Замечание}

\renewcommand*{\proofname}{Доказательство}
\renewcommand\qedsymbol{$\blacksquare$}

\newcommand{\R}{\mathbb{R}}
\newcommand{\N}{\mathbb{N}}
\DeclareMathOperator{\sgn}{sgn}

\begin{document}
        \include{title_page/doc}

        \tableofcontents
        \clearpage
        
        \include{formulation_of_the_problem/doc}
        \include{research_of_the_system/doc}
        \include{algorithm/doc}
        \include{examples/doc}

        \begin{thebibliography}{9}
                \bibitem{pontryagin83} Л.~С.~Понтрягин, В.~Г.~Болтянский, Р.~В.~Гамрелидзе, Е.~Ф.~Мищенко. Математическая теория оптимальных процеccов. М.: Наука, 1983.
                \bibitem{li72} Э.~Б.~Ли, Л.~Маркус. Основы теории оптимального управления. М: Наука, 1972.
        \end{thebibliography} 
\end{document}
        \documentclass[a4paper, 11pt]{article}


\usepackage{amsmath}
\usepackage{amssymb}
\usepackage{hyperref}
\usepackage{url}
\usepackage{a4wide}
\usepackage[utf8]{inputenc}
\usepackage[main = russian, english]{babel}
\usepackage[pdftex]{graphicx}
\usepackage{float}
\usepackage{subcaption}
\usepackage{indentfirst}

% Красивый внешний вид теорем, определений и доказательств
\usepackage{amsthm}


\newenvironment{compactlist}{
        \begin{list}{{$\bullet$}}{
                        \setlength\partopsep{0pt}
                        \setlength\parskip{0pt}
                        \setlength\parsep{0pt}
                        \setlength\topsep{0pt}
                        \setlength\itemsep{0pt}
                }
        }{
        \end{list}
}
\theoremstyle{definition}
\newtheorem{definition}{Определение}

\theoremstyle{plane}
\newtheorem{theorem}{Теорема}
\newtheorem{assertion}{Утверждение}

\theoremstyle{remark}
\newtheorem{remark}{Замечание}

\renewcommand*{\proofname}{Доказательство}
\renewcommand\qedsymbol{$\blacksquare$}

\newcommand{\R}{\mathbb{R}}
\newcommand{\N}{\mathbb{N}}
\DeclareMathOperator{\sgn}{sgn}

\begin{document}
        \include{title_page/doc}

        \tableofcontents
        \clearpage
        
        \include{formulation_of_the_problem/doc}
        \include{research_of_the_system/doc}
        \include{algorithm/doc}
        \include{examples/doc}

        \begin{thebibliography}{9}
                \bibitem{pontryagin83} Л.~С.~Понтрягин, В.~Г.~Болтянский, Р.~В.~Гамрелидзе, Е.~Ф.~Мищенко. Математическая теория оптимальных процеccов. М.: Наука, 1983.
                \bibitem{li72} Э.~Б.~Ли, Л.~Маркус. Основы теории оптимального управления. М: Наука, 1972.
        \end{thebibliography} 
\end{document}

        \begin{thebibliography}{9}
                \bibitem{pontryagin83} Л.~С.~Понтрягин, В.~Г.~Болтянский, Р.~В.~Гамрелидзе, Е.~Ф.~Мищенко. Математическая теория оптимальных процеccов. М.: Наука, 1983.
                \bibitem{li72} Э.~Б.~Ли, Л.~Маркус. Основы теории оптимального управления. М: Наука, 1972.
        \end{thebibliography} 
\end{document}
        \documentclass[a4paper, 11pt]{article}


\usepackage{amsmath}
\usepackage{amssymb}
\usepackage{hyperref}
\usepackage{url}
\usepackage{a4wide}
\usepackage[utf8]{inputenc}
\usepackage[main = russian, english]{babel}
\usepackage[pdftex]{graphicx}
\usepackage{float}
\usepackage{subcaption}
\usepackage{indentfirst}

% Красивый внешний вид теорем, определений и доказательств
\usepackage{amsthm}


\newenvironment{compactlist}{
        \begin{list}{{$\bullet$}}{
                        \setlength\partopsep{0pt}
                        \setlength\parskip{0pt}
                        \setlength\parsep{0pt}
                        \setlength\topsep{0pt}
                        \setlength\itemsep{0pt}
                }
        }{
        \end{list}
}
\theoremstyle{definition}
\newtheorem{definition}{Определение}

\theoremstyle{plane}
\newtheorem{theorem}{Теорема}
\newtheorem{assertion}{Утверждение}

\theoremstyle{remark}
\newtheorem{remark}{Замечание}

\renewcommand*{\proofname}{Доказательство}
\renewcommand\qedsymbol{$\blacksquare$}

\newcommand{\R}{\mathbb{R}}
\newcommand{\N}{\mathbb{N}}
\DeclareMathOperator{\sgn}{sgn}

\begin{document}
        \documentclass[a4paper, 11pt]{article}


\usepackage{amsmath}
\usepackage{amssymb}
\usepackage{hyperref}
\usepackage{url}
\usepackage{a4wide}
\usepackage[utf8]{inputenc}
\usepackage[main = russian, english]{babel}
\usepackage[pdftex]{graphicx}
\usepackage{float}
\usepackage{subcaption}
\usepackage{indentfirst}

% Красивый внешний вид теорем, определений и доказательств
\usepackage{amsthm}


\newenvironment{compactlist}{
        \begin{list}{{$\bullet$}}{
                        \setlength\partopsep{0pt}
                        \setlength\parskip{0pt}
                        \setlength\parsep{0pt}
                        \setlength\topsep{0pt}
                        \setlength\itemsep{0pt}
                }
        }{
        \end{list}
}
\theoremstyle{definition}
\newtheorem{definition}{Определение}

\theoremstyle{plane}
\newtheorem{theorem}{Теорема}
\newtheorem{assertion}{Утверждение}

\theoremstyle{remark}
\newtheorem{remark}{Замечание}

\renewcommand*{\proofname}{Доказательство}
\renewcommand\qedsymbol{$\blacksquare$}

\newcommand{\R}{\mathbb{R}}
\newcommand{\N}{\mathbb{N}}
\DeclareMathOperator{\sgn}{sgn}

\begin{document}
        \include{title_page/doc}

        \tableofcontents
        \clearpage
        
        \include{formulation_of_the_problem/doc}
        \include{research_of_the_system/doc}
        \include{algorithm/doc}
        \include{examples/doc}

        \begin{thebibliography}{9}
                \bibitem{pontryagin83} Л.~С.~Понтрягин, В.~Г.~Болтянский, Р.~В.~Гамрелидзе, Е.~Ф.~Мищенко. Математическая теория оптимальных процеccов. М.: Наука, 1983.
                \bibitem{li72} Э.~Б.~Ли, Л.~Маркус. Основы теории оптимального управления. М: Наука, 1972.
        \end{thebibliography} 
\end{document}

        \tableofcontents
        \clearpage
        
        \documentclass[a4paper, 11pt]{article}


\usepackage{amsmath}
\usepackage{amssymb}
\usepackage{hyperref}
\usepackage{url}
\usepackage{a4wide}
\usepackage[utf8]{inputenc}
\usepackage[main = russian, english]{babel}
\usepackage[pdftex]{graphicx}
\usepackage{float}
\usepackage{subcaption}
\usepackage{indentfirst}

% Красивый внешний вид теорем, определений и доказательств
\usepackage{amsthm}


\newenvironment{compactlist}{
        \begin{list}{{$\bullet$}}{
                        \setlength\partopsep{0pt}
                        \setlength\parskip{0pt}
                        \setlength\parsep{0pt}
                        \setlength\topsep{0pt}
                        \setlength\itemsep{0pt}
                }
        }{
        \end{list}
}
\theoremstyle{definition}
\newtheorem{definition}{Определение}

\theoremstyle{plane}
\newtheorem{theorem}{Теорема}
\newtheorem{assertion}{Утверждение}

\theoremstyle{remark}
\newtheorem{remark}{Замечание}

\renewcommand*{\proofname}{Доказательство}
\renewcommand\qedsymbol{$\blacksquare$}

\newcommand{\R}{\mathbb{R}}
\newcommand{\N}{\mathbb{N}}
\DeclareMathOperator{\sgn}{sgn}

\begin{document}
        \include{title_page/doc}

        \tableofcontents
        \clearpage
        
        \include{formulation_of_the_problem/doc}
        \include{research_of_the_system/doc}
        \include{algorithm/doc}
        \include{examples/doc}

        \begin{thebibliography}{9}
                \bibitem{pontryagin83} Л.~С.~Понтрягин, В.~Г.~Болтянский, Р.~В.~Гамрелидзе, Е.~Ф.~Мищенко. Математическая теория оптимальных процеccов. М.: Наука, 1983.
                \bibitem{li72} Э.~Б.~Ли, Л.~Маркус. Основы теории оптимального управления. М: Наука, 1972.
        \end{thebibliography} 
\end{document}
        \documentclass[a4paper, 11pt]{article}


\usepackage{amsmath}
\usepackage{amssymb}
\usepackage{hyperref}
\usepackage{url}
\usepackage{a4wide}
\usepackage[utf8]{inputenc}
\usepackage[main = russian, english]{babel}
\usepackage[pdftex]{graphicx}
\usepackage{float}
\usepackage{subcaption}
\usepackage{indentfirst}

% Красивый внешний вид теорем, определений и доказательств
\usepackage{amsthm}


\newenvironment{compactlist}{
        \begin{list}{{$\bullet$}}{
                        \setlength\partopsep{0pt}
                        \setlength\parskip{0pt}
                        \setlength\parsep{0pt}
                        \setlength\topsep{0pt}
                        \setlength\itemsep{0pt}
                }
        }{
        \end{list}
}
\theoremstyle{definition}
\newtheorem{definition}{Определение}

\theoremstyle{plane}
\newtheorem{theorem}{Теорема}
\newtheorem{assertion}{Утверждение}

\theoremstyle{remark}
\newtheorem{remark}{Замечание}

\renewcommand*{\proofname}{Доказательство}
\renewcommand\qedsymbol{$\blacksquare$}

\newcommand{\R}{\mathbb{R}}
\newcommand{\N}{\mathbb{N}}
\DeclareMathOperator{\sgn}{sgn}

\begin{document}
        \include{title_page/doc}

        \tableofcontents
        \clearpage
        
        \include{formulation_of_the_problem/doc}
        \include{research_of_the_system/doc}
        \include{algorithm/doc}
        \include{examples/doc}

        \begin{thebibliography}{9}
                \bibitem{pontryagin83} Л.~С.~Понтрягин, В.~Г.~Болтянский, Р.~В.~Гамрелидзе, Е.~Ф.~Мищенко. Математическая теория оптимальных процеccов. М.: Наука, 1983.
                \bibitem{li72} Э.~Б.~Ли, Л.~Маркус. Основы теории оптимального управления. М: Наука, 1972.
        \end{thebibliography} 
\end{document}
        \documentclass[a4paper, 11pt]{article}


\usepackage{amsmath}
\usepackage{amssymb}
\usepackage{hyperref}
\usepackage{url}
\usepackage{a4wide}
\usepackage[utf8]{inputenc}
\usepackage[main = russian, english]{babel}
\usepackage[pdftex]{graphicx}
\usepackage{float}
\usepackage{subcaption}
\usepackage{indentfirst}

% Красивый внешний вид теорем, определений и доказательств
\usepackage{amsthm}


\newenvironment{compactlist}{
        \begin{list}{{$\bullet$}}{
                        \setlength\partopsep{0pt}
                        \setlength\parskip{0pt}
                        \setlength\parsep{0pt}
                        \setlength\topsep{0pt}
                        \setlength\itemsep{0pt}
                }
        }{
        \end{list}
}
\theoremstyle{definition}
\newtheorem{definition}{Определение}

\theoremstyle{plane}
\newtheorem{theorem}{Теорема}
\newtheorem{assertion}{Утверждение}

\theoremstyle{remark}
\newtheorem{remark}{Замечание}

\renewcommand*{\proofname}{Доказательство}
\renewcommand\qedsymbol{$\blacksquare$}

\newcommand{\R}{\mathbb{R}}
\newcommand{\N}{\mathbb{N}}
\DeclareMathOperator{\sgn}{sgn}

\begin{document}
        \include{title_page/doc}

        \tableofcontents
        \clearpage
        
        \include{formulation_of_the_problem/doc}
        \include{research_of_the_system/doc}
        \include{algorithm/doc}
        \include{examples/doc}

        \begin{thebibliography}{9}
                \bibitem{pontryagin83} Л.~С.~Понтрягин, В.~Г.~Болтянский, Р.~В.~Гамрелидзе, Е.~Ф.~Мищенко. Математическая теория оптимальных процеccов. М.: Наука, 1983.
                \bibitem{li72} Э.~Б.~Ли, Л.~Маркус. Основы теории оптимального управления. М: Наука, 1972.
        \end{thebibliography} 
\end{document}
        \documentclass[a4paper, 11pt]{article}


\usepackage{amsmath}
\usepackage{amssymb}
\usepackage{hyperref}
\usepackage{url}
\usepackage{a4wide}
\usepackage[utf8]{inputenc}
\usepackage[main = russian, english]{babel}
\usepackage[pdftex]{graphicx}
\usepackage{float}
\usepackage{subcaption}
\usepackage{indentfirst}

% Красивый внешний вид теорем, определений и доказательств
\usepackage{amsthm}


\newenvironment{compactlist}{
        \begin{list}{{$\bullet$}}{
                        \setlength\partopsep{0pt}
                        \setlength\parskip{0pt}
                        \setlength\parsep{0pt}
                        \setlength\topsep{0pt}
                        \setlength\itemsep{0pt}
                }
        }{
        \end{list}
}
\theoremstyle{definition}
\newtheorem{definition}{Определение}

\theoremstyle{plane}
\newtheorem{theorem}{Теорема}
\newtheorem{assertion}{Утверждение}

\theoremstyle{remark}
\newtheorem{remark}{Замечание}

\renewcommand*{\proofname}{Доказательство}
\renewcommand\qedsymbol{$\blacksquare$}

\newcommand{\R}{\mathbb{R}}
\newcommand{\N}{\mathbb{N}}
\DeclareMathOperator{\sgn}{sgn}

\begin{document}
        \include{title_page/doc}

        \tableofcontents
        \clearpage
        
        \include{formulation_of_the_problem/doc}
        \include{research_of_the_system/doc}
        \include{algorithm/doc}
        \include{examples/doc}

        \begin{thebibliography}{9}
                \bibitem{pontryagin83} Л.~С.~Понтрягин, В.~Г.~Болтянский, Р.~В.~Гамрелидзе, Е.~Ф.~Мищенко. Математическая теория оптимальных процеccов. М.: Наука, 1983.
                \bibitem{li72} Э.~Б.~Ли, Л.~Маркус. Основы теории оптимального управления. М: Наука, 1972.
        \end{thebibliography} 
\end{document}

        \begin{thebibliography}{9}
                \bibitem{pontryagin83} Л.~С.~Понтрягин, В.~Г.~Болтянский, Р.~В.~Гамрелидзе, Е.~Ф.~Мищенко. Математическая теория оптимальных процеccов. М.: Наука, 1983.
                \bibitem{li72} Э.~Б.~Ли, Л.~Маркус. Основы теории оптимального управления. М: Наука, 1972.
        \end{thebibliography} 
\end{document}

        \begin{thebibliography}{9}
                \bibitem{pontryagin83} Л.~С.~Понтрягин, В.~Г.~Болтянский, Р.~В.~Гамрелидзе, Е.~Ф.~Мищенко. Математическая теория оптимальных процеccов. М.: Наука, 1983.
                \bibitem{li72} Э.~Б.~Ли, Л.~Маркус. Основы теории оптимального управления. М: Наука, 1972.
        \end{thebibliography} 
\end{document}
