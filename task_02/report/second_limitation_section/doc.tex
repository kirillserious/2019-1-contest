\section{Решение задачи при втором типе ограничения на управление}

В этом разделе рассматриваются ограничения на значения управляющих параметров вида
$$
        u(t) \in \Omega = \{\,[u_1,\,u_2]^\T\;:\; u_1 \in \R,\; u_2\in[k_1,\,k_2]\,\}.
$$
Функция Гамильтона"--~Понтрягина принимает вид:
$$
        \psi_0(u_1^2 + \alpha|u_1|) + \psi_1 x_2 + \psi_2(u_1 - x_2(1 + u_2)), \mbox{ где $\psi_0 < 0$}.
$$
Задача ее максимизации равносильна
$$
        -\psi_0(u_1^2 + \alpha|u_1|) + \psi_2(u_1 - x_2(1+u_2)) \to \max\limits_{u}.
$$
Аналогично первому случаю можно считать, что $\psi_0 = \frac{1}{2}$. Тогда задача максимизации распадается на две:
$$
\begin{aligned}
        u_1^2 + \alpha|u_1| + \psi_2 u_1 \to \max\limits_{u_1\in\R}, \\
        -\psi_2 x_2 u_2 \to \max\limits_{u_2\in[k_1,\,k_2]}.
\end{aligned}
$$
Эти задачи имеют следующие решения:
$$
\begin{aligned}
        u_1^* &= 
        \begin{cases}
                \psi_2 + \frac{\alpha}{2}, &\mbox{при $\psi_2 < -\frac{\alpha}{2}$}\\
                0, &\mbox{при $-\frac{\alpha}{2} \leqslant \psi_2 \leqslant \frac{\alpha}{2}$}\\
                \psi_2 - \frac{\alpha}{2}, &\mbox{при $\psi_2 > \frac{\alpha}{2}$},
        \end{cases}
        \\
        u_2^* &=
        \begin{cases}
                k_2, &\mbox{при $\psi_2 x_2 < 0$}\\
                [k_1,\,k_2], &\mbox{при $\psi_2 x_2 = 0$}\\
                k_1, &\mbox{при $\psi_2 x_2 > 0$}.
        \end{cases}
\end{aligned}
$$

Теорема \ref{th:first} требует, чтобы
$\psi_2^0 > \frac{\alpha}{2}$. Аналогично в некоторой окрестности нуля систему описывают соотношения (\ref{eq:firstlim_psi2}),(\ref{firstlim_x2}),(\ref{firstlim_x1}). Все так же нас будут интересовать моменты $t_1$ и $t_2$, в которые
$\psi_2(t_1) = \frac{\alpha}{2}$
и
$\psi_2(t_2) = 0$,
а также момент времени $t_3$ такой, что
$\psi_2(t_3) = -\frac{\alpha}{2}$.
Введем классификацию рассматриваемых управлений.
\begin{enumerate}
        \item Пусть $A \geqslant 0$. Тогда будем говорить, что управление реализует \textit{режим акселерации}.
        \item Пусть $A < 0$.
        \begin{enumerate}
                \item Если $t_1 > T$, то будем говорить, что управление реализует \textit{режим отсутствия торможения}.
                \item Если $t_1 < T, \; t_2 > T$, то будем говорить, что управление реализует \textit{режим слабого торможения}.
                \item Если $t_2 < T,\; t_3 > T$, то будем говорить, что управление реализует \textit{режим сильного торможения}.
                \item Если $t_3 < T$, то будем говорить, что управление реализует \textit{режим интенсивного торможения}.
        \end{enumerate}
\end{enumerate}

При этом видно, что режимы акселерации, отсутствия и слабого торможения дословно переносятся на случай второго типа ограничений на управления. В случае же сильного торможения необходимо добавить единственную проверку.

\subsection{Исследование неподвижных точек}
\begin{definition}
        Неподвижной точкой системы $x = f(x)$ называется точка $x^*$ такая, что $f(x^*) = 0$.
\end{definition}
Для рассматриваемой в рамках задачи системы (\ref{eq:main_system}) это определение эквивалентно тому, что
$$
        \left\{
        \begin{aligned}
                & x_2^* = 0 \\
                & u - x_1^* - 3 \sin(3(x_1^*)^3) - x_1^* x_2^* = 0.
        \end{aligned}
        \right.
$$
Покажем, что данная система имеет единственное решение. Рассмотрим производную второго уравнения системы $g(x_1) = u - x_1 -  3 \sin(3x_1^3)$ при условии, что $x_2^* = 0$ и $u = const$
$$
        g'(x_1) = - 1 - 27\cos(3x_1^3)x_1^2 < 0,
$$
то есть функция $g(x_1)$ монотонно убывает. Заметим также, что $\lim_{x_1\to-\infty}g(x_1) = +\infty$ и $\lim_{x_1\to+\infty}g(x_1) = -\infty$, то есть функция $g(x_1)$ имеет ровно один корень.

Рассмотрим системы, дающие оптимальные траектории
$$
        \left\{
        \begin{aligned}
                & x_2^* = 0 \\
                & \pm \alpha - x_1^* - 3 \sin(3(x_1^*)^3) = 0.
        \end{aligned}
        \right.
$$
Стоит отметить, что если точка $(x_1^*, 0)$ является решением одной системы, то точка $(-x_1^*, 0)$ будет решением для второй системы. В точке $(x_1^*, 0)$ происходит переключение, так как она лежит на прямой $x_2 = 0$. После переключения эта точка уже не будет неподвижной для новой системы, поэтому она не будет влиять на множество достижимости.
