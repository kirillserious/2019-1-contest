\subsection{Режим интенсивного торможения}

При $t \in [t_2,\,T]$
$$
        \psi_2(t) 
= 
        -\int\limits_{t_2}^{t}
                e^{(k_2+1)(t-\tau)}
        \,d\tau
=
        \frac{B(k_1+1)}{k_2+1}
        \left(
                1 - e^{(k_2+1)(t-t_2)}
        \right),
$$
откуда
$$
        t_3
=
        t_2 + \frac{1}{k_2 + 1}
        \ln \left(
                \frac{k_2 + 1}{2B(k_1 + 1)} + 1
        \right).
$$
Обозначим для краткости $C = B \frac{k_1 + 1}{k_2 + 1}$.
Теперь мы можем выписать уравнения, описывающие поведение системы на отрезке $[t_2,\,T]$:
\begin{multline}
        x_2(t)
=
        x_1^3 e^{(k_2 + 1)(t_3 - t)} +
        \int\limits_{t_3}^{t}
                e^{(k_2 + 1)(\tau - t)}
                \left(
                        C + \frac{1}{2} + Ce^{(k_2 + 1)(\tau - t_2)}
                \right)
                \,d\tau
=\\=
        x_2^3 e^{(k_2 + 1)(t_3 - t)} +
        \frac{1}{k_2 + 1}
        \left\{
                \left(
                        C + \frac{1}{2}
                \right)
                \left(
                        1 - e^{(k_2 + 1)(t_3 - t)}
                \right)
                - \frac{C}{2}
                \left(
                        e^{(k_2 + 1)(t - t_2)} -
                        e^{(k_2 + 1)(2t_3 -t - t_2)}
                \right)
        \right\}. 
\end{multline}
\begin{multline}
        x_1(t) = x_1^3 + \int\limits_{t_3}^{t} x_2(\tau)\,d\tau
=\\=
        x_1^3 + \frac{x_2^3}{k_2 + 1}
        \left(
                1 - e^{(k_2+1)(t_3-t)}
        \right)
        +
        \frac{1}{k_2+1}
        \left\{
                \left(
                        C + \frac{1}{2}
                \right)
                \left(
                        t - t_3 - \frac{1}{k_2+1}
                        \left(
                                1 - e^{(k_2+1)(t_3-t)}
                        \right)
                \right)
\right.-\\-\left.
                \frac{C}{2(k_2+1)}
                \left(
                        e^{(k_2+1)(t-t_2)}
                        -
                        2e^{(k_2+1)(t_3-t_2)}
                        +
                        e^{(k_2+1)(2t_3-t-t_2)}
                \right)
        \right\}.
\end{multline}

Таким образом, мы можем выстроить цепочку взаимосвязанных уравнений, которые будут описывать траекторию системы на всем отрезке $[0,
,T]$.
При этом в уравнения, описывающие последующий этап, подставляются результаты, полученные на предыдущем этапе.

Соотношение $x_1(T) = L$ является уравнением относительно $t_1$ и $t_2$ --- зная их, можно однозначно восстановить $\psi_1^0$ и $\psi_2^0$, необходимые для построения всей траектории. Предположим, что мы знаем $t_2$, тогда мы получаем уравнение относительно одной неизвестной, разрешим которое и восстановив все параметры системы, мы должны проверить следующие условия:
\begin{enumerate}
        \item $t_1 \leqslant t_2 \leqslant t_3 < T$,
        \item $x_2(T) \leqslant \varepsilon$.
