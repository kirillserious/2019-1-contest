\section{Постановка задачи}

Движение материальной точки на прямой описывается обыкновенным дифференциальным уравнением
\begin{equation} \label{eq:main_equation}
        \ddot x = u_1 - \dot x (1 + u_2), \; t \in [0, T],
\end{equation}
где $x \in \R$, $u = (u_1, u_2)^\top \in \R^2$. На возможные значения управляющих параметров наложены следующие ограничения:
\begin{enumerate}
        \item либо $u_1 \geqslant 0$, $u_2 \in [k_1, k_2]$, $k_2 > k_1 > 0$;
        \item либо $u_1 \in \R$, $u_2 \in [k_1, k_2]$, $k_2 > k_1 > 0$.
\end{enumerate}
Задан начальный момент времени $t_0 = 0$ и начальная позиция $x(0) = 0$, $\dot x(0) = 0$. Необходимо за счет выбора программного управления $u$ перевести материальную точку из заданной начальной позиции в такую позицию в момент времени $T$, в которой $x(T) = L > 0$, $|\dot x(T) - S| \leqslant \varepsilon$. На множестве вех реализаций программных управлений, переводящих материальную точку в указанное множество, необходимо решить задачу оптимизации:
$$
        J = \int\limits_0^T \left( u_1^2(t) + \alpha |u_1(t)| \right) dt \to \min\limits_{u(\cdot)}, \; \alpha \geqslant 0.
$$
\begin{enumerate}
        \item Необходимо написать в среде MatLab программу с пользовательским интерфейсом, которая по заданным параметрам $T$, $k_1$, $k_2$, $L$, $\varepsilon$, $S$, $\alpha$ определяет, разрешима ли задача оптимального управления (при одном из указанным двух ограничений на управления). Если задача разрешима, то программа должна построить графики компонент оптимального управления, компонент оптимальной траектории, сопряженных переменных. Кроме того, программа должна определить количество переключений найденного оптимального управления, а также моменты переключений.
        \item В соответствующем заданию отчете необходимо привести все теоретические выкладки, сделанные в ходе решения задачи оптимального управления, привести примеры построенных оптимальных управлений и траекторий (с иллюстрациями). Все вспомогательные утверждения (за исключением принципа максимума Понтрягина), указанные в отчете, должны быть доказаны. В отчете должны быть приведены примеры оптимальных траекторий для всех возможных качественно различных \textit{режимов}.
\end{enumerate}
\begin{remark}
        Алгоритм построения оптимальной траектории не должен содержать перебор по параметрам, значения которых не ограничены, а также по более чем двум скалярным параметрам.
\end{remark}