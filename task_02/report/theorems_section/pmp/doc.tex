\subsection{Принцип максимума и условия трансверсальности как необходимые условия оптимальности}

В этом разделе мы сформулируем принцип максимума для общего случая нелинейных автономных управляемых систем в задаче с подвижными концами на конечном или бесконечном интервале времени. Принцип максимума, вместе с условиями трансверсальности, является необходимым условием, которому должно удовлетворять оптимальное управление. Доказательство этих результатов можно посмотреть в \cite{li}[Гл.~5].
        
Итак, рассмотрим автономный управляемый процесс
$$
        \dot x = f(x, u),
$$
с непрерывными $f(x, u)$ и $\frac{\partial f}{\partial x}(x, u)$ в пространстве $\R^{n+m}$. Пусть $X_0$ и $X_1$ $\subset \R^n$ есть заданные начальное и целевое множества и пусть $\Omega$ есть непустое ограничивающее множество в $\R^m$. Допустимое управление $u(t) \in \Omega$ на некотором конечном интервале времени $0 \leqslant t \leqslant t_1$ есть ограниченная измеримая функция, которой соответствует траектория $x(t, x^0)$, переводящая точку $x(0, x^0) = x^0 \in X_0$ в точку $x(t_1, x^0) = x^1 \in X_1$. Конечный момент времени $t_1$, начальная точка $x^0 \in X_0$ и конечная точка $x^1 \in X_1$ меняются вместе с управлением. Класс всех допустимых управлений обозначим через $\Delta$.
        
Каждому управлению $u(t)$ ($0 \leqslant t \leqslant t_1$) в $\Delta$ с траекторией $x(t)$ поставим в соответствие критерий качества
$$
        J(u) = \int\limits_0^{t_1} f_0(x(t), u(t)) dt,
$$
где $f_0(x, u)$ и $\frac{\partial f_0}{\partial x}(x, u)$ --- непрерывные в $\R^{n+m}$ функции. Допустимое управление $u^*(t)$ из $\Delta$ является (минимальным) оптимальным, если
$$
        J(u^*) \leqslant J(u) \mbox{ для всех } u \in \Delta.
$$
Согласно \cite{li}[Гл.~5] оптимальное управление $u^*(t)$ на интервале $0 \leqslant t \leqslant t_1$ удовлетворяет принципу максимума
$$
        \hat{H}(\hat{\psi}^*(t), \hat{x}^*(t), u^*(t)) = \hat{M}(\hat{\psi}^*(t), \hat{x}^*(t)) \mbox{ почти всюду}
$$
и
$$
        \hat{M}(\hat{\psi}^*(t), \hat{x}^*(t)) \equiv 0,\; \hat{\psi}^*_0(t) \leqslant 0 \mbox{ всюду.}
$$
Здесь расширенное состояние
$$
        \hat{x}^*(t) =
        \left[
        \begin{aligned}
                &x_0^* (t) \\
                &x^*   (t)
        \end{aligned}
        \right]
$$
есть решение расширенной системы уравнений
$$
        \begin{aligned}
                &\dot x_0 = f_0(x, u),&\\
                &\dot x_i = f_i(x, u),  &i = 1, \ldots, n,
        \end{aligned}
$$
а $\hat\psi^*(t)$ --- нетривиальное решение расширенной сопряженной системы уравнений
$$
        \begin{aligned}
                &\dot\psi_0 = 0, \\
                &\dot\psi_i = - \sum\limits_{j=0}^{n}\frac{\partial f_j}{\partial x_i}(x^*(t),\: u^*(t)), & i = 1,\: \ldots ,\: n,
        \end{aligned}
$$
где последние $n$ уравнений (с $f_0 \equiv 0$) образуют сопряженную систему. Расширенная функция Гамильтона--Понтрягина имеет вид
$$
        \hat H (\hat\psi,\: \hat x,\: u) = \psi_0 \cdot f_0(x,\: u) + \psi_1 \cdot f_1(x,\: u) + \ldots + \psi_n \cdot f_n(x,\: u)
$$
и
$$
        \hat M (\hat\psi,\: \hat x) = \max\limits_{u \in \Omega}\hat H (\hat\psi,\:\hat x,\: u) \mbox{ (если $\hat M$ существует).}
$$

\texttt{Сюда надо бы ещё написать про условия трансверсальности и условия дополняющей нежесткости.}