\section{Алгоритм решения задачи}

Рассмотрим две вспомогательные системы:
$$
S^+ :
\left\{
\begin{aligned}
        & \dot x_1 = x_2 \\
        & \dot x_2 = \alpha - x_1 - 3 \sin(3x_1^3) - x_1 x_2, \\
\end{aligned}
\right.
\;\;\;\;\;\;
S^- :
\left\{
\begin{aligned}
        & \dot x_1 = x_2 \\
        & \dot x_2 = - \alpha - x_1 - 3 \sin(3x_1^3) - x_1 x_2. \\
\end{aligned}
\right.
$$
а так же сопряженную систему (\ref{eq:congugate_system})
$$
        \left\{
        \begin{aligned}
                & \dot\psi_1 = \psi_2 + 27\psi_2\cos(3x_1^3)x_1^2 + \psi_2 x_2 \\
                & \dot \psi_2 = -\psi_1 + \psi_2 x_1.
        \end{aligned}
        \right.
$$

В каждый момент времени управление $u = \alpha$ или $u = -\alpha$. Переключения управления возможны при $\psi_2 = 0$. Для построения множества достижимости найдем концы всех траекторий, удовлетворяющих теореме \ref{th:pmp}, а затем удалим из них все точки, которые не принадлежат границе.

Сформумируем алгоритм построения множества достижимости:
\begin{enumerate}
        \item Решим систему $S^+$ с нулевыми начальными условиями до времени $\tau\;:\; x^+_2(\tau) = 0$;
        \item Организуем перебор по времени первого переключения $\tau^* \in [0, \tau]$. Для каждого $\tau^*$ необходимо:
        \begin{enumerate}
                \item Решить систему $S^-$ и сопряжённую систему с начальными условиями $x^0 = x^+(\tau^*)$ и $\psi^0 = (1,\; 0)^{\top}$ до момента переключения $\tau^{**}$ (то есть до того момента, когда $\psi_2(\tau^{**}) = 0$);
                \item Решить систему $S^+$ и сопряжённую систему с начальными условиями $x^0 = x(\tau^{**})$ и $\psi^0 = \psi(\tau^{**})$ до момента переключения $\tau^{***}$ (то есть до того момента, когда $\psi_2(\tau^{***}) = 0$);
                \item Повторять пункты (a) - (b) до того момента, пока время не превысит с заданным;
        \end{enumerate}
        \item Проделаем аналогичные действия для системы $S^-$;
        \item Объединим полученные конечные точки траекторий в одну кривую;
        \item Удалим из кривой все самопересечения. Для этого начнем перебирать отрезки, начиная с точки с минимальными координатами (так как она обязательно принадлежит границе множества достижимости). Будем последовательно обходить точки построенной в предыдущем пункте ломанной и проверять пересечение последнего отрезка со всеми предыдущими отрезками. В случае, если для некоторой точки $a_n$ отзезок $(a_{n-1},\; a_{n})$ пересекает отрезок $(a_{k-1},\; a_{k})$ (при $k<n-2$), то удалим из кривой точки $a_i$, $i = \overline{k+1, n-1}$.
\end{enumerate}