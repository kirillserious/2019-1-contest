\subsection{Исследование системы}

Модифицируем систему (\ref{eq:main_equation}):
\begin{equation} \label{eq:main_system}
        \left\{
        \begin{aligned}
                & \dot x_1 = x_2 \\
                & \dot x_2 = u - x_1 - 3 \sin(3x_1^3) - x_1 x_2 \\
                & x_1(0) = 0 \\
                & x_2(0) = 0.
        \end{aligned}
        \right.
\end{equation}

Выпишем сопряженную систему, функцию Гамильтона--Понтрягина и управление, на котором достигается максимум функции Гамильтона--Понтрягина для данной задачи:
\begin{equation} \label{eq:congugate_system}
        \left\{
        \begin{aligned}
                & \dot\psi_1 = -\psi_2 - 27\psi_2\cos(3x_1^3)x_1^2 - \psi_2 x_2 \\
                & \dot \psi_2 = \psi_1 - \psi_2 x_1
        \end{aligned}
        \right.
\end{equation}
\begin{equation}
        H(\psi, x, u) = \psi_1 x_2 + \psi_2(u - x_1 - 3\sin(3x_1^3) - x_1x_2)
\end{equation}
\begin{equation}
        u = 
        \begin{cases}
                \alpha, &\psi_2 > 0 \\
                -\alpha, &\psi_2 < 0 \\
                [-\alpha,\;\alpha], &\psi_2 = 0.
        \end{cases}
\end{equation}

\begin{assertion}
        Множество достижимости $X[t_1] \subseteq X[t_2]$, $0 \leqslant t_1 \leqslant t_2$.
\end{assertion}
\begin{proof}
        Пусть $u_1(t)$ --- управление для некоторой точки  $x^* \in X[t_1]$. Тогда достаточно взять управление
        $$
                u_2(t) = 
                \begin{cases}
                        0, & 0 \leqslant t \leqslant t_2 - t_1 \\
                        u_1(t - t_2 + t_1), & t_2 - t_1 < t \leqslant t_2.
                \end{cases}
        $$
        В таком случае $x^* \in X[t_2]$.
\end{proof}

\begin{assertion}
        Компонента $\psi_2$ сопряженной системы (\ref{eq:congugate_system}) имеет конечное число нулей. Причем каждый из них является простым.
\end{assertion}
\begin{proof}
        Предположим противное: $\psi_2$ имеет счетное число нулей на ограниченом отрезке $[0, T]$. В таком случае у множества нулей $\psi_2$ есть точка накопления $\tau$, в которой $\psi_2(\tau) = \dot \psi_2(\tau) = 0$. Но тогда из системы (\ref{eq:congugate_system}) вытекает, что $\psi_1(\tau) = 0$, что протеворечит условию невырожденности $\psi$ теоремы (\ref{th:pmp}).
\end{proof}
\begin{remark}
        Получается, что особый режим для данной задачи невозможен, поэтому будем рассматривать максимальное управление $u(t)$, модифицированное на множестве меры нуль таким образом, чтобы оно совпадало всюду с соответствующим управлением
        $$
                u = \alpha \sgn \psi_2.
        $$
\end{remark}

\begin{assertion}
        Пусть $(x_1(t), x_2(t))$ --- траектория с концом $(x_1(T), x_2(T))$, лежащим на границе множества достижимости $X[T]$, $(\psi_1(t), \psi_2(t))$ --- соответствующее сопряженное решение. Пусть в некоторой точке $\tau \in [0,\; T]$ $\psi_2(\tau) = 0$. Тогда
        \begin{enumerate}
                \item если $x_2(\tau) > 0$, то $\psi_1(\tau) > 0$;
                \item если $x_2(\tau) < 0$, то $\psi(\tau) < 0$.
        \end{enumerate}
\end{assertion}
\begin{proof}
        Из принципа максимума теоремы (\ref{th:pmp}) получаем, что 
        $$
                M(\psi(\tau),x(\tau)) = \psi_1(\tau) x_2(\tau) > 0,
        $$
        так как $M(\psi(0), x(0)) = \alpha|\psi_2| \geqslant 0$ и $\psi(\tau)$ невырожден. 
\end{proof}
\begin{remark}
        При пересечении прямой $\psi_2 = 0$ происходит переключение управления следующим образом: из верхней полуплоскости происходит смена управления с $u = \alpha$ на $u = -\alpha$, из нижней --- с $u = -\alpha$ на $u = \alpha$.
\end{remark}

\begin{assertion}
        Пусть управление $u(t),\; 0 \leqslant t \leqslant T$ соответствует решению $(x_1(t), x_2(t))$ системы (\ref{eq:main_system}) с концом $(x_1(T), x_2(T))$ на границе множества достижимости $X[T]$. Пусть далее $t_1,\; t_2$ --- моменты времени из интервала $[0, T]$ такие, что $0 \leqslant t_1 \leqslant t_2 \leqslant T$. Тогда справедливы следующие утверждения:
        \begin{enumerate}
                \item если $\psi_2(t_1) = \psi_2(t_2) = 0$ и $x_2(t_1) = 0$, то $x_2(t_2) = 0$;
                \item если $\psi_2(t_1) = \psi_2(t_2) = 0$ и $x_2(t_1) \neq 0$, то $x_2(t_2) \neq 0$ и существует точка $\tau \in (t_1, t_2)$ такая, что $x_2(\tau) = 0$;
                \item если $x_2(t_1) = x_2(t_2) = 0$, $x_2(t) \neq 0$ для всех $t \in (t_1, t_2)$ и $\psi_2(t_1) = 0$, то $\psi_2(t_2) = 0$;
                \item если $x_2(t_1) = x_2(t_2) = 0$, $x_2(t) \neq 0$ для всех $t \in (t_1, t_2)$ и $\psi_2(t_1) \neq 0$, то $\psi_2(t_2) \neq 0$ и существует точка $\tau \in (t_1, t_2)$ такая, что $\psi_2(\tau) = 0$.
        \end{enumerate}
\end{assertion}
\begin{proof}
        Во всех пунктах доказательства применяется теорема (\ref{th:pmp}).
        \begin{enumerate}
                \item Предположим, что $\psi_2(t_1) = \psi_2(t_2) = 0$. Так как
                $$
                        M(\psi, x) = \psi_1x_2 + \psi_2(\alpha\psi_2-x_1-3\sin(3x_1^3)-x_1x_2) = M(\psi(0), x(0)) \geqslant 0
                $$
                для всех $t$ из интервала $0\leqslant t \leqslant T$, то $\psi_1(t_1)\psi_2(t_2) < 0$ и
                $$
                \psi_1(t_1)x_2(t_1) = \psi_1(t_2)x_2(t_2) \geqslant 0.
                $$
                Таким образом, $x_2(t_1) = 0$ тогда и только тогда, когда $x_2(t_2) = 0$. Если $x_2(t_1) \neq 0$, то $x_2(t_1)x_2(t_2) < 0$ и, таким образом, у функции $x_2(t)$ имеется только один нуль на интервале $t_1<t<t_2$. Поэтому утверждения 1) и 2) доказаны.

                \item Теперь предположим, что $x_2(t_1) = x_2(t_2) = 0$ и $x_2(t) \neq 0$ на интервале $t_1 < t < t_2$. Предположим также, что $\psi_2(t_1) = 0$. Мы уже показали, что функция $\psi_2(t)$ не обращается в нуль на интервале $t_1 < t < t_2$. Таким образом, на замкнутом интервале $t_1 \leqslant t \leqslant t_2$ функция $x_2(t) \in C^2$, причем под производными в концевых точках понимаются односторонние производные. На замкнутом интервале имеет место равенство
                $$
                        \frac{d}{dt}[x_2\psi_1+\dot x_2 \psi_2] = 0.
                $$
                Поэтому
                $$
                        \dot x_2(t_1)\psi_2(t_1) = \dot x_2(t_2)\psi_2(t_2).
                $$
                Далее, $x_2^2(t) + x_2^2(t) \neq 0$ на интервале $t_1 < t < t_2$, так как в противном случае из свойства единственности решений системы дифференциальных уравнений (\ref{eq:main_system}) вытекало бы, что $x_2(t) \equiv 0$ на этом интервале. Так как $\psi_2(t_1) = 0$, то мы находим, что $\psi_2(t_2) = 0$. Таким образом утверждение 3) доказано.

                \item Допустим, наконец, что $x_2(t_1) = x_2(t_2) = 0$, $x_2 \neq 0$, ($t_1 < t < t_2$) и $\psi_2(t_1) \neq 0$. Тогда ясно, что $\psi_2(t_2) \neq 0$. Но если функция $\psi_2(t)$ не обращается в нуль нигде на интервале $t_1 \geqslant t \geqslant t_2$, то $\dot x_2(t_1)\dot x_2(t_2) > 0$, что невозможно, так как $t_1$ и $t_2$ --- последовательные нули функции $x_2(t)$. Теорема доказана. 
        \end{enumerate}
\end{proof}
\begin{remark}
        Таким образом нули функции $x_2(t)$ являются изолированными, они либо совпадают с нулями функции $\psi_2(t)$ или никакой из нулей функции $x_2(t)$ не является нулем функции $\psi_2(t)$, но эти два множества нулей <<чередуются>>.
\end{remark}

\begin{assertion}
        Функция $\psi(t)$ положительно однородна.
\end{assertion}
\begin{proof}
        Действительно, если в сопряженную систему (\ref{eq:congugate_system}) подставить вместо  $\psi(t)$ переменную $\beta\psi(t),\; \beta > 0$, то $\beta$ будет являться общим множителем и на неё можно сократить уравнения.
\end{proof}
