\subsection{Исследование неподвижных точек}
\begin{definition}
        Неподвижной точкой системы $x = f(x)$ называется точка $x^*$ такая, что $f(x^*) = 0$.
\end{definition}
Для рассматриваемой в рамках задачи системы (\ref{eq:main_system}) это определение эквивалентно тому, что
$$
        \left\{
        \begin{aligned}
                & x_2^* = 0 \\
                & u - x_1^* - 3 \sin(3(x_1^*)^3) - x_1^* x_2^* = 0.
        \end{aligned}
        \right.
$$
Покажем, что данная система имеет единственное решение. Рассмотрим производную второго уравнения системы $g(x_1) = u - x_1 -  3 \sin(3x_1^3)$ при условии, что $x_2^* = 0$ и $u = const$
$$
        g'(x_1) = - 1 - 27\cos(3x_1^3)x_1^2 < 0,
$$
то есть функция $g(x_1)$ монотонно убывает. Заметим также, что $\lim_{x_1\to-\infty}g(x_1) = +\infty$ и $\lim_{x_1\to+\infty}g(x_1) = -\infty$, то есть функция $g(x_1)$ имеет ровно один корень.

Рассмотрим системы, дающие оптимальные траектории
$$
        \left\{
        \begin{aligned}
                & x_2^* = 0 \\
                & \pm \alpha - x_1^* - 3 \sin(3(x_1^*)^3) = 0.
        \end{aligned}
        \right.
$$
Стоит отметить, что если точка $(x_1^*, 0)$ является решением одной системы, то точка $(-x_1^*, 0)$ будет решением для второй системы. В точке $(x_1^*, 0)$ происходит переключение, так как она лежит на прямой $x_2 = 0$. После переключения эта точка уже не будет неподвижной для новой системы, поэтому она не будет влиять на множество достижимости.