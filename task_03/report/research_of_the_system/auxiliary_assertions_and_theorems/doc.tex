\subsection{Вспомогательные утверждения и теоремы}

Рассмотрим в $\R^n$ систему уравнений, описывающую управляемый процесс
\begin{equation} \label{eq:base_system}
\dot x = f(x, u),
\end{equation}
где $f(x, u)$ и $\frac{\partial f}{\partial x}(x, u)$ --- непрерывные функции, определенные в $\R^{n+m}$. Пусть $U$ --- множество всех измеримых управлений $u(t)$ на интервале $0 \leqslant t \leqslant T$, удовлетворяющих ограничению $u(t) \in \Omega \subset \R^m$, исходящее из точки $x^0$.

\theoremstyle{definition}
\begin{definition}
        Множеством достижимости задачи (\ref{eq:base_system}) $X(t, t_0, x(t_0))$ называется множество концов траекторий системы (\ref{eq:base_system}) при всевозможных допустимых управлениях $u(\cdot)$.
\end{definition}
С целью формулировки принципа максимума введем функцию Гамильтона--Понтрягина
$$
        H(\psi, x, u) = \langle \psi,\; f(x, u) \rangle.
$$

\begin{theorem}[Принцип максимума Л.~С.~Понтрягина] \label{th:pmp}
        Пусть некоторому допустимому управлению системы (\ref{eq:base_system}) $u^*(\cdot) \in U$ соответствует решение $x^*(\cdot)$ с концом $x^*(T)$, лежащем на границе множества достижимоти $X(T, t_0, x(t_0))$. Тогда существует нетривиальное сопряженное решение $\psi^*(t)$ системы
        $$
                \dot \psi(t) = -\left\langle \psi(t),\;\frac{\partial f}{\partial x} (x^*(t), u^*(t))\right\rangle,
        $$
        такое, что почти всюду выполняется принцип максимума:
        $$
                H(\psi^*, x^*, u^*) = M(\psi^*, x^*) = \sup\limits_{u(t) \in \Omega} H(\psi^*, x^*, u).
        $$
        Если же управление $u^*(t)$ ограничено, то $M(\psi^*, x^*)$ почти всюду постоянна.
\end{theorem}

Доказательство теоремы (\ref{th:pmp}) можно найти в \cite{li72}[Гл.~4, \S1].