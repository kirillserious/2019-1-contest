\section{Постановка задачи}
Задано обыкновенное дифференциальное уравнение:
\begin{equation} \label{eq:main_equation}
        \ddot x + x + 3 \sin(3x^3) + x \dot x = u,
\end{equation}
где $x \in \R, u \in \R$. На возможные значения управляющего параметра $u$ наложено ограничение: $u \in [-\alpha, \alpha],\; \alpha > 0$. Задан начальный момент времени $t_0 = 0$ и начальная позиция $x(t_0) = 0,\; \dot x(t_0) = 0$. Необходимо построить множество достижимости $X(t, t_0, x(t_0), \dot x(t_0))$ (множество пар $(x(t), \dot x(t))$) в классе программных управлений в заданный момент времени $t \geqslant t_0$.
\begin{enumerate}
        \item Необходимо написать в среде MatLab функцию \texttt{reachset(alpha,t)}, которая по заданным параметрам $\alpha > 0,\; t \geqslant t_0$ рассчитывает приближенно множество достижимости $X(t, t_0, x(t_0), \dot x(t_0))$. На выходе функции --- два массива $X$, $Y$ с упорядоченными координатами точек многоугольника, образующего границу искомого множества. Точки в этих массивах должны быть упорядочены так, чтобы результаты работы функции без дополнительной обработки можно было подавать на вход функциям визуализации (например, \texttt{plot}). Предусмотреть такой режим работы функции, при котором она возвращает также координаты линий переключения оптимального управления (с возможностью их визуализации).
        \item Необходимо реализовать функцию \texttt{reachsetdyn(alpha,t1,t2,N,filename)}, которая, используя функцию \texttt{reachset(alpha,t)}, строит множества достижимости для моментов времени $\tau_i = t_1 + \frac{(t_2 - t_1)i}{N},\; i=\overline{0,N}$. Здесь $t_2 \geqslant t_1 \geqslant t_0$, $N$ --- натуральное число. Для каждого момента времени $\tau_i$ функция должна отобразить многоугольник, аппроксимирующий границу множества достижимости. Результат работы функции должен быть сохранен в виде видео-файла \texttt{filename.avi}. Необходимо также предусмотреть вариант работы функции (при отсутствии параметра \texttt{filename}) без сохранения в файл, с выводом непосредственно на экран. Как частный случай, функция должна иметь возможность строить границу множества достижимости в один фиксированный момент времени (при $t_2 = t_1$).
        \item В соответствующем заданию отчете необходимо привести все теоретические выкладки, сделанные в ходе построения множества достижимости, описать схему алгоритма построения множества достижимости программой, привести примеры построенных множеств достижимости (с иллюстрациями), исследовать зависимость множеств достижимости от величины параметра $\alpha$. Все вспомогательные утверждения (за исключением принципа максимума Понтрягина), указанные в отчете, должны быть доказаны.
\end{enumerate}